\documentclass[6pt,a4paper, mathserif]{article} % Prepara un documento con un font grande

\usepackage{iftex}

\ifLuaTeX
  \usepackage{tikz}
\usepackage{pgfplots}

\usepackage{lipsum} % Package to generate dummy text throughout this template

\usepackage{fontspec}
\setmainfont[Ligatures=TeX]{Alegreya}

%\usepackage[sc]{mathpazo} % Use the Palatino font
%\usepackage[T1]{fontenc} % Use 8-bit encoding that has 256 glyphs
%%%%%
%\usepackage{Alegreya} %% Option 'black' gives heavier bold face 
%\renewcommand*\oldstylenums[1]{{\AlegreyaOsF #1}}

%\usepackage[euler-digits,euler-hat-accent]{eulervm}
%%%%%%
%\usepackage[utf8]{inputenc} % Consente l'uso caratteri accentati italiani
%\linespread{1.05} % Line spacing - Palatino needs more space between lines
\usepackage{amsmath, amsthm, amssymb, amsfonts}
\usepackage{microtype} % Slightly tweak font spacing for aesthetics

%%%%%%%%%%%%%%%%%%%%%%%%%%%%%%%%%%%%%%%%%%%%%
%Miei package
\usepackage[italian]{babel}
\usepackage{graphicx}		% Per le immagini

\usepackage{tabularx}		% Per le tabelle con le colonne tutte uguali
\usepackage{tabulary}		% Tabelle migliorate, nelle celle il testo va a capo da solo...
%%%%%%%%%%%%%%%%%%%%%%%%%%%%%%%%%%%%%%%%%%%%%
\usepackage[
	    %hmargin=0.18\paperwidth,% metti la larghezza del testo (margini orizzontali) al 18% del foglio
	    %textwidth=3.1\alphabet,  % http://tex.stackexchange.com/questions/59626/nicely-force-66-characters-per-line
	    hmarginratio=1:1,       % margini destro e sinistro uguali
	    top=35mm,	            % margine sopra a 32mm...
	    vmarginratio=4:5,       % quello sotto uguale (default 2:3)
	    columnsep=20pt]         % Spazio tra le colonne]
	    {geometry} % Document margins
\usepackage{multicol} % Used for the two-column layout of the document
\usepackage[hang, small,labelfont=bf,up,textfont=it,up]{caption} % Custom captions under/above floats in tables or figures
\usepackage{booktabs} % Horizontal rules in tables
\usepackage{float} % Required for tables and figures in the multi-column environment - they need to be placed in specific locations with the [H]
\usepackage[titles]{tocloft} %Pare causi meno casini con fancyhdr
\usepackage{nicefrac} % Per le frazioni tipo ⅛
\usepackage{pdfpages} % Per includere pagine intere in pdf (per la copertina)

\usepackage{lettrine} % The lettrine is the first enlarged letter at the beginning of the text
\usepackage{paralist} % Used for the compactitem environment which makes bullet points with less space between them

\usepackage{abstract} % Allows abstract customization
\renewcommand{\abstractnamefont}{\normalfont\bfseries} % Set the "Abstract" text to bold
\renewcommand{\abstracttextfont}{\normalfont\small\itshape} % Set the abstract itself to small italic text

\usepackage{listingsutf8} % Per includere codice sorgente meglio che con verbatim (e con caratteri non inglesi)
\lstset{ 
  %Preso anche questo da http://en.wikibooks.org/wiki/LaTeX/Source_Code_Listings
  %backgroundcolor=\color{white},   % choose the background color; you must add \usepackage{color} or \usepackage{xcolor}
  basicstyle=\footnotesize\ttfamily,        % the size of the fonts that are used for the code E MESSO IN MONOSPACE
  breakatwhitespace=true,         % sets if automatic breaks should only happen at whitespace
  breaklines=true,                 % sets automatic line breaking
  captionpos=b,                    % sets the caption-position to bottom
  %commentstyle=\color{mygreen},    % comment style
  %deletekeywords={...},            % if you want to delete keywords from the given language
  %escapeinside={\%*}{*)},          % if you want to add LaTeX within your code
  %extendedchars=true,              % lets you use non-ASCII characters; for 8-bits encodings only, does not work with UTF-8
  frame=l,                    % adds a frame around the code
				    %you can control the rules at the top, right, bottom, and left directly by using the four initial 
				    %letters for single rules and their upper case versions for double rules. http://mirror.hmc.edu/ctan/macros/latex/contrib/listings/listings.pdf
				    % Es frame frame=trBL ha doppia linea a sinistra e sotto, e singola a destra e sopra
  keepspaces=true,                 % keeps spaces in text, useful for keeping indentation of code (possibly needs columns=flexible)
  %keywordstyle=\color{blue},       % keyword style
  %language=Octave,                 % the language of the code
  %morekeywords={*,...},            % if you want to add more keywords to the set
  numbers=left,                    % where to put the line-numbers; possible values are (none, left, right)
  numbersep=5pt,                   % how far the line-numbers are from the code
  %numberstyle=\tiny\color{mygray}, % the style that is used for the line-numbers
  %rulecolor=\color{black},         % if not set, the frame-color may be changed on line-breaks within not-black text (e.g. comments (green here))
  showspaces=false,                % show spaces everywhere adding particular underscores; it overrides 'showstringspaces'
  showstringspaces=false,          % underline spaces within strings only
  showtabs=false,                  % show tabs within strings adding particular underscores
  stepnumber=1,                    % the step between two line-numbers. If it's 1, each line will be numbered
  %stringstyle=\color{mymauve},     % string literal style
  tabsize=2,                       % sets default tabsize to 2 spaces
  title=\lstname                   % show the filename of files included with \lstinputlisting; also try caption instead of title
}

\usepackage{titlesec} % Allows customization of titles
%\renewcommand\thesection{\Roman{section}} % Roman numerals for the sections
%\renewcommand\thesubsection{\Roman{subsection}} % Roman numerals for subsections
% \usefont {encoding} {family} {series} {shape}
%\titleformat{\section}[block]{\AlegreyaSansSC \bfseries \LARGE}{\thesection.}{1em}{} % Change the look of the section titles. Pezzi spostati \scshape\centering\bfseries
%\titleformat{\subsection}[block]{\AlegreyaSans \bfseries \Large}{\thesection.\thesubsection }{1em}{} % Change the look of the section titles

\usepackage{fancyhdr} % Headers and footers
\pagestyle{fancy} % All pages have headers and footers
\fancyhead{} % Blank out the default header
\fancyfoot{} % Blank out the default footer
\headheight=14pt % Perchè sennò continua a lamentarsi che 12pt è troppo poco e la mette a 14 lo stesso
%\fancyhead[C]{Chiappara, Labanca, Forcher - \textit{Ottica geometrica} $\bullet$ \thesection}
\fancyhead[L]{\textit{Lusiani Enrico}} % Custom header text. \nouppercase{\leftmark} per sezione, ma non ci sta
\fancyhead[R]{\textsc{\nouppercase{\leftmark}}}
\fancyfoot[RO,LE]{\thepage} % Custom footer text 

\usepackage[hidelinks]{hyperref} % For hyperlinks in the PDF
\hypersetup{
    bookmarks=true,         % show bookmarks bar?
    unicode=false,          % non-Latin characters in Acrobat’s bookmarks
   % pdftoolbar=true,        % show Acrobat’s toolbar?
   % pdfmenubar=true,        % show Acrobat’s menu?
   % pdffitwindow=false,     % window fit to page when opened
    %pdfstartview={FitH},    % fits the width of the page to the window
    pdftitle={Relazione di Elettronica},    % title
    pdfauthor={E. Lusiani},     % author
    pdfsubject={Tesi di Laurea - Anno accademico 2015-2016 - },   % subject of the document
    %pdfcreator={Creator},   % creator of the document
    %pdfproducer={Producer}, % producer of the document
    pdfkeywords={photomultipliers} {tesi} {fotomoltiplicatori} {JUNO}, % list of keywords
    %pdfnewwindow=true,      % links in new PDF window
    %colorlinks=false,       % false: boxed links; true: colored links
    linkcolor=red,          % color of internal links (change box color with linkbordercolor)
    citecolor=green,        % color of links to bibliography
    filecolor=magenta,      % color of file links
    urlcolor=cyan           % color of external links
}

\usepackage{etoolbox}

\usepackage{textgreek}

\usepackage{standalone}

\usepackage{circuitikz}

\usepackage{siunitx}

\usepackage{subcaption}


\else
  \usepackage{tikz}
\usepackage{pgfplots}

\usepackage{lipsum} % Package to generate dummy text throughout this template

%\usepackage[sc]{mathpazo} % Use the Palatino font
\usepackage{tgpagella} % TeX Gyre Pagella, versione migliorata di Palatino. Si ma bo, no
\usepackage{inconsolata} % Font monospace
%\usepackage{textcomp}
\usepackage[scale=0.98,ttdefault]{AnonymousPro}

%%%%%
%\usepackage{Alegreya} %% Option 'black' gives heavier bold face 
\usepackage{AlegreyaSans} %% Option 'black' gives heavier bold face 
%\renewcommand*\oldstylenums[1]{{\AlegreyaOsF #1}}
%\usepackage{opensans}
%\usepackage[euler-digits,euler-hat-accent]{eulervm}
\usepackage[euler-hat-accent]{eulervm}
%%%%%%

\usepackage[T1]{fontenc} % Use 8-bit encoding that has 256 glyphs
%\usepackage{fontspec}
%\setmainfont{tgpagella}
%\setsansfont{AlegreyaSans}
%\setmonofont{inconsolata}

\usepackage[utf8]{inputenc} % Consente l'uso caratteri accentati italiani
\linespread{1.08} % Line spacing - Palatino needs more space between lines, messo a 1.08 da 1.11 che era per alegreya
\usepackage{amsmath, amsthm, amssymb, amsfonts}
\usepackage[italian]{babel}
%\usepackage[kerning,spacing,tracking,letterspace = 2,babel]{microtype} % Slightly tweak font spacing for aesthetics. Il tre è pensato per Alegreya
\usepackage[kerning,spacing,babel]{microtype}
\SetTracking[]{encoding = *,shape = *}{3} % Aumenta la distanza fra le lettere
							     % http://tex.stackexchange.com/questions/66494/new-command-for-spacing-letters-in-microtype
%%%%%%%%%%%%%%%%%%%%%%%%%%%%%%%%%%%%%%%%%%%%%
%Miei package



\usepackage{graphicx}		% Per le immagini
%\usepackage{color}		% COLORI!

%\definecolor{grigio-molto-scuro}{gray}{0.1}	%colore

\usepackage{tabularx}		% Per le tabelle con le colonne tutte uguali
\usepackage{tabulary}		% Tabelle migliorate, nelle celle il testo va a capo da solo...
%%%%%%%%%%%%%%%%%%%%%%%%%%%%%%%%%%%%%%%%%%%%%
\newlength{\alphabet}
\settowidth{\alphabet}{\normalfont abcdefghijklmnopqrstuvwxyz}
\usepackage[
	    %hmargin=0.18\paperwidth,% metti la larghezza del testo (margini orizzontali) al 18% del foglio
	    textwidth=3.3\alphabet,  % http://tex.stackexchange.com/questions/59626/nicely-force-66-characters-per-line
	    hmarginratio=1:1,       % margini destro e sinistro uguali
	    top=35mm,	            % margine sopra a 32mm...
	    vmarginratio=4:5,       % quello sotto uguale (default 2:3)
	    columnsep=20pt]         % Spazio tra le colonne?
	    {geometry} % Document margins
\usepackage{multicol} % Used for the two-column layout of the document
\usepackage[hang, small,labelfont=bf,up,textfont=it,up]{caption} % Custom captions under/above floats in tables or figures
\usepackage{booktabs} % Horizontal rules in tables
\usepackage{float} % Required for tables and figures in the multi-column environment - they need to be placed in specific locations with the [H] (e.g. \begin{table}[H])
%\usepackage{tocloft} % Per customizzare le liste di floats (per i custom float!)
\usepackage[titles]{tocloft} %Pare causi meno casini con fancyhdr
\usepackage{nicefrac} % Per le frazioni tipo ⅛
\usepackage{pdfpages} % Per includere pagine intere in pdf (per la copertina)

\usepackage{lettrine} % The lettrine is the first enlarged letter at the beginning of the text
\usepackage{paralist} % Used for the compactitem environment which makes bullet points with less space between them
\usepackage[section]{placeins} % Per \FloatBarrier. L'opzione section comporta che le sezioni siano floatbarriers

\usepackage{abstract} % Allows abstract customization
\renewcommand{\abstractnamefont}{\normalfont\bfseries} % Set the "Abstract" text to bold
\renewcommand{\abstracttextfont}{\normalfont\small\itshape} % Set the abstract itself to small italic text

\usepackage{caption} % Per captions avanzate

\usepackage{listingsutf8} % Per includere codice sorgente meglio che con verbatim (e con caratteri non inglesi)
\lstset{ 
  %Preso anche questo da http://en.wikibooks.org/wiki/LaTeX/Source_Code_Listings
  %backgroundcolor=\color{white},   % choose the background color; you must add \usepackage{color} or \usepackage{xcolor}
  basicstyle=\footnotesize\ttfamily,        % the size of the fonts that are used for the code E MESSO IN MONOSPACE
  breakatwhitespace=true,         % sets if automatic breaks should only happen at whitespace
  breaklines=true,                 % sets automatic line breaking
  captionpos=b,                    % sets the caption-position to bottom
  %commentstyle=\color{mygreen},    % comment style
  %deletekeywords={...},            % if you want to delete keywords from the given language
  %escapeinside={\%*}{*)},          % if you want to add LaTeX within your code
  %extendedchars=true,              % lets you use non-ASCII characters; for 8-bits encodings only, does not work with UTF-8
  frame=l,                    % adds a frame around the code
				    %you can control the rules at the top, right, bottom, and left directly by using the four initial 
				    %letters for single rules and their upper case versions for double rules. http://mirror.hmc.edu/ctan/macros/latex/contrib/listings/listings.pdf
				    % Es frame frame=trBL ha doppia linea a sinistra e sotto, e singola a destra e sopra
  keepspaces=true,                 % keeps spaces in text, useful for keeping indentation of code (possibly needs columns=flexible)
  %keywordstyle=\color{blue},       % keyword style
  %language=Octave,                 % the language of the code
  %morekeywords={*,...},            % if you want to add more keywords to the set
  numbers=left,                    % where to put the line-numbers; possible values are (none, left, right)
  numbersep=5pt,                   % how far the line-numbers are from the code
  %numberstyle=\tiny\color{mygray}, % the style that is used for the line-numbers
  %rulecolor=\color{black},         % if not set, the frame-color may be changed on line-breaks within not-black text (e.g. comments (green here))
  showspaces=false,                % show spaces everywhere adding particular underscores; it overrides 'showstringspaces'
  showstringspaces=false,          % underline spaces within strings only
  showtabs=false,                  % show tabs within strings adding particular underscores
  stepnumber=1,                    % the step between two line-numbers. If it's 1, each line will be numbered
  %stringstyle=\color{mymauve},     % string literal style
  tabsize=2,                       % sets default tabsize to 2 spaces
  title=\lstname                   % show the filename of files included with \lstinputlisting; also try caption instead of title
}


\usepackage{titlesec} % Allows customization of titles
%\renewcommand\thesection{\Roman{section}} % Roman numerals for the sections
%\renewcommand\thesubsection{\Roman{subsection}} % Roman numerals for subsections
% \usefont {encoding} {family} {series} {shape}
%\AlegreyaSansSC
\titleformat{\section}[block]{ \bfseries \LARGE}{\thesection.}{1em}{} % Change the look of the section titles. Pezzi spostati \scshape\centering\bfseries
\titleformat{\subsection}[block]{\bfseries \Large}{\thesection.\thesubsection }{1em}{} % Change the look of the section titles

\usepackage{fancyhdr} % Headers and footers
\pagestyle{fancy} % All pages have headers and footers
\fancyhead{} % Blank out the default header
\fancyfoot{} % Blank out the default footer
\headheight=14pt % Perchè sennò continua a lamentarsi che 12pt è troppo poco e la mette a 14 lo stesso
%\fancyhead[C]{Chiappara, Labanca, Forcher - \textit{Ottica geometrica} $\bullet$ \thesection}
\fancyhead[L]{\textit{Gruppo 8}} % Custom header text. \nouppercase{\leftmark} per sezione, ma non ci sta
\fancyhead[R]{\textsc{\nouppercase{\leftmark}}}
\fancyfoot[RO,LE]{\thepage} % Custom footer text

\usepackage[hidelinks]{hyperref} % For hyperlinks in the PDF
\hypersetup{
    bookmarks=true,         % show bookmarks bar?
    unicode=false,          % non-Latin characters in Acrobat’s bookmarks
   % pdftoolbar=true,        % show Acrobat’s toolbar?
   % pdfmenubar=true,        % show Acrobat’s menu?
   % pdffitwindow=false,     % window fit to page when opened
    %pdfstartview={FitH},    % fits the width of the page to the window
    pdftitle={Relazione di laboratorio: Distribuzioni angolari},    % title
    pdfauthor={D.Chiappara, I. Di Terlizzi, E. Lusiani},     % author
    pdfsubject={Relazione di laboratorio},   % subject of the document
    %pdfcreator={Creator},   % creator of the document
    %pdfproducer={Producer}, % producer of the document
    %pdfkeywords={photomultipliers} {tesi} {fotomoltiplicatori} {JUNO}, % list of keywords
    %pdfnewwindow=true,      % links in new PDF window
    %colorlinks=false,       % false: boxed links; true: colored links
    linkcolor=red,          % color of internal links (change box color with linkbordercolor)
    citecolor=green,        % color of links to bibliography
    filecolor=magenta,      % color of file links
    urlcolor=cyan           % color of external links
}

\usepackage{etoolbox}

\usepackage{textgreek}

\usepackage{standalone}

%\usepackage{circuitikz}

\usepackage{siunitx}

\usepackage{subcaption}

%\extrafloats{100}

\fi
\DeclareGraphicsExtensions{.pdf, .png, .jpg} % Se due immagini hanno lo stesso nome sceglile secondo l'ordine di filetype qui
\graphicspath{ {../img/nofloat/} }					 % Path delle immagini 

%%%%%%%%%%%%%%%%%%%%%%%%%%%%%%%%%%%%%%5%%%%%%%%%%%%%%%%%%%%%%%%%%%%%%%%%%%%%%%%%%%
%\usepackage{float}
%\usepackage{caption}
%\usepackage{multirow}
%\usepackage[top=3.6cm, bottom=1.5in, left=0.5in, right=0.5in]{geometry}

%%%%%%%%%%%%%%%%%
% Robe del package tocloft per fare gli indici delle mie tabelle e grafici
% texblog.org/2008/07/13/define-your-own-list-of/
%\newcommand{\listtabellaname}{Lista delle tabelle}
%\newlistof{tabella}{tab}{\listtabellaname}


%%%%%%%%%%%%%%%%%

% I miei stili di float, con le righe
\floatstyle{plaintop}
\newfloat{tabella}{tb}{lop} 
\floatname{tabella}{Tabella}

\floatstyle{ruled}
\newfloat{grafico}{tb}{loi} 
\floatname{grafico}{Grafico}

\newcommand{\tabellaautorefname}{\bfseries Tabella} % per \autoref del package hyperref
\newcommand{\graficoautorefname}{\bfseries Grafico} % Idem



%%%%%%%%%%%%%%%%%%%%%%%%%%%%%%%%%%%%%%%%%%%%%%%%%%%%%%%%%%%%%%%%%%%%%5%%%%%%%%%%%%%
% Comandi personalizzati

% \newcommand{\cm}{\,\mathrm{cm}}
% \DeclareMathOperator{\cov}{cov} % Covarianza
% \DeclareMathOperator{\var}{var} % Covarianza
% \newcommand{\mm}{\,\mathrm{mm}}
% \newcommand{\nm}{\,\mathrm{nm}}
% \newcommand{\usuq}{\nicefrac{1}{q}}
% \newcommand{\usup}{\nicefrac{1}{p}}


\newtoggle{draft}
\togglefalse{draft}

\newcommand{\toprof}[1]
{
  \iftoggle{draft}{\emph{#1}}{}
}

\newcommand{\dd}{\mathop{}\,\mathrm{d}}

\newcommand\blankpage{%
    \null
    \thispagestyle{empty}%
    \addtocounter{page}{-1}%
    \newpage}
    
\newcommand{\isotope}[2]
{%
	$^{#2}\mathrm{#1}$%
}



%////////////////////////////////////////////////////////////////////////////////////////////////////////////////////////////
%////////////////////////////////////////////////////////////////////////////////////////////////////////////////////////////
% Fine dei dati iniziali per il latex: il documento finale inizierà da qui
\begin{document}


\begin{titlepage}

\begin{center}
\LARGE{Università degli Studi di Padova}\\
\line(1,0){450}\\
\vspace{1em}
\Huge{\textsc{\textbf{Relazione di laboratorio:\\ timing rapido}}}\\
\vspace{2em}
\LARGE{\textit{Laboratorio di fisica, primo anno LM}}\\
\vspace{4em}
\huge{\textbf\textsc\textit{{{Davide Chiappara}}}}\\
\vspace{0.5em}
\normalsize{Università di Padova, facoltà di fisica,}\\
\normalsize{davide.chiappara@studenti.unipd.it}\\
\normalsize{Matricola: 1153465}\\
\vspace{1em}
\huge{\textbf\textsc\textit{{{Ivan Di Terlizzi}}}}\\
\vspace{0.5em}
\normalsize{Università di Padova, facoltà di fisica,}\\
\normalsize{ivan.diterlizzi@studenti.unipd.it}\\
\normalsize{Matricola: 1155188}\\
\vspace{1em}
\huge{\textbf\textsc\textit{{{Enrico Lusiani}}}}\\
\vspace{0.5em}
\normalsize{Università di Padova, facoltà di fisica,}\\
\normalsize{enrico.lusiani@studenti.unipd.it}\\
\normalsize{Matricola: 1153399}\\
%\vspace{7em}
\vfill
\line(1,0){450}\\
\LARGE{Anno accademico 2016-2017}
\end{center}

\end{titlepage}



{
  %\color{grigio-molto-scuro}
  %\lsstyle % Abilita il letterspacing personalizzato
  %\unclfamily % Cagata per il font Uncial
  \vspace{ \stretch{1}
}

%\blankpage


\vspace{ \stretch{1} }
\noindent
\begin{abstract}
La seguente è la relazione sull'esperimento di distribuzioni angolari eseguito da Chiappara Davide, Di Terlizzi Ivan e Lusiani Enrico facenti parte del gruppo 8. I dati sono
stati raccolti presso il laboratorio di fisica in via Loredan in data 5-6-7 Novembrebre 2016, e sono stati successivamente analizzati durante lo stesso anno accademico.\\
L'esperienza consiste nella misura della correlazione angolare di una sorgente gamma di \isotope{Co}{60} tramite rivelatori a scintallazione di NaI(Tl) di cui si misura inoltre l'efficienza.
\end{abstract}


\blankpage


\microtypesetup{protrusion=false} % disables protrusion locally in the document
\tableofcontents % prints Table of Contents
%\listoftabella
%\listoffigures
%\listoftables
\microtypesetup{protrusion=true} % enables protrusion

\blankpage

%chapters

\section{Esecuzione esperimento}
L'apparato sperimentale consiste in una serie di moduli NIM (un generatore di alta tensione per alimentare i due PMT, un fan in/out, un CFTD, un TAC, una scatola di ritardi e una coincidence unit), due scintillatore di NaI(Tl) collegati ciascuno ad un PMT XP2020, un oscilloscopio e un digitizer CAEN DT5720.\\

Durante la prima giornata si sono analizzate le varie parti dell'apparato strumentale e si sono calibrati i sistemi di acquisizione. Per prima cosa si sono collegate le uscite dei due rivelatori al fan in/out e da lì all'oscilloscopio, e si è analizzata la forma (polarità, ampiezza media e tempi caratteristici) dei due segnali. Si è inoltre identificata l'ampiezza caratteristica dei segnali corrispondenti al fotone da 1333~keV.\\

Subito dopo si è passati all'analisi del segnale del CFTD. Si è perciò collegato un uscita del fan in/out (su ciascun segnale) all'entrate del CFTD e le uscite prompt e delayed di quest ultimo all'oscilloscopio. Triggherando sul segnale di prompt si è analizzato l'effetto dei microswitch sul segnale delayed.\\

Per evitare che il CFTD scattasse sul rumore bianco dell'elettronica è stata poi settata la soglia del modulo. Si è collegata un uscita del fan in/out all'oscilloscopio, triggherando sull'uscita delayed del CFTD. Tramite l'uso della funzione ''persistenza'' dell'oscilloscopio si è regolata la soglia facendo in modo che in corrispondenza del trigger i segnali avessero tutti un ampiezza minima che li identificasse come eventi reali. Il procedimento è stato ripetuto per il secondo rivelatore.\\

Per la calibrazione in energia si è preso uno spettro con un campione di \isotope{Co}{60}, mandando il segnale del CFTD alla coincidence unit settata in modalità ''OR'' (ovvero semplicemente il segnale stesso), che in precedenza era stata collegata all'entrate TRG IN del digitizer. Le misure sono state acquisite per 10~min su ogni rivelatore. Dato che i fotoni del decadimento del \isotope{Co}{60} hanno energie molto alte e vicine tra loro, è stato necessario acquisire anche uno spettro con un campione di \isotope{Am}{241}, che contiene un fotone di energia di 59.5~keV, per eliminare la forte correlazione che si avrebbe in caso contrario tra i parametri della retta del fit.\\

Si è poi verificato che i segnali di CFTD si trovassero effettivamente sovrapposti in presenza di una coincidenza, trovando che effettivamente lo erano, e non è stato perciò necessario cambiare il ritardo del segnale delayed.\\

In preparazione ai giorni seguenti, si è definita la geometria dell'apparato. Le distanze dalla sorgente e le aree sottese dai rivelatori sono infatti necessarie sia per una stima dell'accettanza dei rivelatori, sia per una buona analisi delle misure della correlazione angolare eseguite il terzo giorno. Subito dopo è stato preso un campione di prova con i rivelatori a 180$^\circ$ l'uno dall'altro, per ottenere una misura della rate da confrontare con quella teorica ricavabile dai parametri geometrici appena misurati.\\

Durante la seconda giornata si sono completate le misure della geometria dell'apparato ed eseguite misure riguardanti l'efficienza dei due rivelatori. Nella prima parte si è cercata la posizione della sorgente rispetto all'asse di rotazione del braccio dell'apparato contenente il rivelatore 2. Per fare ciò si è posto il trigger del digitizer sul CFTD di tale rivelatore e si sono presi campioni da 10~min l'uno facendo variare l'angolo del braccio a 0, 20, 40, 50, 70 e 90$^\circ$. Dalle differenze delle rate misurate è possibile ricavare una stima della posizione della sorgente.\\

Una volta conosciuta la struttura precisa dell'apparato si è passati a misure dell'efficienza dei rivelatori. Questa misura è stata fatta utilizzando sia il metodo dei due fotoni, sia con il metodo del picco somma. Entrambe le misure hanno richiesto run di circa 60/90~min, con il trigger sulla coincidence unit in modalità ''OR'', ma mentre nella prima si cercavano gli eventi in cui un fotone era stato rivelato dal primo e uno dal secondo rivelatore, nella seconda si cercavano gli eventi in cui entrambi i fotoni erano stati raccolti dallo stesso rivelatore. Dato che quest'ultimo evento è molto raro e nello spettro in energia si trova sommerso dal rumore si è deciso che sarebbe stata presa anche una run notturna per avere una campione dalla statistica molto alta.\\

Si sono poi cominciate a prendere le misure per la correlazione angolare dei due fotoni, poi completate il giorno seguente. Tali misure sono state prese con il trigger sulla coincidence unit in modalità ''AND'', con una durata di 10~min per ogni run, facendo variare l'angolo del braccio dell'apparato di 10 in 10$^\circ$ tra 0 e 90$^\circ$. Grazie alla misura della rate di coincidenze al variare dell'angolo, si ha una stima dei parametri della funzione di correlazione angolare.\\

\FloatBarrier


\input{sections/chapters/PLACEHOLDER.txt}
\FloatBarrier


\clearpage
\begin{appendix}
%appendix

\section{Interpolazioni gaussiane}
\begin{tikzpicture}
\pgfdeclareplotmark{cross} {
\pgfpathmoveto{\pgfpoint{-0.3\pgfplotmarksize}{\pgfplotmarksize}}
\pgfpathlineto{\pgfpoint{+0.3\pgfplotmarksize}{\pgfplotmarksize}}
\pgfpathlineto{\pgfpoint{+0.3\pgfplotmarksize}{0.3\pgfplotmarksize}}
\pgfpathlineto{\pgfpoint{+1\pgfplotmarksize}{0.3\pgfplotmarksize}}
\pgfpathlineto{\pgfpoint{+1\pgfplotmarksize}{-0.3\pgfplotmarksize}}
\pgfpathlineto{\pgfpoint{+0.3\pgfplotmarksize}{-0.3\pgfplotmarksize}}
\pgfpathlineto{\pgfpoint{+0.3\pgfplotmarksize}{-1.\pgfplotmarksize}}
\pgfpathlineto{\pgfpoint{-0.3\pgfplotmarksize}{-1.\pgfplotmarksize}}
\pgfpathlineto{\pgfpoint{-0.3\pgfplotmarksize}{-0.3\pgfplotmarksize}}
\pgfpathlineto{\pgfpoint{-1.\pgfplotmarksize}{-0.3\pgfplotmarksize}}
\pgfpathlineto{\pgfpoint{-1.\pgfplotmarksize}{0.3\pgfplotmarksize}}
\pgfpathlineto{\pgfpoint{-0.3\pgfplotmarksize}{0.3\pgfplotmarksize}}
\pgfpathclose
\pgfusepathqstroke
}
\pgfdeclareplotmark{cross*} {
\pgfpathmoveto{\pgfpoint{-0.3\pgfplotmarksize}{\pgfplotmarksize}}
\pgfpathlineto{\pgfpoint{+0.3\pgfplotmarksize}{\pgfplotmarksize}}
\pgfpathlineto{\pgfpoint{+0.3\pgfplotmarksize}{0.3\pgfplotmarksize}}
\pgfpathlineto{\pgfpoint{+1\pgfplotmarksize}{0.3\pgfplotmarksize}}
\pgfpathlineto{\pgfpoint{+1\pgfplotmarksize}{-0.3\pgfplotmarksize}}
\pgfpathlineto{\pgfpoint{+0.3\pgfplotmarksize}{-0.3\pgfplotmarksize}}
\pgfpathlineto{\pgfpoint{+0.3\pgfplotmarksize}{-1.\pgfplotmarksize}}
\pgfpathlineto{\pgfpoint{-0.3\pgfplotmarksize}{-1.\pgfplotmarksize}}
\pgfpathlineto{\pgfpoint{-0.3\pgfplotmarksize}{-0.3\pgfplotmarksize}}
\pgfpathlineto{\pgfpoint{-1.\pgfplotmarksize}{-0.3\pgfplotmarksize}}
\pgfpathlineto{\pgfpoint{-1.\pgfplotmarksize}{0.3\pgfplotmarksize}}
\pgfpathlineto{\pgfpoint{-0.3\pgfplotmarksize}{0.3\pgfplotmarksize}}
\pgfpathclose
\pgfusepathqfillstroke
}
\pgfdeclareplotmark{newstar} {
\pgfpathmoveto{\pgfqpoint{0pt}{\pgfplotmarksize}}
\pgfpathlineto{\pgfqpointpolar{44}{0.5\pgfplotmarksize}}
\pgfpathlineto{\pgfqpointpolar{18}{\pgfplotmarksize}}
\pgfpathlineto{\pgfqpointpolar{-20}{0.5\pgfplotmarksize}}
\pgfpathlineto{\pgfqpointpolar{-54}{\pgfplotmarksize}}
\pgfpathlineto{\pgfqpointpolar{-90}{0.5\pgfplotmarksize}}
\pgfpathlineto{\pgfqpointpolar{234}{\pgfplotmarksize}}
\pgfpathlineto{\pgfqpointpolar{198}{0.5\pgfplotmarksize}}
\pgfpathlineto{\pgfqpointpolar{162}{\pgfplotmarksize}}
\pgfpathlineto{\pgfqpointpolar{134}{0.5\pgfplotmarksize}}
\pgfpathclose
\pgfusepathqstroke
}
\pgfdeclareplotmark{newstar*} {
\pgfpathmoveto{\pgfqpoint{0pt}{\pgfplotmarksize}}
\pgfpathlineto{\pgfqpointpolar{44}{0.5\pgfplotmarksize}}
\pgfpathlineto{\pgfqpointpolar{18}{\pgfplotmarksize}}
\pgfpathlineto{\pgfqpointpolar{-20}{0.5\pgfplotmarksize}}
\pgfpathlineto{\pgfqpointpolar{-54}{\pgfplotmarksize}}
\pgfpathlineto{\pgfqpointpolar{-90}{0.5\pgfplotmarksize}}
\pgfpathlineto{\pgfqpointpolar{234}{\pgfplotmarksize}}
\pgfpathlineto{\pgfqpointpolar{198}{0.5\pgfplotmarksize}}
\pgfpathlineto{\pgfqpointpolar{162}{\pgfplotmarksize}}
\pgfpathlineto{\pgfqpointpolar{134}{0.5\pgfplotmarksize}}
\pgfpathclose
\pgfusepathqfillstroke
}
\definecolor{c}{rgb}{1,1,1};
\draw [color=c, fill=c] (0,0) rectangle (20,13.2794);
\draw [color=c, fill=c] (2,1.32794) rectangle (18,11.9514);
\definecolor{c}{rgb}{0,0,0};
\draw [c,line width=0.9] (2,1.32794) -- (2,11.9514) -- (18,11.9514) -- (18,1.32794) -- (2,1.32794);
\definecolor{c}{rgb}{1,1,1};
\draw [color=c, fill=c] (2,1.32794) rectangle (18,11.9514);
\definecolor{c}{rgb}{0,0,0};
\draw [c,line width=0.9] (2,1.32794) -- (2,11.9514) -- (18,11.9514) -- (18,1.32794) -- (2,1.32794);
\definecolor{c}{rgb}{0,0,0.6};
\draw [c,line width=0.9] (2,3.69907) -- (2.13445,3.69907) -- (2.13445,3.84727) -- (2.26891,3.84727) -- (2.26891,3.86121) -- (2.40336,3.86121) -- (2.40336,4.03161) -- (2.53782,4.03161) -- (2.53782,4.17206) -- (2.67227,4.17206) -- (2.67227,4.2082) --
 (2.80672,4.2082) -- (2.80672,4.49633) -- (2.94118,4.49633) -- (2.94118,4.612) -- (3.07563,4.612) -- (3.07563,4.67913) -- (3.21008,4.67913) -- (3.21008,4.69617) -- (3.34454,4.69617) -- (3.34454,4.98326) -- (3.47899,4.98326) -- (3.47899,5.10719) --
 (3.61345,5.10719) -- (3.61345,5.29205) -- (3.7479,5.29205) -- (3.7479,5.45057) -- (3.88235,5.45057) -- (3.88235,5.56365) -- (4.01681,5.56365) -- (4.01681,5.83526) -- (4.15126,5.83526) -- (4.15126,5.97313) -- (4.28571,5.97313) -- (4.28571,6.14818) --
 (4.42017,6.14818) -- (4.42017,6.34388) -- (4.55462,6.34388) -- (4.55462,6.55455) -- (4.68908,6.55455) -- (4.68908,6.7234) -- (4.82353,6.7234) -- (4.82353,6.99346) -- (4.95798,6.99346) -- (4.95798,7.17884) -- (5.09244,7.17884) -- (5.09244,7.4458) --
 (5.22689,7.4458) -- (5.22689,7.67145) -- (5.36134,7.67145) -- (5.36134,7.92963) -- (5.4958,7.92963) -- (5.4958,8.10674) -- (5.63025,8.10674) -- (5.63025,8.39693) -- (5.76471,8.39693) -- (5.76471,8.45838) -- (5.89916,8.45838) -- (5.89916,8.80486) --
 (6.03361,8.80486) -- (6.03361,8.90658) -- (6.16807,8.90658) -- (6.16807,9.17044) -- (6.30252,9.17044) -- (6.30252,9.23706) -- (6.43698,9.23706) -- (6.43698,9.49317) -- (6.57143,9.49317) -- (6.57143,9.67545) -- (6.70588,9.67545) -- (6.70588,9.87425)
 -- (6.84034,9.87425) -- (6.84034,10.1923) -- (6.97479,10.1923) -- (6.97479,10.2037) -- (7.10924,10.2037) -- (7.10924,10.4257) -- (7.2437,10.4257) -- (7.2437,10.5249) -- (7.37815,10.5249) -- (7.37815,10.7149) -- (7.51261,10.7149) -- (7.51261,10.9137)
 -- (7.64706,10.9137) -- (7.64706,11.0046) -- (7.78151,11.0046) -- (7.78151,11.1244) -- (7.91597,11.1244) -- (7.91597,11.1063) -- (8.05042,11.1063) -- (8.05042,11.286) -- (8.18487,11.286) -- (8.18487,11.3495) -- (8.31933,11.3495) -- (8.31933,11.3531)
 -- (8.45378,11.3531) -- (8.45378,11.4125) -- (8.58823,11.4125) -- (8.58823,11.443) -- (8.72269,11.443) -- (8.72269,11.4259) -- (8.85714,11.4259) -- (8.85714,11.4455) -- (8.9916,11.4455) -- (8.9916,11.1775) -- (9.12605,11.1775) -- (9.12605,11.2137)
 -- (9.2605,11.2137) -- (9.2605,11.2142) -- (9.39496,11.2142) -- (9.39496,11.0082) -- (9.52941,11.0082) -- (9.52941,10.8925) -- (9.66387,10.8925) -- (9.66387,10.891) -- (9.79832,10.891) -- (9.79832,10.8502) -- (9.93277,10.8502) -- (9.93277,10.435) --
 (10.0672,10.435) -- (10.0672,10.388) -- (10.2017,10.388) -- (10.2017,10.3689) -- (10.3361,10.3689) -- (10.3361,9.87683) -- (10.4706,9.87683) -- (10.4706,9.7782) -- (10.605,9.7782) -- (10.605,9.64756) -- (10.7395,9.64756) -- (10.7395,9.42036) --
 (10.874,9.42036) -- (10.874,9.17148) -- (11.0084,9.17148) -- (11.0084,9.03567) -- (11.1429,9.03567) -- (11.1429,8.64582) -- (11.2773,8.64582) -- (11.2773,8.48368) -- (11.4118,8.48368) -- (11.4118,8.18006) -- (11.5462,8.18006) -- (11.5462,8.00295) --
 (11.6807,8.00295) -- (11.6807,7.56456) -- (11.8151,7.56456) -- (11.8151,7.39261) -- (11.9496,7.39261) -- (11.9496,7.15198) -- (12.084,7.15198) -- (12.084,7.02238) -- (12.2185,7.02238) -- (12.2185,6.7864) -- (12.3529,6.7864) -- (12.3529,6.44095) --
 (12.4874,6.44095) -- (12.4874,6.19052) -- (12.6218,6.19052) -- (12.6218,6.0005) -- (12.7563,6.0005) -- (12.7563,5.71237) -- (12.8908,5.71237) -- (12.8908,5.543) -- (13.0252,5.543) -- (13.0252,5.21821) -- (13.1597,5.21821) -- (13.1597,5.05865) --
 (13.2941,5.05865) -- (13.2941,4.83042) -- (13.4286,4.83042) -- (13.4286,4.68997) -- (13.563,4.68997) -- (13.563,4.39823) -- (13.6975,4.39823) -- (13.6975,4.17206) -- (13.8319,4.17206) -- (13.8319,4.10132) -- (13.9664,4.10132) -- (13.9664,3.91336) --
 (14.1008,3.91336) -- (14.1008,3.69752) -- (14.2353,3.69752) -- (14.2353,3.5421) -- (14.3697,3.5421) -- (14.3697,3.39803) -- (14.5042,3.39803) -- (14.5042,3.22195) -- (14.6387,3.22195) -- (14.6387,3.09957) -- (14.7731,3.09957) -- (14.7731,2.98339) --
 (14.9076,2.98339) -- (14.9076,2.87341) -- (15.042,2.87341) -- (15.042,2.76807) -- (15.1765,2.76807) -- (15.1765,2.59251) -- (15.3109,2.59251) -- (15.3109,2.58063) -- (15.4454,2.58063) -- (15.4454,2.45774) -- (15.5798,2.45774) -- (15.5798,2.41385) --
 (15.7143,2.41385) -- (15.7143,2.3271) -- (15.8487,2.3271) -- (15.8487,2.21918) -- (15.9832,2.21918) -- (15.9832,2.09835) -- (16.1176,2.09835) -- (16.1176,2.08854) -- (16.2521,2.08854) -- (16.2521,2.01676) -- (16.3866,2.01676) -- (16.3866,2.0054) --
 (16.521,2.0054) -- (16.521,1.93931) -- (16.6555,1.93931) -- (16.6555,1.86857) -- (16.7899,1.86857) -- (16.7899,1.82003) -- (16.9244,1.82003) -- (16.9244,1.80815) -- (17.0588,1.80815) -- (17.0588,1.77614) -- (17.1933,1.77614) -- (17.1933,1.73225) --
 (17.3277,1.73225) -- (17.3277,1.72244) -- (17.4622,1.72244) -- (17.4622,1.72399) -- (17.5966,1.72399) -- (17.5966,1.68164) -- (17.7311,1.68164) -- (17.7311,1.67183) -- (17.8655,1.67183) -- (17.8655,1.64395) -- (18,1.64395);
\definecolor{c}{rgb}{1,1,1};
\draw [color=c, fill=c] (13.6235,8.05668) rectangle (18.6032,12.4291);
\definecolor{c}{rgb}{0,0,0};
\draw [c,line width=0.9] (13.6235,8.05668) -- (18.6032,8.05668);
\draw [c,line width=0.9] (18.6032,8.05668) -- (18.6032,12.4291);
\draw [c,line width=0.9] (18.6032,12.4291) -- (13.6235,12.4291);
\draw [c,line width=0.9] (13.6235,12.4291) -- (13.6235,8.05668);
\draw [anchor= west] (13.8725,11.8826) node[scale=0.80926, color=c, rotate=0]{Entries };
\draw [anchor= east] (18.3543,11.8826) node[scale=0.80926, color=c, rotate=0]{ 1822954};
\draw [anchor= west] (13.8725,10.7895) node[scale=0.80926, color=c, rotate=0]{Constant };
\draw [anchor= east] (18.3543,10.7895) node[scale=0.80926, color=c, rotate=0]{$ 1.95e+04 \pm 2.48e+01$};
\draw [anchor= west] (13.8725,9.69636) node[scale=0.80926, color=c, rotate=0]{Mean     };
\draw [anchor= east] (18.3543,9.69636) node[scale=0.80926, color=c, rotate=0]{$ 599.5 \pm 0.1$};
\draw [anchor= west] (13.8725,8.60324) node[scale=0.80926, color=c, rotate=0]{Sigma    };
\draw [anchor= east] (18.3543,8.60324) node[scale=0.80926, color=c, rotate=0]{$ 49.34 \pm 0.08$};
\definecolor{c}{rgb}{1,0,0};
\draw [c,line width=1.8] (4.87664,6.76735) -- (4.98286,6.96127) -- (5.08908,7.15611) -- (5.19529,7.35151) -- (5.30151,7.54709) -- (5.40773,7.74243) -- (5.51395,7.93713) -- (5.62017,8.13075) -- (5.72639,8.32287) -- (5.8326,8.51305) --
 (5.93882,8.70083) -- (6.04504,8.88576) -- (6.15126,9.06739) -- (6.25748,9.24526) -- (6.3637,9.41891) -- (6.46992,9.5879) -- (6.57613,9.75178) -- (6.68235,9.9101) -- (6.78857,10.0624) -- (6.89479,10.2084) -- (7.00101,10.3475) -- (7.10723,10.4794) --
 (7.21345,10.6037) -- (7.31966,10.7201) -- (7.42588,10.8281) -- (7.5321,10.9276) -- (7.63832,11.0182) -- (7.74454,11.0996) -- (7.85076,11.1715) -- (7.95697,11.2339) -- (8.06319,11.2864) -- (8.16941,11.3289) -- (8.27563,11.3614) -- (8.38185,11.3836)
 -- (8.48807,11.3955) -- (8.59429,11.3971) -- (8.7005,11.3884) -- (8.80672,11.3694) -- (8.91294,11.3402) -- (9.01916,11.3008) -- (9.12538,11.2514) -- (9.2316,11.1921) -- (9.33782,11.1231) -- (9.44403,11.0446) -- (9.55025,10.9569) -- (9.65647,10.8602)
 -- (9.76269,10.7548) -- (9.86891,10.641) -- (9.97513,10.5191) -- (10.0813,10.3895);
\draw [c,line width=1.8] (10.0813,10.3895) -- (10.1876,10.2526) -- (10.2938,10.1088) -- (10.4,9.95841) -- (10.5062,9.80192) -- (10.6124,9.63973) -- (10.7187,9.47229) -- (10.8249,9.30005) -- (10.9311,9.12344) -- (11.0373,8.94294) -- (11.1435,8.75899)
 -- (11.2497,8.57205) -- (11.356,8.38258) -- (11.4622,8.19102) -- (11.5684,7.99782) -- (11.6746,7.80341) -- (11.7808,7.60823) -- (11.8871,7.41268) -- (11.9933,7.21718) -- (12.0995,7.02212) -- (12.2057,6.82788) -- (12.3119,6.63482) --
 (12.4182,6.44328) -- (12.5244,6.25361) -- (12.6306,6.0661) -- (12.7368,5.88107) -- (12.843,5.69877) -- (12.9492,5.51947) -- (13.0555,5.34341) -- (13.1617,5.1708) -- (13.2679,5.00184) -- (13.3741,4.8367) -- (13.4803,4.67556) -- (13.5866,4.51854) --
 (13.6928,4.36577) -- (13.799,4.21735) -- (13.9052,4.07336) -- (14.0114,3.93388) -- (14.1176,3.79895) -- (14.2239,3.6686) -- (14.3301,3.54285) -- (14.4363,3.42172) -- (14.5425,3.30518) -- (14.6487,3.19321) -- (14.755,3.08577) -- (14.8612,2.98283) --
 (14.9674,2.88432) -- (15.0736,2.79018) -- (15.1798,2.70032) -- (15.2861,2.61466);
\draw [c,line width=1.8] (15.2861,2.61466) -- (15.3923,2.53311);
\definecolor{c}{rgb}{0,0,0};
\draw [c,line width=0.9] (2,1.32794) -- (18,1.32794);
\draw [anchor= east] (18,0.584292) node[scale=1.03405, color=c, rotate=0]{Canali};
\draw [c,line width=0.9] (3.21008,1.64664) -- (3.21008,1.32794);
\draw [c,line width=0.9] (3.54622,1.48729) -- (3.54622,1.32794);
\draw [c,line width=0.9] (3.88235,1.48729) -- (3.88235,1.32794);
\draw [c,line width=0.9] (4.21849,1.48729) -- (4.21849,1.32794);
\draw [c,line width=0.9] (4.55462,1.64664) -- (4.55462,1.32794);
\draw [c,line width=0.9] (4.89076,1.48729) -- (4.89076,1.32794);
\draw [c,line width=0.9] (5.22689,1.48729) -- (5.22689,1.32794);
\draw [c,line width=0.9] (5.56303,1.48729) -- (5.56303,1.32794);
\draw [c,line width=0.9] (5.89916,1.64664) -- (5.89916,1.32794);
\draw [c,line width=0.9] (6.23529,1.48729) -- (6.23529,1.32794);
\draw [c,line width=0.9] (6.57143,1.48729) -- (6.57143,1.32794);
\draw [c,line width=0.9] (6.90756,1.48729) -- (6.90756,1.32794);
\draw [c,line width=0.9] (7.2437,1.64664) -- (7.2437,1.32794);
\draw [c,line width=0.9] (7.57983,1.48729) -- (7.57983,1.32794);
\draw [c,line width=0.9] (7.91597,1.48729) -- (7.91597,1.32794);
\draw [c,line width=0.9] (8.2521,1.48729) -- (8.2521,1.32794);
\draw [c,line width=0.9] (8.58823,1.64664) -- (8.58823,1.32794);
\draw [c,line width=0.9] (8.92437,1.48729) -- (8.92437,1.32794);
\draw [c,line width=0.9] (9.2605,1.48729) -- (9.2605,1.32794);
\draw [c,line width=0.9] (9.59664,1.48729) -- (9.59664,1.32794);
\draw [c,line width=0.9] (9.93277,1.64664) -- (9.93277,1.32794);
\draw [c,line width=0.9] (10.2689,1.48729) -- (10.2689,1.32794);
\draw [c,line width=0.9] (10.605,1.48729) -- (10.605,1.32794);
\draw [c,line width=0.9] (10.9412,1.48729) -- (10.9412,1.32794);
\draw [c,line width=0.9] (11.2773,1.64664) -- (11.2773,1.32794);
\draw [c,line width=0.9] (11.6134,1.48729) -- (11.6134,1.32794);
\draw [c,line width=0.9] (11.9496,1.48729) -- (11.9496,1.32794);
\draw [c,line width=0.9] (12.2857,1.48729) -- (12.2857,1.32794);
\draw [c,line width=0.9] (12.6218,1.64664) -- (12.6218,1.32794);
\draw [c,line width=0.9] (12.958,1.48729) -- (12.958,1.32794);
\draw [c,line width=0.9] (13.2941,1.48729) -- (13.2941,1.32794);
\draw [c,line width=0.9] (13.6303,1.48729) -- (13.6303,1.32794);
\draw [c,line width=0.9] (13.9664,1.64664) -- (13.9664,1.32794);
\draw [c,line width=0.9] (14.3025,1.48729) -- (14.3025,1.32794);
\draw [c,line width=0.9] (14.6387,1.48729) -- (14.6387,1.32794);
\draw [c,line width=0.9] (14.9748,1.48729) -- (14.9748,1.32794);
\draw [c,line width=0.9] (15.3109,1.64664) -- (15.3109,1.32794);
\draw [c,line width=0.9] (15.6471,1.48729) -- (15.6471,1.32794);
\draw [c,line width=0.9] (15.9832,1.48729) -- (15.9832,1.32794);
\draw [c,line width=0.9] (16.3193,1.48729) -- (16.3193,1.32794);
\draw [c,line width=0.9] (16.6555,1.64664) -- (16.6555,1.32794);
\draw [c,line width=0.9] (16.9916,1.48729) -- (16.9916,1.32794);
\draw [c,line width=0.9] (17.3277,1.48729) -- (17.3277,1.32794);
\draw [c,line width=0.9] (17.6639,1.48729) -- (17.6639,1.32794);
\draw [c,line width=0.9] (18,1.64664) -- (18,1.32794);
\draw [c,line width=0.9] (3.21008,1.64664) -- (3.21008,1.32794);
\draw [c,line width=0.9] (2.87395,1.48729) -- (2.87395,1.32794);
\draw [c,line width=0.9] (2.53782,1.48729) -- (2.53782,1.32794);
\draw [c,line width=0.9] (2.20168,1.48729) -- (2.20168,1.32794);
\draw [anchor=base] (3.21008,0.942834) node[scale=0.899178, color=c, rotate=0]{520};
\draw [anchor=base] (4.55462,0.942834) node[scale=0.899178, color=c, rotate=0]{540};
\draw [anchor=base] (5.89916,0.942834) node[scale=0.899178, color=c, rotate=0]{560};
\draw [anchor=base] (7.2437,0.942834) node[scale=0.899178, color=c, rotate=0]{580};
\draw [anchor=base] (8.58823,0.942834) node[scale=0.899178, color=c, rotate=0]{600};
\draw [anchor=base] (9.93277,0.942834) node[scale=0.899178, color=c, rotate=0]{620};
\draw [anchor=base] (11.2773,0.942834) node[scale=0.899178, color=c, rotate=0]{640};
\draw [anchor=base] (12.6218,0.942834) node[scale=0.899178, color=c, rotate=0]{660};
\draw [anchor=base] (13.9664,0.942834) node[scale=0.899178, color=c, rotate=0]{680};
\draw [anchor=base] (15.3109,0.942834) node[scale=0.899178, color=c, rotate=0]{700};
\draw [anchor=base] (16.6555,0.942834) node[scale=0.899178, color=c, rotate=0]{720};
\draw [anchor=base] (18,0.942834) node[scale=0.899178, color=c, rotate=0]{740};
\draw [c,line width=0.9] (2,1.32794) -- (2,11.9514);
\draw [anchor= east] (0.4,11.9514) node[scale=1.48364, color=c, rotate=90]{Eventi};
\draw [c,line width=0.9] (2.48,1.32794) -- (2,1.32794);
\draw [c,line width=0.9] (2.24,1.58612) -- (2,1.58612);
\draw [c,line width=0.9] (2.24,1.8443) -- (2,1.8443);
\draw [c,line width=0.9] (2.24,2.10248) -- (2,2.10248);
\draw [c,line width=0.9] (2.48,2.36066) -- (2,2.36066);
\draw [c,line width=0.9] (2.24,2.61884) -- (2,2.61884);
\draw [c,line width=0.9] (2.24,2.87702) -- (2,2.87702);
\draw [c,line width=0.9] (2.24,3.1352) -- (2,3.1352);
\draw [c,line width=0.9] (2.48,3.39338) -- (2,3.39338);
\draw [c,line width=0.9] (2.24,3.65157) -- (2,3.65157);
\draw [c,line width=0.9] (2.24,3.90975) -- (2,3.90975);
\draw [c,line width=0.9] (2.24,4.16793) -- (2,4.16793);
\draw [c,line width=0.9] (2.48,4.42611) -- (2,4.42611);
\draw [c,line width=0.9] (2.24,4.68429) -- (2,4.68429);
\draw [c,line width=0.9] (2.24,4.94247) -- (2,4.94247);
\draw [c,line width=0.9] (2.24,5.20065) -- (2,5.20065);
\draw [c,line width=0.9] (2.48,5.45883) -- (2,5.45883);
\draw [c,line width=0.9] (2.24,5.71701) -- (2,5.71701);
\draw [c,line width=0.9] (2.24,5.9752) -- (2,5.9752);
\draw [c,line width=0.9] (2.24,6.23338) -- (2,6.23338);
\draw [c,line width=0.9] (2.48,6.49156) -- (2,6.49156);
\draw [c,line width=0.9] (2.24,6.74974) -- (2,6.74974);
\draw [c,line width=0.9] (2.24,7.00792) -- (2,7.00792);
\draw [c,line width=0.9] (2.24,7.2661) -- (2,7.2661);
\draw [c,line width=0.9] (2.48,7.52428) -- (2,7.52428);
\draw [c,line width=0.9] (2.24,7.78246) -- (2,7.78246);
\draw [c,line width=0.9] (2.24,8.04064) -- (2,8.04064);
\draw [c,line width=0.9] (2.24,8.29883) -- (2,8.29883);
\draw [c,line width=0.9] (2.48,8.55701) -- (2,8.55701);
\draw [c,line width=0.9] (2.24,8.81519) -- (2,8.81519);
\draw [c,line width=0.9] (2.24,9.07337) -- (2,9.07337);
\draw [c,line width=0.9] (2.24,9.33155) -- (2,9.33155);
\draw [c,line width=0.9] (2.48,9.58973) -- (2,9.58973);
\draw [c,line width=0.9] (2.24,9.84791) -- (2,9.84791);
\draw [c,line width=0.9] (2.24,10.1061) -- (2,10.1061);
\draw [c,line width=0.9] (2.24,10.3643) -- (2,10.3643);
\draw [c,line width=0.9] (2.48,10.6225) -- (2,10.6225);
\draw [c,line width=0.9] (2.24,10.8806) -- (2,10.8806);
\draw [c,line width=0.9] (2.24,11.1388) -- (2,11.1388);
\draw [c,line width=0.9] (2.24,11.397) -- (2,11.397);
\draw [c,line width=0.9] (2.48,11.6552) -- (2,11.6552);
\draw [c,line width=0.9] (2.48,11.6552) -- (2,11.6552);
\draw [c,line width=0.9] (2.24,11.9134) -- (2,11.9134);
\draw [anchor= east] (1.9,1.32794) node[scale=0.899178, color=c, rotate=0]{0};
\draw [anchor= east] (1.9,2.36066) node[scale=0.899178, color=c, rotate=0]{2000};
\draw [anchor= east] (1.9,3.39338) node[scale=0.899178, color=c, rotate=0]{4000};
\draw [anchor= east] (1.9,4.42611) node[scale=0.899178, color=c, rotate=0]{6000};
\draw [anchor= east] (1.9,5.45883) node[scale=0.899178, color=c, rotate=0]{8000};
\draw [anchor= east] (1.9,6.49156) node[scale=0.899178, color=c, rotate=0]{10000};
\draw [anchor= east] (1.9,7.52428) node[scale=0.899178, color=c, rotate=0]{12000};
\draw [anchor= east] (1.9,8.55701) node[scale=0.899178, color=c, rotate=0]{14000};
\draw [anchor= east] (1.9,9.58973) node[scale=0.899178, color=c, rotate=0]{16000};
\draw [anchor= east] (1.9,10.6225) node[scale=0.899178, color=c, rotate=0]{18000};
\draw [anchor= east] (1.9,11.6552) node[scale=0.899178, color=c, rotate=0]{20000};
\definecolor{c}{rgb}{1,1,1};
\draw [color=c, fill=c] (13.6235,8.05668) rectangle (18.6032,12.4291);
\definecolor{c}{rgb}{0,0,0};
\draw [c,line width=0.9] (13.6235,8.05668) -- (18.6032,8.05668);
\draw [c,line width=0.9] (18.6032,8.05668) -- (18.6032,12.4291);
\draw [c,line width=0.9] (18.6032,12.4291) -- (13.6235,12.4291);
\draw [c,line width=0.9] (13.6235,12.4291) -- (13.6235,8.05668);
\draw [anchor= west] (13.8725,11.8826) node[scale=0.80926, color=c, rotate=0]{Entries };
\draw [anchor= east] (18.3543,11.8826) node[scale=0.80926, color=c, rotate=0]{ 1822954};
\draw [anchor= west] (13.8725,10.7895) node[scale=0.80926, color=c, rotate=0]{Constant };
\draw [anchor= east] (18.3543,10.7895) node[scale=0.80926, color=c, rotate=0]{$ 1.95e+04 \pm 2.48e+01$};
\draw [anchor= west] (13.8725,9.69636) node[scale=0.80926, color=c, rotate=0]{Mean     };
\draw [anchor= east] (18.3543,9.69636) node[scale=0.80926, color=c, rotate=0]{$ 599.5 \pm 0.1$};
\draw [anchor= west] (13.8725,8.60324) node[scale=0.80926, color=c, rotate=0]{Sigma    };
\draw [anchor= east] (18.3543,8.60324) node[scale=0.80926, color=c, rotate=0]{$ 49.34 \pm 0.08$};
\draw (8.12753,12.6113) node[scale=1.39373, color=c, rotate=0]{Americio R1 Picco 59.5 keV};
\end{tikzpicture}

\begin{tikzpicture}
\pgfdeclareplotmark{cross} {
\pgfpathmoveto{\pgfpoint{-0.3\pgfplotmarksize}{\pgfplotmarksize}}
\pgfpathlineto{\pgfpoint{+0.3\pgfplotmarksize}{\pgfplotmarksize}}
\pgfpathlineto{\pgfpoint{+0.3\pgfplotmarksize}{0.3\pgfplotmarksize}}
\pgfpathlineto{\pgfpoint{+1\pgfplotmarksize}{0.3\pgfplotmarksize}}
\pgfpathlineto{\pgfpoint{+1\pgfplotmarksize}{-0.3\pgfplotmarksize}}
\pgfpathlineto{\pgfpoint{+0.3\pgfplotmarksize}{-0.3\pgfplotmarksize}}
\pgfpathlineto{\pgfpoint{+0.3\pgfplotmarksize}{-1.\pgfplotmarksize}}
\pgfpathlineto{\pgfpoint{-0.3\pgfplotmarksize}{-1.\pgfplotmarksize}}
\pgfpathlineto{\pgfpoint{-0.3\pgfplotmarksize}{-0.3\pgfplotmarksize}}
\pgfpathlineto{\pgfpoint{-1.\pgfplotmarksize}{-0.3\pgfplotmarksize}}
\pgfpathlineto{\pgfpoint{-1.\pgfplotmarksize}{0.3\pgfplotmarksize}}
\pgfpathlineto{\pgfpoint{-0.3\pgfplotmarksize}{0.3\pgfplotmarksize}}
\pgfpathclose
\pgfusepathqstroke
}
\pgfdeclareplotmark{cross*} {
\pgfpathmoveto{\pgfpoint{-0.3\pgfplotmarksize}{\pgfplotmarksize}}
\pgfpathlineto{\pgfpoint{+0.3\pgfplotmarksize}{\pgfplotmarksize}}
\pgfpathlineto{\pgfpoint{+0.3\pgfplotmarksize}{0.3\pgfplotmarksize}}
\pgfpathlineto{\pgfpoint{+1\pgfplotmarksize}{0.3\pgfplotmarksize}}
\pgfpathlineto{\pgfpoint{+1\pgfplotmarksize}{-0.3\pgfplotmarksize}}
\pgfpathlineto{\pgfpoint{+0.3\pgfplotmarksize}{-0.3\pgfplotmarksize}}
\pgfpathlineto{\pgfpoint{+0.3\pgfplotmarksize}{-1.\pgfplotmarksize}}
\pgfpathlineto{\pgfpoint{-0.3\pgfplotmarksize}{-1.\pgfplotmarksize}}
\pgfpathlineto{\pgfpoint{-0.3\pgfplotmarksize}{-0.3\pgfplotmarksize}}
\pgfpathlineto{\pgfpoint{-1.\pgfplotmarksize}{-0.3\pgfplotmarksize}}
\pgfpathlineto{\pgfpoint{-1.\pgfplotmarksize}{0.3\pgfplotmarksize}}
\pgfpathlineto{\pgfpoint{-0.3\pgfplotmarksize}{0.3\pgfplotmarksize}}
\pgfpathclose
\pgfusepathqfillstroke
}
\pgfdeclareplotmark{newstar} {
\pgfpathmoveto{\pgfqpoint{0pt}{\pgfplotmarksize}}
\pgfpathlineto{\pgfqpointpolar{44}{0.5\pgfplotmarksize}}
\pgfpathlineto{\pgfqpointpolar{18}{\pgfplotmarksize}}
\pgfpathlineto{\pgfqpointpolar{-20}{0.5\pgfplotmarksize}}
\pgfpathlineto{\pgfqpointpolar{-54}{\pgfplotmarksize}}
\pgfpathlineto{\pgfqpointpolar{-90}{0.5\pgfplotmarksize}}
\pgfpathlineto{\pgfqpointpolar{234}{\pgfplotmarksize}}
\pgfpathlineto{\pgfqpointpolar{198}{0.5\pgfplotmarksize}}
\pgfpathlineto{\pgfqpointpolar{162}{\pgfplotmarksize}}
\pgfpathlineto{\pgfqpointpolar{134}{0.5\pgfplotmarksize}}
\pgfpathclose
\pgfusepathqstroke
}
\pgfdeclareplotmark{newstar*} {
\pgfpathmoveto{\pgfqpoint{0pt}{\pgfplotmarksize}}
\pgfpathlineto{\pgfqpointpolar{44}{0.5\pgfplotmarksize}}
\pgfpathlineto{\pgfqpointpolar{18}{\pgfplotmarksize}}
\pgfpathlineto{\pgfqpointpolar{-20}{0.5\pgfplotmarksize}}
\pgfpathlineto{\pgfqpointpolar{-54}{\pgfplotmarksize}}
\pgfpathlineto{\pgfqpointpolar{-90}{0.5\pgfplotmarksize}}
\pgfpathlineto{\pgfqpointpolar{234}{\pgfplotmarksize}}
\pgfpathlineto{\pgfqpointpolar{198}{0.5\pgfplotmarksize}}
\pgfpathlineto{\pgfqpointpolar{162}{\pgfplotmarksize}}
\pgfpathlineto{\pgfqpointpolar{134}{0.5\pgfplotmarksize}}
\pgfpathclose
\pgfusepathqfillstroke
}
\definecolor{c}{rgb}{1,1,1};
\draw [color=c, fill=c] (0,0) rectangle (20,14.9811);
\draw [color=c, fill=c] (2.19697,1.53409) rectangle (17.9735,13.4848);
\definecolor{c}{rgb}{0,0,0};
\draw [c,line width=0.9] (2.19697,1.53409) -- (2.19697,13.4848) -- (17.9735,13.4848) -- (17.9735,1.53409) -- (2.19697,1.53409);
\definecolor{c}{rgb}{1,1,1};
\draw [color=c, fill=c] (2.19697,1.53409) rectangle (17.9735,13.4848);
\definecolor{c}{rgb}{0,0,0};
\draw [c,line width=0.9] (2.19697,1.53409) -- (2.19697,13.4848) -- (17.9735,13.4848) -- (17.9735,1.53409) -- (2.19697,1.53409);
\definecolor{c}{rgb}{0,0,0.6};
\draw [c,line width=0.9] (2.19697,3.01298) -- (2.30076,3.01298) -- (2.30076,3.00235) -- (2.40456,3.00235) -- (2.40456,2.98979) -- (2.50835,2.98979) -- (2.50835,3.06808) -- (2.61214,3.06808) -- (2.61214,3.15991) -- (2.71593,3.15991) --
 (2.71593,3.12994) -- (2.81973,3.12994) -- (2.81973,3.1947) -- (2.92352,3.1947) -- (2.92352,3.21307) -- (3.02731,3.21307) -- (3.02731,3.2817) -- (3.13111,3.2817) -- (3.13111,3.34259) -- (3.2349,3.34259) -- (3.2349,3.41122) -- (3.33869,3.41122) --
 (3.33869,3.57168) -- (3.44248,3.57168) -- (3.44248,3.62871) -- (3.54628,3.62871) -- (3.54628,3.63934) -- (3.65007,3.63934) -- (3.65007,3.74566) -- (3.75386,3.74566) -- (3.75386,3.88872) -- (3.85766,3.88872) -- (3.85766,3.94768) -- (3.96145,3.94768)
 -- (3.96145,4.09944) -- (4.06524,4.09944) -- (4.06524,4.28019) -- (4.16903,4.28019) -- (4.16903,4.50058) -- (4.27283,4.50058) -- (4.27283,4.52087) -- (4.37662,4.52087) -- (4.37662,4.6823) -- (4.48041,4.6823) -- (4.48041,4.89011) -- (4.58421,4.89011)
 -- (4.58421,4.91911) -- (4.688,4.91911) -- (4.688,5.25452) -- (4.79179,5.25452) -- (4.79179,5.27385) -- (4.89558,5.27385) -- (4.89558,5.58896) -- (4.99938,5.58896) -- (4.99938,5.83641) -- (5.10317,5.83641) -- (5.10317,5.84318) -- (5.20696,5.84318)
 -- (5.20696,6.16022) -- (5.31076,6.16022) -- (5.31076,6.41444) -- (5.41455,6.41444) -- (5.41455,6.54879) -- (5.51834,6.54879) -- (5.51834,6.85424) -- (5.62213,6.85424) -- (5.62213,7.10459) -- (5.72593,7.10459) -- (5.72593,7.27471) --
 (5.82972,7.27471) -- (5.82972,7.61398) -- (5.93351,7.61398) -- (5.93351,7.69228) -- (6.03731,7.69228) -- (6.03731,7.88656) -- (6.1411,7.88656) -- (6.1411,8.32926) -- (6.24489,8.32926) -- (6.24489,8.52162) -- (6.34868,8.52162) -- (6.34868,8.84156) --
 (6.45248,8.84156) -- (6.45248,9.04358) -- (6.55627,9.04358) -- (6.55627,9.33839) -- (6.66006,9.33839) -- (6.66006,9.56168) -- (6.76386,9.56168) -- (6.76386,9.80526) -- (6.86765,9.80526) -- (6.86765,9.93188) -- (6.97144,9.93188) -- (6.97144,10.1803)
 -- (7.07523,10.1803) -- (7.07523,10.4616) -- (7.17903,10.4616) -- (7.17903,10.5563) -- (7.28282,10.5563) -- (7.28282,10.9217) -- (7.38661,10.9217) -- (7.38661,11.1633) -- (7.49041,11.1633) -- (7.49041,11.2378) -- (7.5942,11.2378) -- (7.5942,11.6167)
 -- (7.69799,11.6167) -- (7.69799,11.6196) -- (7.80178,11.6196) -- (7.80178,11.9076) -- (7.90558,11.9076) -- (7.90558,12.013) -- (8.00937,12.013) -- (8.00937,12.2372) -- (8.11316,12.2372) -- (8.11316,12.3058) -- (8.21696,12.3058) -- (8.21696,12.4209)
 -- (8.32075,12.4209) -- (8.32075,12.532) -- (8.42454,12.532) -- (8.42454,12.6384) -- (8.52833,12.6384) -- (8.52833,12.6451) -- (8.63213,12.6451) -- (8.63213,12.6239) -- (8.73592,12.6239) -- (8.73592,12.705) -- (8.83971,12.705) -- (8.83971,12.7321)
 -- (8.94351,12.7321) -- (8.94351,12.8877) -- (9.0473,12.8877) -- (9.0473,12.9158) -- (9.15109,12.9158) -- (9.15109,12.7737) -- (9.25488,12.7737) -- (9.25488,12.7737) -- (9.35868,12.7737) -- (9.35868,12.5833) -- (9.46247,12.5833) -- (9.46247,12.5195)
 -- (9.56626,12.5195) -- (9.56626,12.5001) -- (9.67006,12.5001) -- (9.67006,12.1454) -- (9.77385,12.1454) -- (9.77385,12.2894) -- (9.87764,12.2894) -- (9.87764,12.0149) -- (9.98143,12.0149) -- (9.98143,11.9124) -- (10.0852,11.9124) --
 (10.0852,11.6573) -- (10.189,11.6573) -- (10.189,11.4862) -- (10.2928,11.4862) -- (10.2928,11.3257) -- (10.3966,11.3257) -- (10.3966,11.2291) -- (10.5004,11.2291) -- (10.5004,10.9333) -- (10.6042,10.9333) -- (10.6042,10.8086) -- (10.708,10.8086) --
 (10.708,10.7361) -- (10.8118,10.7361) -- (10.8118,10.1842) -- (10.9156,10.1842) -- (10.9156,10.1523) -- (11.0194,10.1523) -- (11.0194,9.89902) -- (11.1232,9.89902) -- (11.1232,9.55394) -- (11.2269,9.55394) -- (11.2269,9.22337) -- (11.3307,9.22337)
 -- (11.3307,9.1702) -- (11.4345,9.1702) -- (11.4345,8.94112) -- (11.5383,8.94112) -- (11.5383,8.56608) -- (11.6421,8.56608) -- (11.6421,8.54965) -- (11.7459,8.54965) -- (11.7459,8.19007) -- (11.8497,8.19007) -- (11.8497,7.7725) -- (11.9535,7.7725)
 -- (11.9535,7.59658) -- (12.0573,7.59658) -- (12.0573,7.30564) -- (12.1611,7.30564) -- (12.1611,7.05916) -- (12.2649,7.05916) -- (12.2649,6.70828) -- (12.3687,6.70828) -- (12.3687,6.69958) -- (12.4725,6.69958) -- (12.4725,6.35741) --
 (12.5763,6.35741) -- (12.5763,5.97174) -- (12.68,5.97174) -- (12.68,5.93307) -- (12.7838,5.93307) -- (12.7838,5.68272) -- (12.8876,5.68272) -- (12.8876,5.32122) -- (12.9914,5.32122) -- (12.9914,5.16559) -- (13.0952,5.16559) -- (13.0952,5.0322) --
 (13.199,5.0322) -- (13.199,4.85725) -- (13.3028,4.85725) -- (13.3028,4.59144) -- (13.4066,4.59144) -- (13.4066,4.53247) -- (13.5104,4.53247) -- (13.5104,4.29662) -- (13.6142,4.29662) -- (13.6142,4.18063) -- (13.718,4.18063) -- (13.718,3.96411) --
 (13.8218,3.96411) -- (13.8218,3.82203) -- (13.9256,3.82203) -- (13.9256,3.66544) -- (14.0294,3.66544) -- (14.0294,3.48178) -- (14.1331,3.48178) -- (14.1331,3.43925) -- (14.2369,3.43925) -- (14.2369,3.30006) -- (14.3407,3.30006) -- (14.3407,3.2092)
 -- (14.4445,3.2092) -- (14.4445,3.09998) -- (14.5483,3.09998) -- (14.5483,2.99462) -- (14.6521,2.99462) -- (14.6521,2.87476) -- (14.7559,2.87476) -- (14.7559,2.83126) -- (14.8597,2.83126) -- (14.8597,2.68144) -- (14.9635,2.68144) --
 (14.9635,2.62055) -- (15.0673,2.62055) -- (15.0673,2.57222) -- (15.1711,2.57222) -- (15.1711,2.50745) -- (15.2749,2.50745) -- (15.2749,2.47556) -- (15.3787,2.47556) -- (15.3787,2.42046) -- (15.4825,2.42046) -- (15.4825,2.31703) -- (15.5862,2.31703)
 -- (15.5862,2.30833) -- (15.69,2.30833) -- (15.69,2.26) -- (15.7938,2.26) -- (15.7938,2.18848) -- (15.8976,2.18848) -- (15.8976,2.19621) -- (16.0014,2.19621) -- (16.0014,2.16528) -- (16.1052,2.16528) -- (16.1052,2.10535) -- (16.209,2.10535) --
 (16.209,2.10728) -- (16.3128,2.10728) -- (16.3128,2.12082) -- (16.4166,2.12082) -- (16.4166,2.08118) -- (16.5204,2.08118) -- (16.5204,2.07442) -- (16.6242,2.07442) -- (16.6242,2.00482) -- (16.728,2.00482) -- (16.728,2.01062) -- (16.8318,2.01062) --
 (16.8318,2.04349) -- (16.9356,2.04349) -- (16.9356,2.05025) -- (17.0393,2.05025) -- (17.0393,2.06862) -- (17.1431,2.06862) -- (17.1431,2.06475) -- (17.2469,2.06475) -- (17.2469,2.07539) -- (17.3507,2.07539) -- (17.3507,2.08022) -- (17.4545,2.08022)
 -- (17.4545,2.07055) -- (17.5583,2.07055) -- (17.5583,2.10245) -- (17.6621,2.10245) -- (17.6621,2.17784) -- (17.7659,2.17784) -- (17.7659,2.16914) -- (17.8697,2.16914) -- (17.8697,2.19621) -- (17.9735,2.19621);
\definecolor{c}{rgb}{1,1,1};
\draw [color=c, fill=c] (13.2197,9.45076) rectangle (18.5985,13.6742);
\definecolor{c}{rgb}{0,0,0};
\draw [c,line width=0.9] (13.2197,9.45076) -- (18.5985,9.45076);
\draw [c,line width=0.9] (18.5985,9.45076) -- (18.5985,13.6742);
\draw [c,line width=0.9] (18.5985,13.6742) -- (13.2197,13.6742);
\draw [c,line width=0.9] (13.2197,13.6742) -- (13.2197,9.45076);
\draw [anchor= west] (13.4886,13.1463) node[scale=0.799213, color=c, rotate=0]{Entries };
\draw [anchor= east] (18.3295,13.1463) node[scale=0.799213, color=c, rotate=0]{ 5011262};
\draw [anchor= west] (13.4886,12.0904) node[scale=0.799213, color=c, rotate=0]{Constant };
\draw [anchor= east] (18.3295,12.0904) node[scale=0.799213, color=c, rotate=0]{$ 1.157e+04 \pm 1.863e+01$};
\draw [anchor= west] (13.4886,11.0346) node[scale=0.799213, color=c, rotate=0]{Mean     };
\draw [anchor= east] (18.3295,11.0346) node[scale=0.799213, color=c, rotate=0]{$ 1.126e+04 \pm 0.4$};
\draw [anchor= west] (13.4886,9.97869) node[scale=0.799213, color=c, rotate=0]{Sigma    };
\draw [anchor= east] (18.3295,9.97869) node[scale=0.799213, color=c, rotate=0]{$ 266.4 \pm 0.4$};
\definecolor{c}{rgb}{1,0,0};
\draw [c,line width=1.8] (4.52764,4.66018) -- (4.62209,4.83349) -- (4.71654,5.01235) -- (4.81099,5.19663) -- (4.90544,5.38617) -- (4.9999,5.5808) -- (5.09435,5.7803) -- (5.1888,5.98444) -- (5.28325,6.19295) -- (5.3777,6.40554) -- (5.47215,6.62189) --
 (5.56661,6.84166) -- (5.66106,7.06445) -- (5.75551,7.28988) -- (5.84996,7.51751) -- (5.94441,7.74688) -- (6.03886,7.97753) -- (6.13331,8.20894) -- (6.22777,8.44059) -- (6.32222,8.67196) -- (6.41667,8.90246) -- (6.51112,9.13154) -- (6.60557,9.3586)
 -- (6.70002,9.58305) -- (6.79447,9.80428) -- (6.88893,10.0217) -- (6.98338,10.2346) -- (7.07783,10.4425) -- (7.17228,10.6447) -- (7.26673,10.8407) -- (7.36118,11.0298) -- (7.45564,11.2114) -- (7.55009,11.3849) -- (7.64454,11.5499) --
 (7.73899,11.7058) -- (7.83344,11.8521) -- (7.92789,11.9882) -- (8.02234,12.1138) -- (8.1168,12.2284) -- (8.21125,12.3317) -- (8.3057,12.4232) -- (8.40015,12.5027) -- (8.4946,12.5699) -- (8.58905,12.6245) -- (8.68351,12.6664) -- (8.77796,12.6954) --
 (8.87241,12.7115) -- (8.96686,12.7145) -- (9.06131,12.7045) -- (9.15576,12.6815);
\draw [c,line width=1.8] (9.15576,12.6815) -- (9.25021,12.6455) -- (9.34467,12.5968) -- (9.43912,12.5354) -- (9.53357,12.4616) -- (9.62802,12.3756) -- (9.72247,12.2778) -- (9.81692,12.1684) -- (9.91137,12.0478) -- (10.0058,11.9165) --
 (10.1003,11.7749) -- (10.1947,11.6234) -- (10.2892,11.4626) -- (10.3836,11.2929) -- (10.4781,11.115) -- (10.5725,10.9293) -- (10.667,10.7365) -- (10.7614,10.5371) -- (10.8559,10.3317) -- (10.9503,10.1211) -- (11.0448,9.90565) -- (11.1392,9.68613) --
 (11.2337,9.4631) -- (11.3281,9.23717) -- (11.4226,9.00895) -- (11.517,8.77904) -- (11.6115,8.548) -- (11.706,8.31641) -- (11.8004,8.08482) -- (11.8949,7.85376) -- (11.9893,7.62373) -- (12.0838,7.39524) -- (12.1782,7.16874) -- (12.2727,6.94467) --
 (12.3671,6.72345) -- (12.4616,6.50547) -- (12.556,6.29109) -- (12.6505,6.08065) -- (12.7449,5.87445) -- (12.8394,5.67276) -- (12.9338,5.47585) -- (13.0283,5.28392) -- (13.1227,5.09718) -- (13.2172,4.91579) -- (13.3116,4.73989) -- (13.4061,4.5696) --
 (13.5005,4.40499) -- (13.595,4.24615) -- (13.6894,4.09311) -- (13.7839,3.94588);
\draw [c,line width=1.8] (13.7839,3.94588) -- (13.8783,3.80448);
\definecolor{c}{rgb}{0,0,0};
\draw [c,line width=0.9] (2.19697,1.53409) -- (17.9735,1.53409);
\draw [anchor= east] (17.9735,0.695152) node[scale=1.17779, color=c, rotate=0]{Canali};
\draw [c,line width=0.9] (4.16903,1.88861) -- (4.16903,1.53409);
\draw [c,line width=0.9] (4.688,1.71135) -- (4.688,1.53409);
\draw [c,line width=0.9] (5.20696,1.71135) -- (5.20696,1.53409);
\draw [c,line width=0.9] (5.72593,1.71135) -- (5.72593,1.53409);
\draw [c,line width=0.9] (6.24489,1.88861) -- (6.24489,1.53409);
\draw [c,line width=0.9] (6.76386,1.71135) -- (6.76386,1.53409);
\draw [c,line width=0.9] (7.28282,1.71135) -- (7.28282,1.53409);
\draw [c,line width=0.9] (7.80178,1.71135) -- (7.80178,1.53409);
\draw [c,line width=0.9] (8.32075,1.88861) -- (8.32075,1.53409);
\draw [c,line width=0.9] (8.83971,1.71135) -- (8.83971,1.53409);
\draw [c,line width=0.9] (9.35868,1.71135) -- (9.35868,1.53409);
\draw [c,line width=0.9] (9.87764,1.71135) -- (9.87764,1.53409);
\draw [c,line width=0.9] (10.3966,1.88861) -- (10.3966,1.53409);
\draw [c,line width=0.9] (10.9156,1.71135) -- (10.9156,1.53409);
\draw [c,line width=0.9] (11.4345,1.71135) -- (11.4345,1.53409);
\draw [c,line width=0.9] (11.9535,1.71135) -- (11.9535,1.53409);
\draw [c,line width=0.9] (12.4725,1.88861) -- (12.4725,1.53409);
\draw [c,line width=0.9] (12.9914,1.71135) -- (12.9914,1.53409);
\draw [c,line width=0.9] (13.5104,1.71135) -- (13.5104,1.53409);
\draw [c,line width=0.9] (14.0294,1.71135) -- (14.0294,1.53409);
\draw [c,line width=0.9] (14.5483,1.88861) -- (14.5483,1.53409);
\draw [c,line width=0.9] (15.0673,1.71135) -- (15.0673,1.53409);
\draw [c,line width=0.9] (15.5862,1.71135) -- (15.5862,1.53409);
\draw [c,line width=0.9] (16.1052,1.71135) -- (16.1052,1.53409);
\draw [c,line width=0.9] (16.6242,1.88861) -- (16.6242,1.53409);
\draw [c,line width=0.9] (4.16903,1.88861) -- (4.16903,1.53409);
\draw [c,line width=0.9] (3.65007,1.71135) -- (3.65007,1.53409);
\draw [c,line width=0.9] (3.13111,1.71135) -- (3.13111,1.53409);
\draw [c,line width=0.9] (2.61214,1.71135) -- (2.61214,1.53409);
\draw [c,line width=0.9] (16.6242,1.88861) -- (16.6242,1.53409);
\draw [c,line width=0.9] (17.1431,1.71135) -- (17.1431,1.53409);
\draw [c,line width=0.9] (17.6621,1.71135) -- (17.6621,1.53409);
\draw [anchor=base] (4.16903,1.09964) node[scale=1.00953, color=c, rotate=0]{10800};
\draw [anchor=base] (6.24489,1.09964) node[scale=1.00953, color=c, rotate=0]{11000};
\draw [anchor=base] (8.32075,1.09964) node[scale=1.00953, color=c, rotate=0]{11200};
\draw [anchor=base] (10.3966,1.09964) node[scale=1.00953, color=c, rotate=0]{11400};
\draw [anchor=base] (12.4725,1.09964) node[scale=1.00953, color=c, rotate=0]{11600};
\draw [anchor=base] (14.5483,1.09964) node[scale=1.00953, color=c, rotate=0]{11800};
\draw [anchor=base] (16.6242,1.09964) node[scale=1.00953, color=c, rotate=0]{12000};
\draw [c,line width=0.9] (2.19697,1.53409) -- (2.19697,13.4848);
\draw [anchor= east] (0.59697,13.4848) node[scale=1.68255, color=c, rotate=90]{Eventi};
\draw [c,line width=0.9] (2.6756,1.53409) -- (2.19697,1.53409);
\draw [c,line width=0.9] (2.43629,2.01739) -- (2.19697,2.01739);
\draw [c,line width=0.9] (2.43629,2.50069) -- (2.19697,2.50069);
\draw [c,line width=0.9] (2.43629,2.98399) -- (2.19697,2.98399);
\draw [c,line width=0.9] (2.6756,3.46728) -- (2.19697,3.46728);
\draw [c,line width=0.9] (2.43629,3.95058) -- (2.19697,3.95058);
\draw [c,line width=0.9] (2.43629,4.43388) -- (2.19697,4.43388);
\draw [c,line width=0.9] (2.43629,4.91718) -- (2.19697,4.91718);
\draw [c,line width=0.9] (2.6756,5.40048) -- (2.19697,5.40048);
\draw [c,line width=0.9] (2.43629,5.88378) -- (2.19697,5.88378);
\draw [c,line width=0.9] (2.43629,6.36707) -- (2.19697,6.36707);
\draw [c,line width=0.9] (2.43629,6.85037) -- (2.19697,6.85037);
\draw [c,line width=0.9] (2.6756,7.33367) -- (2.19697,7.33367);
\draw [c,line width=0.9] (2.43629,7.81697) -- (2.19697,7.81697);
\draw [c,line width=0.9] (2.43629,8.30027) -- (2.19697,8.30027);
\draw [c,line width=0.9] (2.43629,8.78356) -- (2.19697,8.78356);
\draw [c,line width=0.9] (2.6756,9.26686) -- (2.19697,9.26686);
\draw [c,line width=0.9] (2.43629,9.75016) -- (2.19697,9.75016);
\draw [c,line width=0.9] (2.43629,10.2335) -- (2.19697,10.2335);
\draw [c,line width=0.9] (2.43629,10.7168) -- (2.19697,10.7168);
\draw [c,line width=0.9] (2.6756,11.2001) -- (2.19697,11.2001);
\draw [c,line width=0.9] (2.43629,11.6834) -- (2.19697,11.6834);
\draw [c,line width=0.9] (2.43629,12.1667) -- (2.19697,12.1667);
\draw [c,line width=0.9] (2.43629,12.65) -- (2.19697,12.65);
\draw [c,line width=0.9] (2.6756,13.1332) -- (2.19697,13.1332);
\draw [c,line width=0.9] (2.6756,13.1332) -- (2.19697,13.1332);
\draw [anchor= east] (2.09697,1.53409) node[scale=1.00953, color=c, rotate=0]{0};
\draw [anchor= east] (2.09697,3.46728) node[scale=1.00953, color=c, rotate=0]{2000};
\draw [anchor= east] (2.09697,5.40048) node[scale=1.00953, color=c, rotate=0]{4000};
\draw [anchor= east] (2.09697,7.33367) node[scale=1.00953, color=c, rotate=0]{6000};
\draw [anchor= east] (2.09697,9.26686) node[scale=1.00953, color=c, rotate=0]{8000};
\draw [anchor= east] (2.09697,11.2001) node[scale=1.00953, color=c, rotate=0]{10000};
\draw [anchor= east] (2.09697,13.1332) node[scale=1.00953, color=c, rotate=0]{12000};
\definecolor{c}{rgb}{1,1,1};
\draw [color=c, fill=c] (13.2197,9.45076) rectangle (18.5985,13.6742);
\definecolor{c}{rgb}{0,0,0};
\draw [c,line width=0.9] (13.2197,9.45076) -- (18.5985,9.45076);
\draw [c,line width=0.9] (18.5985,9.45076) -- (18.5985,13.6742);
\draw [c,line width=0.9] (18.5985,13.6742) -- (13.2197,13.6742);
\draw [c,line width=0.9] (13.2197,13.6742) -- (13.2197,9.45076);
\draw [anchor= west] (13.4886,13.1463) node[scale=0.799213, color=c, rotate=0]{Entries };
\draw [anchor= east] (18.3295,13.1463) node[scale=0.799213, color=c, rotate=0]{ 5011262};
\draw [anchor= west] (13.4886,12.0904) node[scale=0.799213, color=c, rotate=0]{Constant };
\draw [anchor= east] (18.3295,12.0904) node[scale=0.799213, color=c, rotate=0]{$ 1.157e+04 \pm 1.863e+01$};
\draw [anchor= west] (13.4886,11.0346) node[scale=0.799213, color=c, rotate=0]{Mean     };
\draw [anchor= east] (18.3295,11.0346) node[scale=0.799213, color=c, rotate=0]{$ 1.126e+04 \pm 0.4$};
\draw [anchor= west] (13.4886,9.97869) node[scale=0.799213, color=c, rotate=0]{Sigma    };
\draw [anchor= east] (18.3295,9.97869) node[scale=0.799213, color=c, rotate=0]{$ 266.4 \pm 0.4$};
\draw (7.07386,14.2992) node[scale=1.55636, color=c, rotate=0]{Cobalto R1 Picco 1173 keV};
\end{tikzpicture}

\begin{tikzpicture}
\pgfdeclareplotmark{cross} {
\pgfpathmoveto{\pgfpoint{-0.3\pgfplotmarksize}{\pgfplotmarksize}}
\pgfpathlineto{\pgfpoint{+0.3\pgfplotmarksize}{\pgfplotmarksize}}
\pgfpathlineto{\pgfpoint{+0.3\pgfplotmarksize}{0.3\pgfplotmarksize}}
\pgfpathlineto{\pgfpoint{+1\pgfplotmarksize}{0.3\pgfplotmarksize}}
\pgfpathlineto{\pgfpoint{+1\pgfplotmarksize}{-0.3\pgfplotmarksize}}
\pgfpathlineto{\pgfpoint{+0.3\pgfplotmarksize}{-0.3\pgfplotmarksize}}
\pgfpathlineto{\pgfpoint{+0.3\pgfplotmarksize}{-1.\pgfplotmarksize}}
\pgfpathlineto{\pgfpoint{-0.3\pgfplotmarksize}{-1.\pgfplotmarksize}}
\pgfpathlineto{\pgfpoint{-0.3\pgfplotmarksize}{-0.3\pgfplotmarksize}}
\pgfpathlineto{\pgfpoint{-1.\pgfplotmarksize}{-0.3\pgfplotmarksize}}
\pgfpathlineto{\pgfpoint{-1.\pgfplotmarksize}{0.3\pgfplotmarksize}}
\pgfpathlineto{\pgfpoint{-0.3\pgfplotmarksize}{0.3\pgfplotmarksize}}
\pgfpathclose
\pgfusepathqstroke
}
\pgfdeclareplotmark{cross*} {
\pgfpathmoveto{\pgfpoint{-0.3\pgfplotmarksize}{\pgfplotmarksize}}
\pgfpathlineto{\pgfpoint{+0.3\pgfplotmarksize}{\pgfplotmarksize}}
\pgfpathlineto{\pgfpoint{+0.3\pgfplotmarksize}{0.3\pgfplotmarksize}}
\pgfpathlineto{\pgfpoint{+1\pgfplotmarksize}{0.3\pgfplotmarksize}}
\pgfpathlineto{\pgfpoint{+1\pgfplotmarksize}{-0.3\pgfplotmarksize}}
\pgfpathlineto{\pgfpoint{+0.3\pgfplotmarksize}{-0.3\pgfplotmarksize}}
\pgfpathlineto{\pgfpoint{+0.3\pgfplotmarksize}{-1.\pgfplotmarksize}}
\pgfpathlineto{\pgfpoint{-0.3\pgfplotmarksize}{-1.\pgfplotmarksize}}
\pgfpathlineto{\pgfpoint{-0.3\pgfplotmarksize}{-0.3\pgfplotmarksize}}
\pgfpathlineto{\pgfpoint{-1.\pgfplotmarksize}{-0.3\pgfplotmarksize}}
\pgfpathlineto{\pgfpoint{-1.\pgfplotmarksize}{0.3\pgfplotmarksize}}
\pgfpathlineto{\pgfpoint{-0.3\pgfplotmarksize}{0.3\pgfplotmarksize}}
\pgfpathclose
\pgfusepathqfillstroke
}
\pgfdeclareplotmark{newstar} {
\pgfpathmoveto{\pgfqpoint{0pt}{\pgfplotmarksize}}
\pgfpathlineto{\pgfqpointpolar{44}{0.5\pgfplotmarksize}}
\pgfpathlineto{\pgfqpointpolar{18}{\pgfplotmarksize}}
\pgfpathlineto{\pgfqpointpolar{-20}{0.5\pgfplotmarksize}}
\pgfpathlineto{\pgfqpointpolar{-54}{\pgfplotmarksize}}
\pgfpathlineto{\pgfqpointpolar{-90}{0.5\pgfplotmarksize}}
\pgfpathlineto{\pgfqpointpolar{234}{\pgfplotmarksize}}
\pgfpathlineto{\pgfqpointpolar{198}{0.5\pgfplotmarksize}}
\pgfpathlineto{\pgfqpointpolar{162}{\pgfplotmarksize}}
\pgfpathlineto{\pgfqpointpolar{134}{0.5\pgfplotmarksize}}
\pgfpathclose
\pgfusepathqstroke
}
\pgfdeclareplotmark{newstar*} {
\pgfpathmoveto{\pgfqpoint{0pt}{\pgfplotmarksize}}
\pgfpathlineto{\pgfqpointpolar{44}{0.5\pgfplotmarksize}}
\pgfpathlineto{\pgfqpointpolar{18}{\pgfplotmarksize}}
\pgfpathlineto{\pgfqpointpolar{-20}{0.5\pgfplotmarksize}}
\pgfpathlineto{\pgfqpointpolar{-54}{\pgfplotmarksize}}
\pgfpathlineto{\pgfqpointpolar{-90}{0.5\pgfplotmarksize}}
\pgfpathlineto{\pgfqpointpolar{234}{\pgfplotmarksize}}
\pgfpathlineto{\pgfqpointpolar{198}{0.5\pgfplotmarksize}}
\pgfpathlineto{\pgfqpointpolar{162}{\pgfplotmarksize}}
\pgfpathlineto{\pgfqpointpolar{134}{0.5\pgfplotmarksize}}
\pgfpathclose
\pgfusepathqfillstroke
}
\definecolor{c}{rgb}{1,1,1};
\draw [color=c, fill=c] (0,0) rectangle (20,14.9811);
\draw [color=c, fill=c] (2.19697,1.53409) rectangle (17.9735,13.4848);
\definecolor{c}{rgb}{0,0,0};
\draw [c,line width=0.9] (2.19697,1.53409) -- (2.19697,13.4848) -- (17.9735,13.4848) -- (17.9735,1.53409) -- (2.19697,1.53409);
\definecolor{c}{rgb}{1,1,1};
\draw [color=c, fill=c] (2.19697,1.53409) rectangle (17.9735,13.4848);
\definecolor{c}{rgb}{0,0,0};
\draw [c,line width=0.9] (2.19697,1.53409) -- (2.19697,13.4848) -- (17.9735,13.4848) -- (17.9735,1.53409) -- (2.19697,1.53409);
\definecolor{c}{rgb}{0,0,0.6};
\draw [c,line width=0.9] (2.19697,2.19424) -- (2.30008,2.19424) -- (2.30008,2.20013) -- (2.4032,2.20013) -- (2.4032,2.18835) -- (2.50631,2.18835) -- (2.50631,2.22725) -- (2.60943,2.22725) -- (2.60943,2.3192) -- (2.71254,2.3192) -- (2.71254,2.30859)
 -- (2.81566,2.30859) -- (2.81566,2.34159) -- (2.91877,2.34159) -- (2.91877,2.46419) -- (3.02189,2.46419) -- (3.02189,2.4583) -- (3.125,2.4583) -- (3.125,2.51724) -- (3.22811,2.51724) -- (3.22811,2.59622) -- (3.33123,2.59622) -- (3.33123,2.62098) --
 (3.43434,2.62098) -- (3.43434,2.71529) -- (3.53746,2.71529) -- (3.53746,2.7648) -- (3.64057,2.7648) -- (3.64057,2.90979) -- (3.74369,2.90979) -- (3.74369,2.99349) -- (3.8468,2.99349) -- (3.8468,3.07365) -- (3.94992,3.07365) -- (3.94992,3.17032) --
 (4.05303,3.17032) -- (4.05303,3.23633) -- (4.15614,3.23633) -- (4.15614,3.38015) -- (4.25926,3.38015) -- (4.25926,3.49214) -- (4.36237,3.49214) -- (4.36237,3.69372) -- (4.46549,3.69372) -- (4.46549,3.71494) -- (4.5686,3.71494) -- (4.5686,3.8788) --
 (4.67172,3.8788) -- (4.67172,4.04973) -- (4.77483,4.04973) -- (4.77483,4.19237) -- (4.87795,4.19237) -- (4.87795,4.36566) -- (4.98106,4.36566) -- (4.98106,4.42578) -- (5.08418,4.42578) -- (5.08418,4.74642) -- (5.18729,4.74642) -- (5.18729,4.92678)
 -- (5.2904,4.92678) -- (5.2904,5.24507) -- (5.39352,5.24507) -- (5.39352,5.3382) -- (5.49663,5.3382) -- (5.49663,5.50913) -- (5.59975,5.50913) -- (5.59975,5.76612) -- (5.70286,5.76612) -- (5.70286,6.06201) -- (5.80598,6.06201) -- (5.80598,6.10209)
 -- (5.90909,6.10209) -- (5.90909,6.42391) -- (6.01221,6.42391) -- (6.01221,6.59366) -- (6.11532,6.59366) -- (6.11532,6.91666) -- (6.21843,6.91666) -- (6.21843,7.04398) -- (6.32155,7.04398) -- (6.32155,7.34812) -- (6.42466,7.34812) --
 (6.42466,7.57445) -- (6.52778,7.57445) -- (6.52778,7.8114) -- (6.63089,7.8114) -- (6.63089,8.1344) -- (6.73401,8.1344) -- (6.73401,8.2794) -- (6.83712,8.2794) -- (6.83712,8.58) -- (6.94024,8.58) -- (6.94024,8.86174) -- (7.04335,8.86174) --
 (7.04335,9.28613) -- (7.14646,9.28613) -- (7.14646,9.41815) -- (7.24958,9.41815) -- (7.24958,9.58791) -- (7.35269,9.58791) -- (7.35269,9.95688) -- (7.45581,9.95688) -- (7.45581,10.004) -- (7.55892,10.004) -- (7.55892,10.3094) -- (7.66204,10.3094) --
 (7.66204,10.3624) -- (7.76515,10.3624) -- (7.76515,10.9141) -- (7.86827,10.9141) -- (7.86827,11.0556) -- (7.97138,11.0556) -- (7.97138,11.085) -- (8.0745,11.085) -- (8.0745,11.4186) -- (8.17761,11.4186) -- (8.17761,11.441) -- (8.28072,11.441) --
 (8.28072,11.6627) -- (8.38384,11.6627) -- (8.38384,11.8784) -- (8.48695,11.8784) -- (8.48695,11.9444) -- (8.59007,11.9444) -- (8.59007,12.0588) -- (8.69318,12.0588) -- (8.69318,12.3311) -- (8.7963,12.3311) -- (8.7963,12.357) -- (8.89941,12.357) --
 (8.89941,12.3935) -- (9.00253,12.3935) -- (9.00253,12.5067) -- (9.10564,12.5067) -- (9.10564,12.7979) -- (9.20875,12.7979) -- (9.20875,12.7554) -- (9.31187,12.7554) -- (9.31187,12.6717) -- (9.41498,12.6717) -- (9.41498,12.9158) -- (9.5181,12.9158)
 -- (9.5181,12.7095) -- (9.62121,12.7095) -- (9.62121,12.7719) -- (9.72433,12.7719) -- (9.72433,12.7979) -- (9.82744,12.7979) -- (9.82744,12.6717) -- (9.93056,12.6717) -- (9.93056,12.6847) -- (10.0337,12.6847) -- (10.0337,12.6623) --
 (10.1368,12.6623) -- (10.1368,12.3452) -- (10.2399,12.3452) -- (10.2399,12.4478) -- (10.343,12.4478) -- (10.343,12.4395) -- (10.4461,12.4395) -- (10.4461,12.3617) -- (10.5492,12.3617) -- (10.5492,12.199) -- (10.6524,12.199) -- (10.6524,11.8902) --
 (10.7555,11.8902) -- (10.7555,11.6179) -- (10.8586,11.6179) -- (10.8586,11.6603) -- (10.9617,11.6603) -- (10.9617,11.487) -- (11.0648,11.487) -- (11.0648,11.4092) -- (11.1679,11.4092) -- (11.1679,11.032) -- (11.271,11.032) -- (11.271,10.8587) --
 (11.3742,10.8587) -- (11.3742,10.7502) -- (11.4773,10.7502) -- (11.4773,10.5487) -- (11.5804,10.5487) -- (11.5804,10.3282) -- (11.6835,10.3282) -- (11.6835,10.1549) -- (11.7866,10.1549) -- (11.7866,9.80363) -- (11.8897,9.80363) -- (11.8897,9.74823)
 -- (11.9928,9.74823) -- (11.9928,9.32621) -- (12.096,9.32621) -- (12.096,9.07865) -- (12.1991,9.07865) -- (12.1991,8.92422) -- (12.3022,8.92422) -- (12.3022,8.66606) -- (12.4053,8.66606) -- (12.4053,8.29119) -- (12.5084,8.29119) -- (12.5084,8.21574)
 -- (12.6115,8.21574) -- (12.6115,7.89392) -- (12.7146,7.89392) -- (12.7146,7.74303) -- (12.8178,7.74303) -- (12.8178,7.53084) -- (12.9209,7.53084) -- (12.9209,7.10881) -- (13.024,7.10881) -- (13.024,7.08052) -- (13.1271,7.08052) -- (13.1271,6.63492)
 -- (13.2302,6.63492) -- (13.2302,6.54651) -- (13.3333,6.54651) -- (13.3333,6.42391) -- (13.4364,6.42391) -- (13.4364,6.03607) -- (13.5396,6.03607) -- (13.5396,5.90404) -- (13.6427,5.90404) -- (13.6427,5.65295) -- (13.7458,5.65295) --
 (13.7458,5.43251) -- (13.8489,5.43251) -- (13.8489,5.23328) -- (13.952,5.23328) -- (13.952,5.10597) -- (14.0551,5.10597) -- (14.0551,4.87727) -- (14.1582,4.87727) -- (14.1582,4.78297) -- (14.2614,4.78297) -- (14.2614,4.51066) -- (14.3645,4.51066) --
 (14.3645,4.33029) -- (14.4676,4.33029) -- (14.4676,4.31143) -- (14.5707,4.31143) -- (14.5707,4.06152) -- (14.6738,4.06152) -- (14.6738,3.97664) -- (14.7769,3.97664) -- (14.7769,3.8175) -- (14.8801,3.8175) -- (14.8801,3.62888) -- (14.9832,3.62888) --
 (14.9832,3.54519) -- (15.0863,3.54519) -- (15.0863,3.38251) -- (15.1894,3.38251) -- (15.1894,3.28466) -- (15.2925,3.28466) -- (15.2925,3.16207) -- (15.3956,3.16207) -- (15.3956,3.08073) -- (15.4987,3.08073) -- (15.4987,2.96638) -- (15.6019,2.96638)
 -- (15.6019,2.96166) -- (15.705,2.96166) -- (15.705,2.88386) -- (15.8081,2.88386) -- (15.8081,2.75537) -- (15.9112,2.75537) -- (15.9112,2.67167) -- (16.0143,2.67167) -- (16.0143,2.57736) -- (16.1174,2.57736) -- (16.1174,2.46773) -- (16.2205,2.46773)
 -- (16.2205,2.41232) -- (16.3237,2.41232) -- (16.3237,2.34395) -- (16.4268,2.34395) -- (16.4268,2.34749) -- (16.5299,2.34749) -- (16.5299,2.23078) -- (16.633,2.23078) -- (16.633,2.26143) -- (16.7361,2.26143) -- (16.7361,2.16359) -- (16.8392,2.16359)
 -- (16.8392,2.16713) -- (16.9423,2.16713) -- (16.9423,2.09168) -- (17.0455,2.09168) -- (17.0455,2.06928) -- (17.1486,2.06928) -- (17.1486,2.01977) -- (17.2517,2.01977) -- (17.2517,2.00916) -- (17.3548,2.00916) -- (17.3548,1.95965) --
 (17.4579,1.95965) -- (17.4579,1.95376) -- (17.561,1.95376) -- (17.561,1.91014) -- (17.6641,1.91014) -- (17.6641,1.92429) -- (17.7673,1.92429) -- (17.7673,1.86888) -- (17.8704,1.86888) -- (17.8704,1.85827) -- (17.9735,1.85827);
\definecolor{c}{rgb}{1,1,1};
\draw [color=c, fill=c] (13.2197,9.45076) rectangle (18.5985,13.6742);
\definecolor{c}{rgb}{0,0,0};
\draw [c,line width=0.9] (13.2197,9.45076) -- (18.5985,9.45076);
\draw [c,line width=0.9] (18.5985,9.45076) -- (18.5985,13.6742);
\draw [c,line width=0.9] (18.5985,13.6742) -- (13.2197,13.6742);
\draw [c,line width=0.9] (13.2197,13.6742) -- (13.2197,9.45076);
\draw [anchor= west] (13.4886,13.1463) node[scale=0.967468, color=c, rotate=0]{Entries };
\draw [anchor= east] (18.3295,13.1463) node[scale=0.967468, color=c, rotate=0]{ 5011262};
\draw [anchor= west] (13.4886,12.0904) node[scale=0.967468, color=c, rotate=0]{Constant };
\draw [anchor= east] (18.3295,12.0904) node[scale=0.967468, color=c, rotate=0]{$  9511 \pm 15.7$};
\draw [anchor= west] (13.4886,11.0346) node[scale=0.967468, color=c, rotate=0]{Mean     };
\draw [anchor= east] (18.3295,11.0346) node[scale=0.967468, color=c, rotate=0]{$ 1.278e+04 \pm 0.4$};
\draw [anchor= west] (13.4886,9.97869) node[scale=0.967468, color=c, rotate=0]{Sigma    };
\draw [anchor= east] (18.3295,9.97869) node[scale=0.967468, color=c, rotate=0]{$   278 \pm 0.4$};
\definecolor{c}{rgb}{1,0,0};
\draw [c,line width=1.8] (3.90455,3.03837) -- (4.02003,3.16355) -- (4.13552,3.29628) -- (4.25101,3.43672) -- (4.3665,3.58504) -- (4.48199,3.74132) -- (4.59747,3.90567) -- (4.71296,4.07812) -- (4.82845,4.25869) -- (4.94394,4.44734) -- (5.05943,4.644)
 -- (5.17492,4.84855) -- (5.2904,5.06084) -- (5.40589,5.28063) -- (5.52138,5.50766) -- (5.63687,5.74163) -- (5.75236,5.98214) -- (5.86785,6.22877) -- (5.98333,6.48105) -- (6.09882,6.73843) -- (6.21431,7.00033) -- (6.3298,7.26609) -- (6.44529,7.53503)
 -- (6.56077,7.8064) -- (6.67626,8.07941) -- (6.79175,8.35323) -- (6.90724,8.62698) -- (7.02273,8.89976) -- (7.13822,9.17063) -- (7.2537,9.43863) -- (7.36919,9.70276) -- (7.48468,9.96202) -- (7.60017,10.2154) -- (7.71566,10.4619) -- (7.83114,10.7006)
 -- (7.94663,10.9303) -- (8.06212,11.1502) -- (8.17761,11.3593) -- (8.2931,11.5566) -- (8.40859,11.7414) -- (8.52407,11.9126) -- (8.63956,12.0697) -- (8.75505,12.2117) -- (8.87054,12.3382) -- (8.98603,12.4484) -- (9.10151,12.5419) -- (9.217,12.6181)
 -- (9.33249,12.6768) -- (9.44798,12.7177) -- (9.56347,12.7404);
\draw [c,line width=1.8] (9.56347,12.7404) -- (9.67896,12.7451) -- (9.79444,12.7315) -- (9.90993,12.6998) -- (10.0254,12.6502) -- (10.1409,12.5828) -- (10.2564,12.4981) -- (10.3719,12.3963) -- (10.4874,12.2781) -- (10.6029,12.1439) --
 (10.7184,11.9943) -- (10.8338,11.8302) -- (10.9493,11.6522) -- (11.0648,11.4612) -- (11.1803,11.258) -- (11.2958,11.0435) -- (11.4113,10.8186) -- (11.5268,10.5844) -- (11.6423,10.3417) -- (11.7577,10.0917) -- (11.8732,9.8353) -- (11.9887,9.57352) --
 (12.1042,9.30737) -- (12.2197,9.03784) -- (12.3352,8.76591) -- (12.4507,8.49254) -- (12.5662,8.21864) -- (12.6817,7.94511) -- (12.7971,7.67281) -- (12.9126,7.40253) -- (13.0281,7.13507) -- (13.1436,6.87112) -- (13.2591,6.61137) -- (13.3746,6.35642)
 -- (13.4901,6.10685) -- (13.6056,5.86317) -- (13.721,5.62583) -- (13.8365,5.39522) -- (13.952,5.17171) -- (14.0675,4.95558) -- (14.183,4.74707) -- (14.2985,4.54638) -- (14.414,4.35364) -- (14.5295,4.16895) -- (14.6449,3.99237) -- (14.7604,3.82391)
 -- (14.8759,3.66353) -- (14.9914,3.51117) -- (15.1069,3.36674) -- (15.2224,3.2301);
\draw [c,line width=1.8] (15.2224,3.2301) -- (15.3379,3.10111);
\definecolor{c}{rgb}{0,0,0};
\draw [c,line width=0.9] (2.19697,1.53409) -- (17.9735,1.53409);
\draw [anchor= east] (17.9735,0.695152) node[scale=1.17779, color=c, rotate=0]{Canali};
\draw [c,line width=0.9] (3.64057,1.88861) -- (3.64057,1.53409);
\draw [c,line width=0.9] (4.15614,1.71135) -- (4.15614,1.53409);
\draw [c,line width=0.9] (4.67172,1.71135) -- (4.67172,1.53409);
\draw [c,line width=0.9] (5.18729,1.71135) -- (5.18729,1.53409);
\draw [c,line width=0.9] (5.70286,1.88861) -- (5.70286,1.53409);
\draw [c,line width=0.9] (6.21843,1.71135) -- (6.21843,1.53409);
\draw [c,line width=0.9] (6.73401,1.71135) -- (6.73401,1.53409);
\draw [c,line width=0.9] (7.24958,1.71135) -- (7.24958,1.53409);
\draw [c,line width=0.9] (7.76515,1.88861) -- (7.76515,1.53409);
\draw [c,line width=0.9] (8.28072,1.71135) -- (8.28072,1.53409);
\draw [c,line width=0.9] (8.7963,1.71135) -- (8.7963,1.53409);
\draw [c,line width=0.9] (9.31187,1.71135) -- (9.31187,1.53409);
\draw [c,line width=0.9] (9.82744,1.88861) -- (9.82744,1.53409);
\draw [c,line width=0.9] (10.343,1.71135) -- (10.343,1.53409);
\draw [c,line width=0.9] (10.8586,1.71135) -- (10.8586,1.53409);
\draw [c,line width=0.9] (11.3742,1.71135) -- (11.3742,1.53409);
\draw [c,line width=0.9] (11.8897,1.88861) -- (11.8897,1.53409);
\draw [c,line width=0.9] (12.4053,1.71135) -- (12.4053,1.53409);
\draw [c,line width=0.9] (12.9209,1.71135) -- (12.9209,1.53409);
\draw [c,line width=0.9] (13.4364,1.71135) -- (13.4364,1.53409);
\draw [c,line width=0.9] (13.952,1.88861) -- (13.952,1.53409);
\draw [c,line width=0.9] (14.4676,1.71135) -- (14.4676,1.53409);
\draw [c,line width=0.9] (14.9832,1.71135) -- (14.9832,1.53409);
\draw [c,line width=0.9] (15.4987,1.71135) -- (15.4987,1.53409);
\draw [c,line width=0.9] (16.0143,1.88861) -- (16.0143,1.53409);
\draw [c,line width=0.9] (3.64057,1.88861) -- (3.64057,1.53409);
\draw [c,line width=0.9] (3.125,1.71135) -- (3.125,1.53409);
\draw [c,line width=0.9] (2.60943,1.71135) -- (2.60943,1.53409);
\draw [c,line width=0.9] (16.0143,1.88861) -- (16.0143,1.53409);
\draw [c,line width=0.9] (16.5299,1.71135) -- (16.5299,1.53409);
\draw [c,line width=0.9] (17.0455,1.71135) -- (17.0455,1.53409);
\draw [c,line width=0.9] (17.561,1.71135) -- (17.561,1.53409);
\draw [anchor=base] (3.64057,1.09964) node[scale=1.00953, color=c, rotate=0]{12200};
\draw [anchor=base] (5.70286,1.09964) node[scale=1.00953, color=c, rotate=0]{12400};
\draw [anchor=base] (7.76515,1.09964) node[scale=1.00953, color=c, rotate=0]{12600};
\draw [anchor=base] (9.82744,1.09964) node[scale=1.00953, color=c, rotate=0]{12800};
\draw [anchor=base] (11.8897,1.09964) node[scale=1.00953, color=c, rotate=0]{13000};
\draw [anchor=base] (13.952,1.09964) node[scale=1.00953, color=c, rotate=0]{13200};
\draw [anchor=base] (16.0143,1.09964) node[scale=1.00953, color=c, rotate=0]{13400};
\draw [c,line width=0.9] (2.19697,1.53409) -- (2.19697,13.4848);
\draw [anchor= east] (0.59697,13.4848) node[scale=1.68255, color=c, rotate=90]{Eventi};
\draw [c,line width=0.9] (2.6756,1.53409) -- (2.19697,1.53409);
\draw [c,line width=0.9] (2.43629,2.12351) -- (2.19697,2.12351);
\draw [c,line width=0.9] (2.43629,2.71293) -- (2.19697,2.71293);
\draw [c,line width=0.9] (2.43629,3.30235) -- (2.19697,3.30235);
\draw [c,line width=0.9] (2.6756,3.89177) -- (2.19697,3.89177);
\draw [c,line width=0.9] (2.43629,4.48118) -- (2.19697,4.48118);
\draw [c,line width=0.9] (2.43629,5.0706) -- (2.19697,5.0706);
\draw [c,line width=0.9] (2.43629,5.66002) -- (2.19697,5.66002);
\draw [c,line width=0.9] (2.6756,6.24944) -- (2.19697,6.24944);
\draw [c,line width=0.9] (2.43629,6.83886) -- (2.19697,6.83886);
\draw [c,line width=0.9] (2.43629,7.42828) -- (2.19697,7.42828);
\draw [c,line width=0.9] (2.43629,8.0177) -- (2.19697,8.0177);
\draw [c,line width=0.9] (2.6756,8.60711) -- (2.19697,8.60711);
\draw [c,line width=0.9] (2.43629,9.19653) -- (2.19697,9.19653);
\draw [c,line width=0.9] (2.43629,9.78595) -- (2.19697,9.78595);
\draw [c,line width=0.9] (2.43629,10.3754) -- (2.19697,10.3754);
\draw [c,line width=0.9] (2.6756,10.9648) -- (2.19697,10.9648);
\draw [c,line width=0.9] (2.43629,11.5542) -- (2.19697,11.5542);
\draw [c,line width=0.9] (2.43629,12.1436) -- (2.19697,12.1436);
\draw [c,line width=0.9] (2.43629,12.733) -- (2.19697,12.733);
\draw [c,line width=0.9] (2.6756,13.3225) -- (2.19697,13.3225);
\draw [c,line width=0.9] (2.6756,13.3225) -- (2.19697,13.3225);
\draw [anchor= east] (2.09697,1.53409) node[scale=1.00953, color=c, rotate=0]{0};
\draw [anchor= east] (2.09697,3.89177) node[scale=1.00953, color=c, rotate=0]{2000};
\draw [anchor= east] (2.09697,6.24944) node[scale=1.00953, color=c, rotate=0]{4000};
\draw [anchor= east] (2.09697,8.60711) node[scale=1.00953, color=c, rotate=0]{6000};
\draw [anchor= east] (2.09697,10.9648) node[scale=1.00953, color=c, rotate=0]{8000};
\draw [anchor= east] (2.09697,13.3225) node[scale=1.00953, color=c, rotate=0]{10000};
\definecolor{c}{rgb}{1,1,1};
\draw [color=c, fill=c] (13.2197,9.45076) rectangle (18.5985,13.6742);
\definecolor{c}{rgb}{0,0,0};
\draw [c,line width=0.9] (13.2197,9.45076) -- (18.5985,9.45076);
\draw [c,line width=0.9] (18.5985,9.45076) -- (18.5985,13.6742);
\draw [c,line width=0.9] (18.5985,13.6742) -- (13.2197,13.6742);
\draw [c,line width=0.9] (13.2197,13.6742) -- (13.2197,9.45076);
\draw [anchor= west] (13.4886,13.1463) node[scale=0.967468, color=c, rotate=0]{Entries };
\draw [anchor= east] (18.3295,13.1463) node[scale=0.967468, color=c, rotate=0]{ 5011262};
\draw [anchor= west] (13.4886,12.0904) node[scale=0.967468, color=c, rotate=0]{Constant };
\draw [anchor= east] (18.3295,12.0904) node[scale=0.967468, color=c, rotate=0]{$  9511 \pm 15.7$};
\draw [anchor= west] (13.4886,11.0346) node[scale=0.967468, color=c, rotate=0]{Mean     };
\draw [anchor= east] (18.3295,11.0346) node[scale=0.967468, color=c, rotate=0]{$ 1.278e+04 \pm 0.4$};
\draw [anchor= west] (13.4886,9.97869) node[scale=0.967468, color=c, rotate=0]{Sigma    };
\draw [anchor= east] (18.3295,9.97869) node[scale=0.967468, color=c, rotate=0]{$   278 \pm 0.4$};
\draw (7.075,14.2992) node[scale=1.55636, color=c, rotate=0]{Cobalto R1 Picco 1333 keV};
\end{tikzpicture}

\begin{tikzpicture}
\pgfdeclareplotmark{cross} {
\pgfpathmoveto{\pgfpoint{-0.3\pgfplotmarksize}{\pgfplotmarksize}}
\pgfpathlineto{\pgfpoint{+0.3\pgfplotmarksize}{\pgfplotmarksize}}
\pgfpathlineto{\pgfpoint{+0.3\pgfplotmarksize}{0.3\pgfplotmarksize}}
\pgfpathlineto{\pgfpoint{+1\pgfplotmarksize}{0.3\pgfplotmarksize}}
\pgfpathlineto{\pgfpoint{+1\pgfplotmarksize}{-0.3\pgfplotmarksize}}
\pgfpathlineto{\pgfpoint{+0.3\pgfplotmarksize}{-0.3\pgfplotmarksize}}
\pgfpathlineto{\pgfpoint{+0.3\pgfplotmarksize}{-1.\pgfplotmarksize}}
\pgfpathlineto{\pgfpoint{-0.3\pgfplotmarksize}{-1.\pgfplotmarksize}}
\pgfpathlineto{\pgfpoint{-0.3\pgfplotmarksize}{-0.3\pgfplotmarksize}}
\pgfpathlineto{\pgfpoint{-1.\pgfplotmarksize}{-0.3\pgfplotmarksize}}
\pgfpathlineto{\pgfpoint{-1.\pgfplotmarksize}{0.3\pgfplotmarksize}}
\pgfpathlineto{\pgfpoint{-0.3\pgfplotmarksize}{0.3\pgfplotmarksize}}
\pgfpathclose
\pgfusepathqstroke
}
\pgfdeclareplotmark{cross*} {
\pgfpathmoveto{\pgfpoint{-0.3\pgfplotmarksize}{\pgfplotmarksize}}
\pgfpathlineto{\pgfpoint{+0.3\pgfplotmarksize}{\pgfplotmarksize}}
\pgfpathlineto{\pgfpoint{+0.3\pgfplotmarksize}{0.3\pgfplotmarksize}}
\pgfpathlineto{\pgfpoint{+1\pgfplotmarksize}{0.3\pgfplotmarksize}}
\pgfpathlineto{\pgfpoint{+1\pgfplotmarksize}{-0.3\pgfplotmarksize}}
\pgfpathlineto{\pgfpoint{+0.3\pgfplotmarksize}{-0.3\pgfplotmarksize}}
\pgfpathlineto{\pgfpoint{+0.3\pgfplotmarksize}{-1.\pgfplotmarksize}}
\pgfpathlineto{\pgfpoint{-0.3\pgfplotmarksize}{-1.\pgfplotmarksize}}
\pgfpathlineto{\pgfpoint{-0.3\pgfplotmarksize}{-0.3\pgfplotmarksize}}
\pgfpathlineto{\pgfpoint{-1.\pgfplotmarksize}{-0.3\pgfplotmarksize}}
\pgfpathlineto{\pgfpoint{-1.\pgfplotmarksize}{0.3\pgfplotmarksize}}
\pgfpathlineto{\pgfpoint{-0.3\pgfplotmarksize}{0.3\pgfplotmarksize}}
\pgfpathclose
\pgfusepathqfillstroke
}
\pgfdeclareplotmark{newstar} {
\pgfpathmoveto{\pgfqpoint{0pt}{\pgfplotmarksize}}
\pgfpathlineto{\pgfqpointpolar{44}{0.5\pgfplotmarksize}}
\pgfpathlineto{\pgfqpointpolar{18}{\pgfplotmarksize}}
\pgfpathlineto{\pgfqpointpolar{-20}{0.5\pgfplotmarksize}}
\pgfpathlineto{\pgfqpointpolar{-54}{\pgfplotmarksize}}
\pgfpathlineto{\pgfqpointpolar{-90}{0.5\pgfplotmarksize}}
\pgfpathlineto{\pgfqpointpolar{234}{\pgfplotmarksize}}
\pgfpathlineto{\pgfqpointpolar{198}{0.5\pgfplotmarksize}}
\pgfpathlineto{\pgfqpointpolar{162}{\pgfplotmarksize}}
\pgfpathlineto{\pgfqpointpolar{134}{0.5\pgfplotmarksize}}
\pgfpathclose
\pgfusepathqstroke
}
\pgfdeclareplotmark{newstar*} {
\pgfpathmoveto{\pgfqpoint{0pt}{\pgfplotmarksize}}
\pgfpathlineto{\pgfqpointpolar{44}{0.5\pgfplotmarksize}}
\pgfpathlineto{\pgfqpointpolar{18}{\pgfplotmarksize}}
\pgfpathlineto{\pgfqpointpolar{-20}{0.5\pgfplotmarksize}}
\pgfpathlineto{\pgfqpointpolar{-54}{\pgfplotmarksize}}
\pgfpathlineto{\pgfqpointpolar{-90}{0.5\pgfplotmarksize}}
\pgfpathlineto{\pgfqpointpolar{234}{\pgfplotmarksize}}
\pgfpathlineto{\pgfqpointpolar{198}{0.5\pgfplotmarksize}}
\pgfpathlineto{\pgfqpointpolar{162}{\pgfplotmarksize}}
\pgfpathlineto{\pgfqpointpolar{134}{0.5\pgfplotmarksize}}
\pgfpathclose
\pgfusepathqfillstroke
}
\definecolor{c}{rgb}{1,1,1};
\draw [color=c, fill=c] (0,0) rectangle (20,14.6047);
\draw [color=c, fill=c] (2,1.46047) rectangle (18,13.1443);
\definecolor{c}{rgb}{0,0,0};
\draw [c,line width=0.9] (2,1.46047) -- (2,13.1443) -- (18,13.1443) -- (18,1.46047) -- (2,1.46047);
\definecolor{c}{rgb}{1,1,1};
\draw [color=c, fill=c] (2,1.46047) rectangle (18,13.1443);
\definecolor{c}{rgb}{0,0,0};
\draw [c,line width=0.9] (2,1.46047) -- (2,13.1443) -- (18,13.1443) -- (18,1.46047) -- (2,1.46047);
\definecolor{c}{rgb}{0,0,0.6};
\draw [c,line width=0.9] (2,2.97477) -- (2.10884,2.97477) -- (2.10884,3.05187) -- (2.21769,3.05187) -- (2.21769,3.06894) -- (2.32653,3.06894) -- (2.32653,3.15075) -- (2.43537,3.15075) -- (2.43537,3.24256) -- (2.54422,3.24256) -- (2.54422,3.36909) --
 (2.65306,3.36909) -- (2.65306,3.34025) -- (2.7619,3.34025) -- (2.7619,3.45502) -- (2.87075,3.45502) -- (2.87075,3.50916) -- (2.97959,3.50916) -- (2.97959,3.63276) -- (3.08844,3.63276) -- (3.08844,3.63334) -- (3.19728,3.63334) -- (3.19728,3.83168) --
 (3.30612,3.83168) -- (3.30612,3.90172) -- (3.41497,3.90172) -- (3.41497,3.9694) -- (3.52381,3.9694) -- (3.52381,4.17126) -- (3.63265,4.17126) -- (3.63265,4.2313) -- (3.7415,4.2313) -- (3.7415,4.37254) -- (3.85034,4.37254) -- (3.85034,4.52262) --
 (3.95918,4.52262) -- (3.95918,4.66387) -- (4.06803,4.66387) -- (4.06803,4.84337) -- (4.17687,4.84337) -- (4.17687,4.87751) -- (4.28571,4.87751) -- (4.28571,5.05171) -- (4.39456,5.05171) -- (4.39456,5.22415) -- (4.5034,5.22415) -- (4.5034,5.42367) --
 (4.61225,5.42367) -- (4.61225,5.61729) -- (4.72109,5.61729) -- (4.72109,5.83564) -- (4.82993,5.83564) -- (4.82993,5.98572) -- (4.93878,5.98572) -- (4.93878,6.19818) -- (5.04762,6.19818) -- (5.04762,6.33766) -- (5.15646,6.33766) -- (5.15646,6.49127)
 -- (5.26531,6.49127) -- (5.26531,6.8232) -- (5.37415,6.8232) -- (5.37415,6.93326) -- (5.48299,6.93326) -- (5.48299,7.11865) -- (5.59184,7.11865) -- (5.59184,7.3417) -- (5.70068,7.3417) -- (5.70068,7.71071) -- (5.80952,7.71071) -- (5.80952,7.93847)
 -- (5.91837,7.93847) -- (5.91837,8.04676) -- (6.02721,8.04676) -- (6.02721,8.29042) -- (6.13605,8.29042) -- (6.13605,8.49464) -- (6.2449,8.49464) -- (6.2449,8.87895) -- (6.35374,8.87895) -- (6.35374,9.04021) -- (6.46258,9.04021) -- (6.46258,9.2874)
 -- (6.57143,9.2874) -- (6.57143,9.45513) -- (6.68027,9.45513) -- (6.68027,9.80001) -- (6.78912,9.80001) -- (6.78912,9.94067) -- (6.89796,9.94067) -- (6.89796,10.1778) -- (7.0068,10.1778) -- (7.0068,10.5086) -- (7.11565,10.5086) -- (7.11565,10.6151)
 -- (7.22449,10.6151) -- (7.22449,10.8282) -- (7.33333,10.8282) -- (7.33333,10.9065) -- (7.44218,10.9065) -- (7.44218,11.2631) -- (7.55102,11.2631) -- (7.55102,11.465) -- (7.65986,11.465) -- (7.65986,11.4597) -- (7.76871,11.4597) -- (7.76871,11.8387)
 -- (7.87755,11.8387) -- (7.87755,11.9152) -- (7.98639,11.9152) -- (7.98639,12.0105) -- (8.09524,12.0105) -- (8.09524,12.1594) -- (8.20408,12.1594) -- (8.20408,12.1965) -- (8.31293,12.1965) -- (8.31293,12.3678) -- (8.42177,12.3678) --
 (8.42177,12.3395) -- (8.53061,12.3395) -- (8.53061,12.2925) -- (8.63946,12.2925) -- (8.63946,12.5667) -- (8.7483,12.5667) -- (8.7483,12.479) -- (8.85714,12.479) -- (8.85714,12.502) -- (8.96599,12.502) -- (8.96599,12.4708) -- (9.07483,12.4708) --
 (9.07483,12.5785) -- (9.18367,12.5785) -- (9.18367,12.5879) -- (9.29252,12.5879) -- (9.29252,12.5026) -- (9.40136,12.5026) -- (9.40136,12.1694) -- (9.5102,12.1694) -- (9.5102,12.0676) -- (9.61905,12.0676) -- (9.61905,12.14) -- (9.72789,12.14) --
 (9.72789,11.947) -- (9.83673,11.947) -- (9.83673,11.7516) -- (9.94558,11.7516) -- (9.94558,11.5362) -- (10.0544,11.5362) -- (10.0544,11.4597) -- (10.1633,11.4597) -- (10.1633,11.2943) -- (10.2721,11.2943) -- (10.2721,11.0295) -- (10.381,11.0295) --
 (10.381,10.8341) -- (10.4898,10.8341) -- (10.4898,10.5451) -- (10.5986,10.5451) -- (10.5986,10.3167) -- (10.7075,10.3167) -- (10.7075,10.1973) -- (10.8163,10.1973) -- (10.8163,9.83826) -- (10.9252,9.83826) -- (10.9252,9.61639) -- (11.034,9.61639) --
 (11.034,9.3539) -- (11.1429,9.3539) -- (11.1429,9.09377) -- (11.2517,9.09377) -- (11.2517,8.75359) -- (11.3605,8.75359) -- (11.3605,8.57586) -- (11.4694,8.57586) -- (11.4694,8.271) -- (11.5782,8.271) -- (11.5782,7.93847) -- (11.6871,7.93847) --
 (11.6871,7.77192) -- (11.7959,7.77192) -- (11.7959,7.46824) -- (11.9048,7.46824) -- (11.9048,7.12041) -- (12.0136,7.12041) -- (12.0136,6.80613) -- (12.1224,6.80613) -- (12.1224,6.65665) -- (12.2313,6.65665) -- (12.2313,6.31) -- (12.3401,6.31) --
 (12.3401,6.11931) -- (12.449,6.11931) -- (12.449,5.87684) -- (12.5578,5.87684) -- (12.5578,5.53137) -- (12.6667,5.53137) -- (12.6667,5.39306) -- (12.7755,5.39306) -- (12.7755,5.23887) -- (12.8844,5.23887) -- (12.8844,4.92282) -- (12.9932,4.92282) --
 (12.9932,4.77804) -- (13.102,4.77804) -- (13.102,4.60207) -- (13.2109,4.60207) -- (13.2109,4.40197) -- (13.3197,4.40197) -- (13.3197,4.19481) -- (13.4286,4.19481) -- (13.4286,3.99058) -- (13.5374,3.99058) -- (13.5374,3.83698) -- (13.6463,3.83698) --
 (13.6463,3.7022) -- (13.7551,3.7022) -- (13.7551,3.56566) -- (13.8639,3.56566) -- (13.8639,3.33201) -- (13.9728,3.33201) -- (13.9728,3.25021) -- (14.0816,3.25021) -- (14.0816,3.1172) -- (14.1905,3.1172) -- (14.1905,3.0248) -- (14.2993,3.0248) --
 (14.2993,2.8659) -- (14.4082,2.8659) -- (14.4082,2.75054) -- (14.517,2.75054) -- (14.517,2.67109) -- (14.6259,2.67109) -- (14.6259,2.61224) -- (14.7347,2.61224) -- (14.7347,2.48335) -- (14.8435,2.48335) -- (14.8435,2.40625) -- (14.9524,2.40625) --
 (14.9524,2.37035) -- (15.0612,2.37035) -- (15.0612,2.30855) -- (15.1701,2.30855) -- (15.1701,2.25382) -- (15.2789,2.25382) -- (15.2789,2.20379) -- (15.3878,2.20379) -- (15.3878,2.13435) -- (15.4966,2.13435) -- (15.4966,2.0802) -- (15.6054,2.0802) --
 (15.6054,2.05666) -- (15.7143,2.05666) -- (15.7143,2.05607) -- (15.8231,2.05607) -- (15.8231,1.97544) -- (15.932,1.97544) -- (15.932,1.92601) -- (16.0408,1.92601) -- (16.0408,1.90011) -- (16.1497,1.90011) -- (16.1497,1.87127) -- (16.2585,1.87127) --
 (16.2585,1.87716) -- (16.3673,1.87716) -- (16.3673,1.85656) -- (16.4762,1.85656) -- (16.4762,1.79241) -- (16.585,1.79241) -- (16.585,1.82537) -- (16.6939,1.82537) -- (16.6939,1.77711) -- (16.8027,1.77711) -- (16.8027,1.77593) -- (16.9116,1.77593) --
 (16.9116,1.76475) -- (17.0204,1.76475) -- (17.0204,1.77063) -- (17.1293,1.77063) -- (17.1293,1.77004) -- (17.2381,1.77004) -- (17.2381,1.71354) -- (17.3469,1.71354) -- (17.3469,1.75239) -- (17.4558,1.75239) -- (17.4558,1.72178) -- (17.5646,1.72178)
 -- (17.5646,1.71825) -- (17.6735,1.71825) -- (17.6735,1.69883) -- (17.7823,1.69883) -- (17.7823,1.72002) -- (17.8912,1.72002) -- (17.8912,1.70001) -- (18,1.70001);
\definecolor{c}{rgb}{1,1,1};
\draw [color=c, fill=c] (13.3004,8.12253) rectangle (18.6166,13.3992);
\definecolor{c}{rgb}{0,0,0};
\draw [c,line width=0.9] (13.3004,8.12253) -- (18.6166,8.12253);
\draw [c,line width=0.9] (18.6166,8.12253) -- (18.6166,13.3992);
\draw [c,line width=0.9] (18.6166,13.3992) -- (13.3004,13.3992);
\draw [c,line width=0.9] (13.3004,13.3992) -- (13.3004,8.12253);
\draw [anchor= west] (13.5662,12.7396) node[scale=0.790068, color=c, rotate=0]{Entries };
\draw [anchor= east] (18.3508,12.7396) node[scale=0.790068, color=c, rotate=0]{ 1763855};
\draw [anchor= west] (13.5662,11.4205) node[scale=0.790068, color=c, rotate=0]{Constant };
\draw [anchor= east] (18.3508,11.4205) node[scale=0.790068, color=c, rotate=0]{$ 1.879e+04 \pm 2.335e+01$};
\draw [anchor= west] (13.5662,10.1013) node[scale=0.790068, color=c, rotate=0]{Mean     };
\draw [anchor= east] (18.3508,10.1013) node[scale=0.790068, color=c, rotate=0]{$ 572.8 \pm 0.1$};
\draw [anchor= west] (13.5662,8.78211) node[scale=0.790068, color=c, rotate=0]{Sigma    };
\draw [anchor= east] (18.3508,8.78211) node[scale=0.790068, color=c, rotate=0]{$ 50.63 \pm 0.07$};
\definecolor{c}{rgb}{1,0,0};
\draw [c,line width=1.8] (5.42422,6.70348) -- (5.52435,6.93791) -- (5.62449,7.17527) -- (5.72463,7.41504) -- (5.82476,7.65669) -- (5.9249,7.89963) -- (6.02503,8.14327) -- (6.12517,8.38696) -- (6.22531,8.63007) -- (6.32544,8.87192) --
 (6.42558,9.11181) -- (6.52571,9.34904) -- (6.62585,9.58289) -- (6.72599,9.81263) -- (6.82612,10.0375) -- (6.92626,10.2569) -- (7.02639,10.4699) -- (7.12653,10.6759) -- (7.22667,10.8742) -- (7.3268,11.0641) -- (7.42694,11.2448) -- (7.52707,11.4158)
 -- (7.62721,11.5764) -- (7.72735,11.7261) -- (7.82748,11.8642) -- (7.92762,11.9902) -- (8.02775,12.1037) -- (8.12789,12.2043) -- (8.22803,12.2914) -- (8.32816,12.3649) -- (8.4283,12.4243) -- (8.52844,12.4696) -- (8.62857,12.5004) --
 (8.72871,12.5167) -- (8.82884,12.5184) -- (8.92898,12.5055) -- (9.02912,12.4781) -- (9.12925,12.4362) -- (9.22939,12.3801) -- (9.32952,12.3099) -- (9.42966,12.2259) -- (9.5298,12.1285) -- (9.62993,12.018) -- (9.73007,11.8949) -- (9.8302,11.7596) --
 (9.93034,11.6126) -- (10.0305,11.4545) -- (10.1306,11.2859) -- (10.2307,11.1073) -- (10.3309,10.9196);
\draw [c,line width=1.8] (10.3309,10.9196) -- (10.431,10.7232) -- (10.5312,10.5189) -- (10.6313,10.3075) -- (10.7314,10.0896) -- (10.8316,9.8659) -- (10.9317,9.63723) -- (11.0318,9.40428) -- (11.132,9.16778) -- (11.2321,8.92845) -- (11.3322,8.687) --
 (11.4324,8.44412) -- (11.5325,8.2005) -- (11.6327,7.9568) -- (11.7328,7.71364) -- (11.8329,7.47163) -- (11.9331,7.23137) -- (12.0332,6.99339) -- (12.1333,6.75823) -- (12.2335,6.52637) -- (12.3336,6.29826) -- (12.4337,6.07432) -- (12.5339,5.85494) --
 (12.634,5.64047) -- (12.7341,5.43121) -- (12.8343,5.22746) -- (12.9344,5.02944) -- (13.0346,4.83737) -- (13.1347,4.65141) -- (13.2348,4.47172) -- (13.335,4.2984) -- (13.4351,4.13152) -- (13.5352,3.97113) -- (13.6354,3.81726) -- (13.7355,3.66991) --
 (13.8356,3.52903) -- (13.9358,3.39457) -- (14.0359,3.26647) -- (14.1361,3.14463) -- (14.2362,3.02894) -- (14.3363,2.91926) -- (14.4365,2.81546) -- (14.5366,2.71739) -- (14.6367,2.62488) -- (14.7369,2.53775) -- (14.837,2.45583) -- (14.9371,2.37892)
 -- (15.0373,2.30683) -- (15.1374,2.23938) -- (15.2376,2.17635);
\draw [c,line width=1.8] (15.2376,2.17635) -- (15.3377,2.11756);
\definecolor{c}{rgb}{0,0,0};
\draw [c,line width=0.9] (2,1.46047) -- (18,1.46047);
\draw [anchor= east] (18,0.759447) node[scale=0.965639, color=c, rotate=0]{Canali};
\draw [c,line width=0.9] (2.10884,1.81099) -- (2.10884,1.46047);
\draw [c,line width=0.9] (2.65306,1.63573) -- (2.65306,1.46047);
\draw [c,line width=0.9] (3.19728,1.63573) -- (3.19728,1.46047);
\draw [c,line width=0.9] (3.7415,1.63573) -- (3.7415,1.46047);
\draw [c,line width=0.9] (4.28571,1.63573) -- (4.28571,1.46047);
\draw [c,line width=0.9] (4.82993,1.81099) -- (4.82993,1.46047);
\draw [c,line width=0.9] (5.37415,1.63573) -- (5.37415,1.46047);
\draw [c,line width=0.9] (5.91837,1.63573) -- (5.91837,1.46047);
\draw [c,line width=0.9] (6.46258,1.63573) -- (6.46258,1.46047);
\draw [c,line width=0.9] (7.0068,1.63573) -- (7.0068,1.46047);
\draw [c,line width=0.9] (7.55102,1.81099) -- (7.55102,1.46047);
\draw [c,line width=0.9] (8.09524,1.63573) -- (8.09524,1.46047);
\draw [c,line width=0.9] (8.63946,1.63573) -- (8.63946,1.46047);
\draw [c,line width=0.9] (9.18367,1.63573) -- (9.18367,1.46047);
\draw [c,line width=0.9] (9.72789,1.63573) -- (9.72789,1.46047);
\draw [c,line width=0.9] (10.2721,1.81099) -- (10.2721,1.46047);
\draw [c,line width=0.9] (10.8163,1.63573) -- (10.8163,1.46047);
\draw [c,line width=0.9] (11.3605,1.63573) -- (11.3605,1.46047);
\draw [c,line width=0.9] (11.9048,1.63573) -- (11.9048,1.46047);
\draw [c,line width=0.9] (12.449,1.63573) -- (12.449,1.46047);
\draw [c,line width=0.9] (12.9932,1.81099) -- (12.9932,1.46047);
\draw [c,line width=0.9] (13.5374,1.63573) -- (13.5374,1.46047);
\draw [c,line width=0.9] (14.0816,1.63573) -- (14.0816,1.46047);
\draw [c,line width=0.9] (14.6259,1.63573) -- (14.6259,1.46047);
\draw [c,line width=0.9] (15.1701,1.63573) -- (15.1701,1.46047);
\draw [c,line width=0.9] (15.7143,1.81099) -- (15.7143,1.46047);
\draw [c,line width=0.9] (2.10884,1.81099) -- (2.10884,1.46047);
\draw [c,line width=0.9] (15.7143,1.81099) -- (15.7143,1.46047);
\draw [c,line width=0.9] (16.2585,1.63573) -- (16.2585,1.46047);
\draw [c,line width=0.9] (16.8027,1.63573) -- (16.8027,1.46047);
\draw [c,line width=0.9] (17.3469,1.63573) -- (17.3469,1.46047);
\draw [c,line width=0.9] (17.8912,1.63573) -- (17.8912,1.46047);
\draw [anchor=base] (2.10884,1.03694) node[scale=0.965639, color=c, rotate=0]{450};
\draw [anchor=base] (4.82993,1.03694) node[scale=0.965639, color=c, rotate=0]{500};
\draw [anchor=base] (7.55102,1.03694) node[scale=0.965639, color=c, rotate=0]{550};
\draw [anchor=base] (10.2721,1.03694) node[scale=0.965639, color=c, rotate=0]{600};
\draw [anchor=base] (12.9932,1.03694) node[scale=0.965639, color=c, rotate=0]{650};
\draw [anchor=base] (15.7143,1.03694) node[scale=0.965639, color=c, rotate=0]{700};
\draw [c,line width=0.9] (2,1.46047) -- (2,13.1443);
\draw [anchor= east] (0.4,13.1443) node[scale=1.62403, color=c, rotate=90]{Eventi};
\draw [c,line width=0.9] (2.48,1.46047) -- (2,1.46047);
\draw [c,line width=0.9] (2.24,1.75474) -- (2,1.75474);
\draw [c,line width=0.9] (2.24,2.04901) -- (2,2.04901);
\draw [c,line width=0.9] (2.24,2.34328) -- (2,2.34328);
\draw [c,line width=0.9] (2.48,2.63754) -- (2,2.63754);
\draw [c,line width=0.9] (2.24,2.93181) -- (2,2.93181);
\draw [c,line width=0.9] (2.24,3.22608) -- (2,3.22608);
\draw [c,line width=0.9] (2.24,3.52035) -- (2,3.52035);
\draw [c,line width=0.9] (2.48,3.81461) -- (2,3.81461);
\draw [c,line width=0.9] (2.24,4.10888) -- (2,4.10888);
\draw [c,line width=0.9] (2.24,4.40315) -- (2,4.40315);
\draw [c,line width=0.9] (2.24,4.69741) -- (2,4.69741);
\draw [c,line width=0.9] (2.48,4.99168) -- (2,4.99168);
\draw [c,line width=0.9] (2.24,5.28595) -- (2,5.28595);
\draw [c,line width=0.9] (2.24,5.58022) -- (2,5.58022);
\draw [c,line width=0.9] (2.24,5.87448) -- (2,5.87448);
\draw [c,line width=0.9] (2.48,6.16875) -- (2,6.16875);
\draw [c,line width=0.9] (2.24,6.46302) -- (2,6.46302);
\draw [c,line width=0.9] (2.24,6.75729) -- (2,6.75729);
\draw [c,line width=0.9] (2.24,7.05155) -- (2,7.05155);
\draw [c,line width=0.9] (2.48,7.34582) -- (2,7.34582);
\draw [c,line width=0.9] (2.24,7.64009) -- (2,7.64009);
\draw [c,line width=0.9] (2.24,7.93435) -- (2,7.93435);
\draw [c,line width=0.9] (2.24,8.22862) -- (2,8.22862);
\draw [c,line width=0.9] (2.48,8.52289) -- (2,8.52289);
\draw [c,line width=0.9] (2.24,8.81716) -- (2,8.81716);
\draw [c,line width=0.9] (2.24,9.11142) -- (2,9.11142);
\draw [c,line width=0.9] (2.24,9.40569) -- (2,9.40569);
\draw [c,line width=0.9] (2.48,9.69996) -- (2,9.69996);
\draw [c,line width=0.9] (2.24,9.99423) -- (2,9.99423);
\draw [c,line width=0.9] (2.24,10.2885) -- (2,10.2885);
\draw [c,line width=0.9] (2.24,10.5828) -- (2,10.5828);
\draw [c,line width=0.9] (2.48,10.877) -- (2,10.877);
\draw [c,line width=0.9] (2.24,11.1713) -- (2,11.1713);
\draw [c,line width=0.9] (2.24,11.4656) -- (2,11.4656);
\draw [c,line width=0.9] (2.24,11.7598) -- (2,11.7598);
\draw [c,line width=0.9] (2.48,12.0541) -- (2,12.0541);
\draw [c,line width=0.9] (2.48,12.0541) -- (2,12.0541);
\draw [c,line width=0.9] (2.24,12.3484) -- (2,12.3484);
\draw [c,line width=0.9] (2.24,12.6426) -- (2,12.6426);
\draw [c,line width=0.9] (2.24,12.9369) -- (2,12.9369);
\draw [anchor= east] (1.9,1.46047) node[scale=0.965639, color=c, rotate=0]{0};
\draw [anchor= east] (1.9,2.63754) node[scale=0.965639, color=c, rotate=0]{2000};
\draw [anchor= east] (1.9,3.81461) node[scale=0.965639, color=c, rotate=0]{4000};
\draw [anchor= east] (1.9,4.99168) node[scale=0.965639, color=c, rotate=0]{6000};
\draw [anchor= east] (1.9,6.16875) node[scale=0.965639, color=c, rotate=0]{8000};
\draw [anchor= east] (1.9,7.34582) node[scale=0.965639, color=c, rotate=0]{10000};
\draw [anchor= east] (1.9,8.52289) node[scale=0.965639, color=c, rotate=0]{12000};
\draw [anchor= east] (1.9,9.69996) node[scale=0.965639, color=c, rotate=0]{14000};
\draw [anchor= east] (1.9,10.877) node[scale=0.965639, color=c, rotate=0]{16000};
\draw [anchor= east] (1.9,12.0541) node[scale=0.965639, color=c, rotate=0]{18000};
\definecolor{c}{rgb}{1,1,1};
\draw [color=c, fill=c] (13.3004,8.12253) rectangle (18.6166,13.3992);
\definecolor{c}{rgb}{0,0,0};
\draw [c,line width=0.9] (13.3004,8.12253) -- (18.6166,8.12253);
\draw [c,line width=0.9] (18.6166,8.12253) -- (18.6166,13.3992);
\draw [c,line width=0.9] (18.6166,13.3992) -- (13.3004,13.3992);
\draw [c,line width=0.9] (13.3004,13.3992) -- (13.3004,8.12253);
\draw [anchor= west] (13.5662,12.7396) node[scale=0.790068, color=c, rotate=0]{Entries };
\draw [anchor= east] (18.3508,12.7396) node[scale=0.790068, color=c, rotate=0]{ 1763855};
\draw [anchor= west] (13.5662,11.4205) node[scale=0.790068, color=c, rotate=0]{Constant };
\draw [anchor= east] (18.3508,11.4205) node[scale=0.790068, color=c, rotate=0]{$ 1.879e+04 \pm 2.335e+01$};
\draw [anchor= west] (13.5662,10.1013) node[scale=0.790068, color=c, rotate=0]{Mean     };
\draw [anchor= east] (18.3508,10.1013) node[scale=0.790068, color=c, rotate=0]{$ 572.8 \pm 0.1$};
\draw [anchor= west] (13.5662,8.78211) node[scale=0.790068, color=c, rotate=0]{Sigma    };
\draw [anchor= east] (18.3508,8.78211) node[scale=0.790068, color=c, rotate=0]{$ 50.63 \pm 0.07$};
\draw (8.83636,13.9822) node[scale=1.53624, color=c, rotate=0]{Americio R2 Am Picco 59.5 keV};
\end{tikzpicture}

\begin{tikzpicture}
\pgfdeclareplotmark{cross} {
\pgfpathmoveto{\pgfpoint{-0.3\pgfplotmarksize}{\pgfplotmarksize}}
\pgfpathlineto{\pgfpoint{+0.3\pgfplotmarksize}{\pgfplotmarksize}}
\pgfpathlineto{\pgfpoint{+0.3\pgfplotmarksize}{0.3\pgfplotmarksize}}
\pgfpathlineto{\pgfpoint{+1\pgfplotmarksize}{0.3\pgfplotmarksize}}
\pgfpathlineto{\pgfpoint{+1\pgfplotmarksize}{-0.3\pgfplotmarksize}}
\pgfpathlineto{\pgfpoint{+0.3\pgfplotmarksize}{-0.3\pgfplotmarksize}}
\pgfpathlineto{\pgfpoint{+0.3\pgfplotmarksize}{-1.\pgfplotmarksize}}
\pgfpathlineto{\pgfpoint{-0.3\pgfplotmarksize}{-1.\pgfplotmarksize}}
\pgfpathlineto{\pgfpoint{-0.3\pgfplotmarksize}{-0.3\pgfplotmarksize}}
\pgfpathlineto{\pgfpoint{-1.\pgfplotmarksize}{-0.3\pgfplotmarksize}}
\pgfpathlineto{\pgfpoint{-1.\pgfplotmarksize}{0.3\pgfplotmarksize}}
\pgfpathlineto{\pgfpoint{-0.3\pgfplotmarksize}{0.3\pgfplotmarksize}}
\pgfpathclose
\pgfusepathqstroke
}
\pgfdeclareplotmark{cross*} {
\pgfpathmoveto{\pgfpoint{-0.3\pgfplotmarksize}{\pgfplotmarksize}}
\pgfpathlineto{\pgfpoint{+0.3\pgfplotmarksize}{\pgfplotmarksize}}
\pgfpathlineto{\pgfpoint{+0.3\pgfplotmarksize}{0.3\pgfplotmarksize}}
\pgfpathlineto{\pgfpoint{+1\pgfplotmarksize}{0.3\pgfplotmarksize}}
\pgfpathlineto{\pgfpoint{+1\pgfplotmarksize}{-0.3\pgfplotmarksize}}
\pgfpathlineto{\pgfpoint{+0.3\pgfplotmarksize}{-0.3\pgfplotmarksize}}
\pgfpathlineto{\pgfpoint{+0.3\pgfplotmarksize}{-1.\pgfplotmarksize}}
\pgfpathlineto{\pgfpoint{-0.3\pgfplotmarksize}{-1.\pgfplotmarksize}}
\pgfpathlineto{\pgfpoint{-0.3\pgfplotmarksize}{-0.3\pgfplotmarksize}}
\pgfpathlineto{\pgfpoint{-1.\pgfplotmarksize}{-0.3\pgfplotmarksize}}
\pgfpathlineto{\pgfpoint{-1.\pgfplotmarksize}{0.3\pgfplotmarksize}}
\pgfpathlineto{\pgfpoint{-0.3\pgfplotmarksize}{0.3\pgfplotmarksize}}
\pgfpathclose
\pgfusepathqfillstroke
}
\pgfdeclareplotmark{newstar} {
\pgfpathmoveto{\pgfqpoint{0pt}{\pgfplotmarksize}}
\pgfpathlineto{\pgfqpointpolar{44}{0.5\pgfplotmarksize}}
\pgfpathlineto{\pgfqpointpolar{18}{\pgfplotmarksize}}
\pgfpathlineto{\pgfqpointpolar{-20}{0.5\pgfplotmarksize}}
\pgfpathlineto{\pgfqpointpolar{-54}{\pgfplotmarksize}}
\pgfpathlineto{\pgfqpointpolar{-90}{0.5\pgfplotmarksize}}
\pgfpathlineto{\pgfqpointpolar{234}{\pgfplotmarksize}}
\pgfpathlineto{\pgfqpointpolar{198}{0.5\pgfplotmarksize}}
\pgfpathlineto{\pgfqpointpolar{162}{\pgfplotmarksize}}
\pgfpathlineto{\pgfqpointpolar{134}{0.5\pgfplotmarksize}}
\pgfpathclose
\pgfusepathqstroke
}
\pgfdeclareplotmark{newstar*} {
\pgfpathmoveto{\pgfqpoint{0pt}{\pgfplotmarksize}}
\pgfpathlineto{\pgfqpointpolar{44}{0.5\pgfplotmarksize}}
\pgfpathlineto{\pgfqpointpolar{18}{\pgfplotmarksize}}
\pgfpathlineto{\pgfqpointpolar{-20}{0.5\pgfplotmarksize}}
\pgfpathlineto{\pgfqpointpolar{-54}{\pgfplotmarksize}}
\pgfpathlineto{\pgfqpointpolar{-90}{0.5\pgfplotmarksize}}
\pgfpathlineto{\pgfqpointpolar{234}{\pgfplotmarksize}}
\pgfpathlineto{\pgfqpointpolar{198}{0.5\pgfplotmarksize}}
\pgfpathlineto{\pgfqpointpolar{162}{\pgfplotmarksize}}
\pgfpathlineto{\pgfqpointpolar{134}{0.5\pgfplotmarksize}}
\pgfpathclose
\pgfusepathqfillstroke
}
\definecolor{c}{rgb}{1,1,1};
\draw [color=c, fill=c] (0,0) rectangle (20,13.2934);
\draw [color=c, fill=c] (2,1.32934) rectangle (18,11.9641);
\definecolor{c}{rgb}{0,0,0};
\draw [c,line width=0.9] (2,1.32934) -- (2,11.9641) -- (18,11.9641) -- (18,1.32934) -- (2,1.32934);
\definecolor{c}{rgb}{1,1,1};
\draw [color=c, fill=c] (2,1.32934) rectangle (18,11.9641);
\definecolor{c}{rgb}{0,0,0};
\draw [c,line width=0.9] (2,1.32934) -- (2,11.9641) -- (18,11.9641) -- (18,1.32934) -- (2,1.32934);
\definecolor{c}{rgb}{0,0,0.6};
\draw [c,line width=0.9] (2,2.62624) -- (2.13333,2.62624) -- (2.13333,2.6452) -- (2.26667,2.6452) -- (2.26667,2.71023) -- (2.4,2.71023) -- (2.4,2.75267) -- (2.53333,2.75267) -- (2.53333,2.95497) -- (2.66667,2.95497) -- (2.66667,2.97303) --
 (2.8,2.97303) -- (2.8,3.01909) -- (2.93333,3.01909) -- (2.93333,3.16629) -- (3.06667,3.16629) -- (3.06667,3.30176) -- (3.2,3.30176) -- (3.2,3.41826) -- (3.33333,3.41826) -- (3.33333,3.43993) -- (3.46667,3.43993) -- (3.46667,3.55643) -- (3.6,3.55643)
 -- (3.6,3.76234) -- (3.73333,3.76234) -- (3.73333,3.88245) -- (3.86667,3.88245) -- (3.86667,3.99805) -- (4,3.99805) -- (4,4.20034) -- (4.13333,4.20034) -- (4.13333,4.31775) -- (4.26667,4.31775) -- (4.26667,4.58235) -- (4.4,4.58235) -- (4.4,4.74581)
 -- (4.53333,4.74581) -- (4.53333,4.95172) -- (4.66667,4.95172) -- (4.66667,5.10525) -- (4.8,5.10525) -- (4.8,5.42495) -- (4.93333,5.42495) -- (4.93333,5.48274) -- (5.06667,5.48274) -- (5.06667,5.69768) -- (5.2,5.69768) -- (5.2,5.91172) --
 (5.33333,5.91172) -- (5.33333,6.24586) -- (5.46667,6.24586) -- (5.46667,6.32263) -- (5.6,6.32263) -- (5.6,6.66039) -- (5.73333,6.66039) -- (5.73333,6.93764) -- (5.86667,6.93764) -- (5.86667,7.10742) -- (6,7.10742) -- (6,7.44879) -- (6.13333,7.44879)
 -- (6.13333,7.69534) -- (6.26667,7.69534) -- (6.26667,7.74591) -- (6.4,7.74591) -- (6.4,8.10264) -- (6.53333,8.10264) -- (6.53333,8.31306) -- (6.66667,8.31306) -- (6.66667,8.48646) -- (6.8,8.48646) -- (6.8,8.75106) -- (6.93333,8.75106) --
 (6.93333,9.06263) -- (7.06667,9.06263) -- (7.06667,9.00754) -- (7.2,9.00754) -- (7.2,9.4013) -- (7.33333,9.4013) -- (7.33333,9.60991) -- (7.46667,9.60991) -- (7.46667,9.75441) -- (7.6,9.75441) -- (7.6,10.0723) -- (7.73333,10.0723) --
 (7.73333,10.2078) -- (7.86667,10.2078) -- (7.86667,10.2809) -- (8,10.2809) -- (8,10.4552) -- (8.13333,10.4552) -- (8.13333,10.5826) -- (8.26667,10.5826) -- (8.26667,10.6593) -- (8.4,10.6593) -- (8.4,10.7207) -- (8.53333,10.7207) -- (8.53333,11.0242)
 -- (8.66667,11.0242) -- (8.66667,11.1885) -- (8.8,11.1885) -- (8.8,11.0919) -- (8.93333,11.0919) -- (8.93333,11.2328) -- (9.06667,11.2328) -- (9.06667,11.3186) -- (9.2,11.3186) -- (9.2,11.296) -- (9.33333,11.296) -- (9.33333,11.1641) --
 (9.46667,11.1641) -- (9.46667,11.2933) -- (9.6,11.2933) -- (9.6,11.4577) -- (9.73333,11.4577) -- (9.73333,11.3764) -- (9.86667,11.3764) -- (9.86667,11.128) -- (10,11.128) -- (10,11.2608) -- (10.1333,11.2608) -- (10.1333,11.0088) -- (10.2667,11.0088)
 -- (10.2667,10.9375) -- (10.4,10.9375) -- (10.4,10.877) -- (10.5333,10.877) -- (10.5333,10.5925) -- (10.6667,10.5925) -- (10.6667,10.5573) -- (10.8,10.5573) -- (10.8,10.522) -- (10.9333,10.522) -- (10.9333,10.3125) -- (11.0667,10.3125) --
 (11.0667,10.1635) -- (11.2,10.1635) -- (11.2,10.0398) -- (11.3333,10.0398) -- (11.3333,9.77518) -- (11.4667,9.77518) -- (11.4667,9.55121) -- (11.6,9.55121) -- (11.6,9.42568) -- (11.7333,9.42568) -- (11.7333,9.3146) -- (11.8667,9.3146) --
 (11.8667,8.95426) -- (12,8.95426) -- (12,8.79532) -- (12.1333,8.79532) -- (12.1333,8.43498) -- (12.2667,8.43498) -- (12.2667,8.31396) -- (12.4,8.31396) -- (12.4,8.16856) -- (12.5333,8.16856) -- (12.5333,7.92202) -- (12.6667,7.92202) --
 (12.6667,7.43525) -- (12.8,7.43525) -- (12.8,7.36752) -- (12.9333,7.36752) -- (12.9333,7.23837) -- (13.0667,7.23837) -- (13.0667,6.66852) -- (13.2,6.66852) -- (13.2,6.66581) -- (13.3333,6.66581) -- (13.3333,6.47435) -- (13.4667,6.47435) --
 (13.4667,6.15375) -- (13.6,6.15375) -- (13.6,5.90359) -- (13.7333,5.90359) -- (13.7333,5.81599) -- (13.8667,5.81599) -- (13.8667,5.48365) -- (14,5.48365) -- (14,5.34547) -- (14.1333,5.34547) -- (14.1333,5.09531) -- (14.2667,5.09531) --
 (14.2667,4.83974) -- (14.4,4.83974) -- (14.4,4.70156) -- (14.5333,4.70156) -- (14.5333,4.63022) -- (14.6667,4.63022) -- (14.6667,4.33852) -- (14.8,4.33852) -- (14.8,4.20486) -- (14.9333,4.20486) -- (14.9333,4.02695) -- (15.0667,4.02695) --
 (15.0667,3.88606) -- (15.2,3.88606) -- (15.2,3.78853) -- (15.3333,3.78853) -- (15.3333,3.52031) -- (15.4667,3.52031) -- (15.4667,3.40832) -- (15.6,3.40832) -- (15.6,3.20964) -- (15.7333,3.20964) -- (15.7333,3.06153) -- (15.8667,3.06153) --
 (15.8667,2.98928) -- (16,2.98928) -- (16,2.93058) -- (16.1333,2.93058) -- (16.1333,2.76893) -- (16.2667,2.76893) -- (16.2667,2.68313) -- (16.4,2.68313) -- (16.4,2.56212) -- (16.5333,2.56212) -- (16.5333,2.47) -- (16.6667,2.47) -- (16.6667,2.39775)
 -- (16.8,2.39775) -- (16.8,2.27222) -- (16.9333,2.27222) -- (16.9333,2.26319) -- (17.0667,2.26319) -- (17.0667,2.13495) -- (17.2,2.13495) -- (17.2,2.11418) -- (17.3333,2.11418) -- (17.3333,2.02839) -- (17.4667,2.02839) -- (17.4667,1.95614) --
 (17.6,1.95614) -- (17.6,1.88389) -- (17.7333,1.88389) -- (17.7333,1.91008) -- (17.8667,1.91008) -- (17.8667,1.81164) -- (18,1.81164);
\definecolor{c}{rgb}{1,1,1};
\draw [color=c, fill=c] (14.012,8.21215) rectangle (19.521,12.7288);
\definecolor{c}{rgb}{0,0,0};
\draw [c,line width=0.9] (14.012,8.21215) -- (19.521,8.21215);
\draw [c,line width=0.9] (19.521,8.21215) -- (19.521,12.7288);
\draw [c,line width=0.9] (19.521,12.7288) -- (14.012,12.7288);
\draw [c,line width=0.9] (14.012,12.7288) -- (14.012,8.21215);
\draw [anchor= west] (14.2874,12.1642) node[scale=0.797953, color=c, rotate=0]{Entries };
\draw [anchor= east] (19.2455,12.1642) node[scale=0.797953, color=c, rotate=0]{ 4967080};
\draw [anchor= west] (14.2874,11.0351) node[scale=0.797953, color=c, rotate=0]{Constant };
\draw [anchor= east] (19.2455,11.0351) node[scale=0.797953, color=c, rotate=0]{$ 1.143e+04 \pm 1.885e+01$};
\draw [anchor= west] (14.2874,9.9059) node[scale=0.797953, color=c, rotate=0]{Mean     };
\draw [anchor= east] (19.2455,9.9059) node[scale=0.797953, color=c, rotate=0]{$ 1.082e+04 \pm 0.4$};
\draw [anchor= west] (14.2874,8.77673) node[scale=0.797953, color=c, rotate=0]{Sigma    };
\draw [anchor= east] (19.2455,8.77673) node[scale=0.797953, color=c, rotate=0]{$ 265.7 \pm 0.5$};
\definecolor{c}{rgb}{1,0,0};
\draw [c,line width=1.8] (3.79133,3.76283) -- (3.90733,3.91526) -- (4.02333,4.07242) -- (4.13933,4.23422) -- (4.25533,4.40052) -- (4.37133,4.57119) -- (4.48733,4.74607) -- (4.60333,4.92497) -- (4.71933,5.1077) -- (4.83533,5.29401) --
 (4.95133,5.48367) -- (5.06733,5.67641) -- (5.18333,5.87192) -- (5.29933,6.0699) -- (5.41533,6.27001) -- (5.53133,6.4719) -- (5.64733,6.67519) -- (5.76333,6.8795) -- (5.87933,7.0844) -- (5.99533,7.28947) -- (6.11133,7.49428) -- (6.22733,7.69837) --
 (6.34333,7.90127) -- (6.45933,8.1025) -- (6.57533,8.30159) -- (6.69133,8.49803) -- (6.80733,8.69133) -- (6.92333,8.88099) -- (7.03933,9.06651) -- (7.15533,9.24739) -- (7.27133,9.42314) -- (7.38733,9.59326) -- (7.50333,9.75729) -- (7.61933,9.91474)
 -- (7.73533,10.0652) -- (7.85133,10.2081) -- (7.96733,10.3432) -- (8.08333,10.4699) -- (8.19933,10.588) -- (8.31533,10.697) -- (8.43133,10.7966) -- (8.54733,10.8865) -- (8.66333,10.9664) -- (8.77933,11.0361) -- (8.89533,11.0954) -- (9.01133,11.144)
 -- (9.12733,11.1818) -- (9.24333,11.2087) -- (9.35933,11.2246) -- (9.47533,11.2294);
\draw [c,line width=1.8] (9.47533,11.2294) -- (9.59133,11.2232) -- (9.70733,11.2059) -- (9.82333,11.1776) -- (9.93933,11.1384) -- (10.0553,11.0884) -- (10.1713,11.0278) -- (10.2873,10.9568) -- (10.4033,10.8756) -- (10.5193,10.7845) --
 (10.6353,10.6836) -- (10.7513,10.5735) -- (10.8673,10.4543) -- (10.9833,10.3265) -- (11.0993,10.1904) -- (11.2153,10.0465) -- (11.3313,9.89512) -- (11.4473,9.73681) -- (11.5633,9.57198) -- (11.6793,9.40112) -- (11.7953,9.22469) -- (11.9113,9.04319)
 -- (12.0273,8.85712) -- (12.1433,8.66697) -- (12.2593,8.47324) -- (12.3753,8.27644) -- (12.4913,8.07705) -- (12.6073,7.87558) -- (12.7233,7.67251) -- (12.8393,7.4683) -- (12.9553,7.26343) -- (13.0713,7.05835) -- (13.1873,6.8535) -- (13.3033,6.64931)
 -- (13.4193,6.44617) -- (13.5353,6.24449) -- (13.6513,6.04463) -- (13.7673,5.84695) -- (13.8833,5.65177) -- (13.9993,5.45941) -- (14.1153,5.27016) -- (14.2313,5.08429) -- (14.3473,4.90203) -- (14.4633,4.72363) -- (14.5793,4.54928) --
 (14.6953,4.37915) -- (14.8113,4.21341) -- (14.9273,4.0522) -- (15.0433,3.89564) -- (15.1593,3.74381);
\draw [c,line width=1.8] (15.1593,3.74381) -- (15.2753,3.5968);
\definecolor{c}{rgb}{0,0,0};
\draw [c,line width=0.9] (2,1.32934) -- (18,1.32934);
\draw [anchor= east] (18,0.58491) node[scale=1.02594, color=c, rotate=0]{Canali};
\draw [c,line width=0.9] (3.86667,1.64838) -- (3.86667,1.32934);
\draw [c,line width=0.9] (4.53333,1.48886) -- (4.53333,1.32934);
\draw [c,line width=0.9] (5.2,1.48886) -- (5.2,1.32934);
\draw [c,line width=0.9] (5.86667,1.48886) -- (5.86667,1.32934);
\draw [c,line width=0.9] (6.53333,1.64838) -- (6.53333,1.32934);
\draw [c,line width=0.9] (7.2,1.48886) -- (7.2,1.32934);
\draw [c,line width=0.9] (7.86667,1.48886) -- (7.86667,1.32934);
\draw [c,line width=0.9] (8.53333,1.48886) -- (8.53333,1.32934);
\draw [c,line width=0.9] (9.2,1.64838) -- (9.2,1.32934);
\draw [c,line width=0.9] (9.86667,1.48886) -- (9.86667,1.32934);
\draw [c,line width=0.9] (10.5333,1.48886) -- (10.5333,1.32934);
\draw [c,line width=0.9] (11.2,1.48886) -- (11.2,1.32934);
\draw [c,line width=0.9] (11.8667,1.64838) -- (11.8667,1.32934);
\draw [c,line width=0.9] (12.5333,1.48886) -- (12.5333,1.32934);
\draw [c,line width=0.9] (13.2,1.48886) -- (13.2,1.32934);
\draw [c,line width=0.9] (13.8667,1.48886) -- (13.8667,1.32934);
\draw [c,line width=0.9] (14.5333,1.64838) -- (14.5333,1.32934);
\draw [c,line width=0.9] (15.2,1.48886) -- (15.2,1.32934);
\draw [c,line width=0.9] (15.8667,1.48886) -- (15.8667,1.32934);
\draw [c,line width=0.9] (16.5333,1.48886) -- (16.5333,1.32934);
\draw [c,line width=0.9] (17.2,1.64838) -- (17.2,1.32934);
\draw [c,line width=0.9] (3.86667,1.64838) -- (3.86667,1.32934);
\draw [c,line width=0.9] (3.2,1.48886) -- (3.2,1.32934);
\draw [c,line width=0.9] (2.53333,1.48886) -- (2.53333,1.32934);
\draw [c,line width=0.9] (17.2,1.64838) -- (17.2,1.32934);
\draw [c,line width=0.9] (17.8667,1.48886) -- (17.8667,1.32934);
\draw [anchor=base] (3.86667,0.943833) node[scale=0.873949, color=c, rotate=0]{10400};
\draw [anchor=base] (6.53333,0.943833) node[scale=0.873949, color=c, rotate=0]{10600};
\draw [anchor=base] (9.2,0.943833) node[scale=0.873949, color=c, rotate=0]{10800};
\draw [anchor=base] (11.8667,0.943833) node[scale=0.873949, color=c, rotate=0]{11000};
\draw [anchor=base] (14.5333,0.943833) node[scale=0.873949, color=c, rotate=0]{11200};
\draw [anchor=base] (17.2,0.943833) node[scale=0.873949, color=c, rotate=0]{11400};
\draw [c,line width=0.9] (2,1.32934) -- (2,11.9641);
\draw [anchor= east] (0.4,11.9641) node[scale=1.48191, color=c, rotate=90]{Eventi};
\draw [c,line width=0.9] (2.48,2.71023) -- (2,2.71023);
\draw [c,line width=0.9] (2.24,3.16178) -- (2,3.16178);
\draw [c,line width=0.9] (2.24,3.61333) -- (2,3.61333);
\draw [c,line width=0.9] (2.24,4.06488) -- (2,4.06488);
\draw [c,line width=0.9] (2.48,4.51643) -- (2,4.51643);
\draw [c,line width=0.9] (2.24,4.96798) -- (2,4.96798);
\draw [c,line width=0.9] (2.24,5.41953) -- (2,5.41953);
\draw [c,line width=0.9] (2.24,5.87108) -- (2,5.87108);
\draw [c,line width=0.9] (2.48,6.32263) -- (2,6.32263);
\draw [c,line width=0.9] (2.24,6.77418) -- (2,6.77418);
\draw [c,line width=0.9] (2.24,7.22573) -- (2,7.22573);
\draw [c,line width=0.9] (2.24,7.67728) -- (2,7.67728);
\draw [c,line width=0.9] (2.48,8.12883) -- (2,8.12883);
\draw [c,line width=0.9] (2.24,8.58038) -- (2,8.58038);
\draw [c,line width=0.9] (2.24,9.03193) -- (2,9.03193);
\draw [c,line width=0.9] (2.24,9.48348) -- (2,9.48348);
\draw [c,line width=0.9] (2.48,9.93503) -- (2,9.93503);
\draw [c,line width=0.9] (2.24,10.3866) -- (2,10.3866);
\draw [c,line width=0.9] (2.24,10.8381) -- (2,10.8381);
\draw [c,line width=0.9] (2.24,11.2897) -- (2,11.2897);
\draw [c,line width=0.9] (2.48,11.7412) -- (2,11.7412);
\draw [c,line width=0.9] (2.48,2.71023) -- (2,2.71023);
\draw [c,line width=0.9] (2.24,2.25868) -- (2,2.25868);
\draw [c,line width=0.9] (2.24,1.80713) -- (2,1.80713);
\draw [c,line width=0.9] (2.24,1.35558) -- (2,1.35558);
\draw [c,line width=0.9] (2.48,11.7412) -- (2,11.7412);
\draw [anchor= east] (1.9,2.71023) node[scale=0.873949, color=c, rotate=0]{2000};
\draw [anchor= east] (1.9,4.51643) node[scale=0.873949, color=c, rotate=0]{4000};
\draw [anchor= east] (1.9,6.32263) node[scale=0.873949, color=c, rotate=0]{6000};
\draw [anchor= east] (1.9,8.12883) node[scale=0.873949, color=c, rotate=0]{8000};
\draw [anchor= east] (1.9,9.93503) node[scale=0.873949, color=c, rotate=0]{10000};
\draw [anchor= east] (1.9,11.7412) node[scale=0.873949, color=c, rotate=0]{12000};
\definecolor{c}{rgb}{1,1,1};
\draw [color=c, fill=c] (14.012,8.21215) rectangle (19.521,12.7288);
\definecolor{c}{rgb}{0,0,0};
\draw [c,line width=0.9] (14.012,8.21215) -- (19.521,8.21215);
\draw [c,line width=0.9] (19.521,8.21215) -- (19.521,12.7288);
\draw [c,line width=0.9] (19.521,12.7288) -- (14.012,12.7288);
\draw [c,line width=0.9] (14.012,12.7288) -- (14.012,8.21215);
\draw [anchor= west] (14.2874,12.1642) node[scale=0.797953, color=c, rotate=0]{Entries };
\draw [anchor= east] (19.2455,12.1642) node[scale=0.797953, color=c, rotate=0]{ 4967080};
\draw [anchor= west] (14.2874,11.0351) node[scale=0.797953, color=c, rotate=0]{Constant };
\draw [anchor= east] (19.2455,11.0351) node[scale=0.797953, color=c, rotate=0]{$ 1.143e+04 \pm 1.885e+01$};
\draw [anchor= west] (14.2874,9.9059) node[scale=0.797953, color=c, rotate=0]{Mean     };
\draw [anchor= east] (19.2455,9.9059) node[scale=0.797953, color=c, rotate=0]{$ 1.082e+04 \pm 0.4$};
\draw [anchor= west] (14.2874,8.77673) node[scale=0.797953, color=c, rotate=0]{Sigma    };
\draw [anchor= east] (19.2455,8.77673) node[scale=0.797953, color=c, rotate=0]{$ 265.7 \pm 0.5$};
\draw (7.91275,12.6518) node[scale=1.40592, color=c, rotate=0]{Cobalto R2 Picco 1173 keV};
\end{tikzpicture}

\begin{tikzpicture}
\pgfdeclareplotmark{cross} {
\pgfpathmoveto{\pgfpoint{-0.3\pgfplotmarksize}{\pgfplotmarksize}}
\pgfpathlineto{\pgfpoint{+0.3\pgfplotmarksize}{\pgfplotmarksize}}
\pgfpathlineto{\pgfpoint{+0.3\pgfplotmarksize}{0.3\pgfplotmarksize}}
\pgfpathlineto{\pgfpoint{+1\pgfplotmarksize}{0.3\pgfplotmarksize}}
\pgfpathlineto{\pgfpoint{+1\pgfplotmarksize}{-0.3\pgfplotmarksize}}
\pgfpathlineto{\pgfpoint{+0.3\pgfplotmarksize}{-0.3\pgfplotmarksize}}
\pgfpathlineto{\pgfpoint{+0.3\pgfplotmarksize}{-1.\pgfplotmarksize}}
\pgfpathlineto{\pgfpoint{-0.3\pgfplotmarksize}{-1.\pgfplotmarksize}}
\pgfpathlineto{\pgfpoint{-0.3\pgfplotmarksize}{-0.3\pgfplotmarksize}}
\pgfpathlineto{\pgfpoint{-1.\pgfplotmarksize}{-0.3\pgfplotmarksize}}
\pgfpathlineto{\pgfpoint{-1.\pgfplotmarksize}{0.3\pgfplotmarksize}}
\pgfpathlineto{\pgfpoint{-0.3\pgfplotmarksize}{0.3\pgfplotmarksize}}
\pgfpathclose
\pgfusepathqstroke
}
\pgfdeclareplotmark{cross*} {
\pgfpathmoveto{\pgfpoint{-0.3\pgfplotmarksize}{\pgfplotmarksize}}
\pgfpathlineto{\pgfpoint{+0.3\pgfplotmarksize}{\pgfplotmarksize}}
\pgfpathlineto{\pgfpoint{+0.3\pgfplotmarksize}{0.3\pgfplotmarksize}}
\pgfpathlineto{\pgfpoint{+1\pgfplotmarksize}{0.3\pgfplotmarksize}}
\pgfpathlineto{\pgfpoint{+1\pgfplotmarksize}{-0.3\pgfplotmarksize}}
\pgfpathlineto{\pgfpoint{+0.3\pgfplotmarksize}{-0.3\pgfplotmarksize}}
\pgfpathlineto{\pgfpoint{+0.3\pgfplotmarksize}{-1.\pgfplotmarksize}}
\pgfpathlineto{\pgfpoint{-0.3\pgfplotmarksize}{-1.\pgfplotmarksize}}
\pgfpathlineto{\pgfpoint{-0.3\pgfplotmarksize}{-0.3\pgfplotmarksize}}
\pgfpathlineto{\pgfpoint{-1.\pgfplotmarksize}{-0.3\pgfplotmarksize}}
\pgfpathlineto{\pgfpoint{-1.\pgfplotmarksize}{0.3\pgfplotmarksize}}
\pgfpathlineto{\pgfpoint{-0.3\pgfplotmarksize}{0.3\pgfplotmarksize}}
\pgfpathclose
\pgfusepathqfillstroke
}
\pgfdeclareplotmark{newstar} {
\pgfpathmoveto{\pgfqpoint{0pt}{\pgfplotmarksize}}
\pgfpathlineto{\pgfqpointpolar{44}{0.5\pgfplotmarksize}}
\pgfpathlineto{\pgfqpointpolar{18}{\pgfplotmarksize}}
\pgfpathlineto{\pgfqpointpolar{-20}{0.5\pgfplotmarksize}}
\pgfpathlineto{\pgfqpointpolar{-54}{\pgfplotmarksize}}
\pgfpathlineto{\pgfqpointpolar{-90}{0.5\pgfplotmarksize}}
\pgfpathlineto{\pgfqpointpolar{234}{\pgfplotmarksize}}
\pgfpathlineto{\pgfqpointpolar{198}{0.5\pgfplotmarksize}}
\pgfpathlineto{\pgfqpointpolar{162}{\pgfplotmarksize}}
\pgfpathlineto{\pgfqpointpolar{134}{0.5\pgfplotmarksize}}
\pgfpathclose
\pgfusepathqstroke
}
\pgfdeclareplotmark{newstar*} {
\pgfpathmoveto{\pgfqpoint{0pt}{\pgfplotmarksize}}
\pgfpathlineto{\pgfqpointpolar{44}{0.5\pgfplotmarksize}}
\pgfpathlineto{\pgfqpointpolar{18}{\pgfplotmarksize}}
\pgfpathlineto{\pgfqpointpolar{-20}{0.5\pgfplotmarksize}}
\pgfpathlineto{\pgfqpointpolar{-54}{\pgfplotmarksize}}
\pgfpathlineto{\pgfqpointpolar{-90}{0.5\pgfplotmarksize}}
\pgfpathlineto{\pgfqpointpolar{234}{\pgfplotmarksize}}
\pgfpathlineto{\pgfqpointpolar{198}{0.5\pgfplotmarksize}}
\pgfpathlineto{\pgfqpointpolar{162}{\pgfplotmarksize}}
\pgfpathlineto{\pgfqpointpolar{134}{0.5\pgfplotmarksize}}
\pgfpathclose
\pgfusepathqfillstroke
}
\definecolor{c}{rgb}{1,1,1};
\draw [color=c, fill=c] (0,0) rectangle (20,13.9372);
\draw [color=c, fill=c] (2,1.39372) rectangle (18,12.5435);
\definecolor{c}{rgb}{0,0,0};
\draw [c,line width=0.9] (2,1.39372) -- (2,12.5435) -- (18,12.5435) -- (18,1.39372) -- (2,1.39372);
\definecolor{c}{rgb}{1,1,1};
\draw [color=c, fill=c] (2,1.39372) rectangle (18,12.5435);
\definecolor{c}{rgb}{0,0,0};
\draw [c,line width=0.9] (2,1.39372) -- (2,12.5435) -- (18,12.5435) -- (18,1.39372) -- (2,1.39372);
\definecolor{c}{rgb}{0,0,0.6};
\draw [c,line width=0.9] (2,2.25006) -- (2.1203,2.25006) -- (2.1203,2.20821) -- (2.2406,2.20821) -- (2.2406,2.24441) -- (2.3609,2.24441) -- (2.3609,2.27947) -- (2.4812,2.27947) -- (2.4812,2.3349) -- (2.6015,2.3349) -- (2.6015,2.43106) --
 (2.7218,2.43106) -- (2.7218,2.56002) -- (2.84211,2.56002) -- (2.84211,2.59961) -- (2.96241,2.59961) -- (2.96241,2.67653) -- (3.08271,2.67653) -- (3.08271,2.73196) -- (3.20301,2.73196) -- (3.20301,2.88807) -- (3.32331,2.88807) -- (3.32331,2.94237) --
 (3.44361,2.94237) -- (3.44361,3.08943) -- (3.56391,3.08943) -- (3.56391,3.22405) -- (3.68421,3.22405) -- (3.68421,3.23649) -- (3.80451,3.23649) -- (3.80451,3.35866) -- (3.92481,3.35866) -- (3.92481,3.57699) -- (4.04511,3.57699) -- (4.04511,3.74668)
 -- (4.16541,3.74668) -- (4.16541,3.87337) -- (4.28571,3.87337) -- (4.28571,3.96613) -- (4.40602,3.96613) -- (4.40602,4.13808) -- (4.52632,4.13808) -- (4.52632,4.34057) -- (4.64662,4.34057) -- (4.64662,4.49442) -- (4.76692,4.49442) --
 (4.76692,4.72293) -- (4.88722,4.72293) -- (4.88722,4.90279) -- (5.00752,4.90279) -- (5.00752,5.1822) -- (5.12782,5.1822) -- (5.12782,5.31456) -- (5.24812,5.31456) -- (5.24812,5.46727) -- (5.36842,5.46727) -- (5.36842,5.7908) -- (5.48872,5.7908) --
 (5.48872,6.03176) -- (5.60902,6.03176) -- (5.60902,6.27836) -- (5.72932,6.27836) -- (5.72932,6.56796) -- (5.84962,6.56796) -- (5.84962,6.75461) -- (5.96992,6.75461) -- (5.96992,7.14601) -- (6.09023,7.14601) -- (6.09023,7.26932) -- (6.21053,7.26932)
 -- (6.21053,7.40733) -- (6.33083,7.40733) -- (6.33083,7.77724) -- (6.45113,7.77724) -- (6.45113,7.90846) -- (6.57143,7.90846) -- (6.57143,8.26367) -- (6.69173,8.26367) -- (6.69173,8.41638) -- (6.81203,8.41638) -- (6.81203,8.71955) --
 (6.93233,8.71955) -- (6.93233,8.89942) -- (7.05263,8.89942) -- (7.05263,9.3587) -- (7.17293,9.3587) -- (7.17293,9.35191) -- (7.29323,9.35191) -- (7.29323,9.71729) -- (7.41353,9.71729) -- (7.41353,9.8768) -- (7.53383,9.8768) -- (7.53383,10.0838) --
 (7.65414,10.0838) -- (7.65414,10.3248) -- (7.77444,10.3248) -- (7.77444,10.4345) -- (7.89474,10.4345) -- (7.89474,10.7546) -- (8.01504,10.7546) -- (8.01504,10.732) -- (8.13534,10.732) -- (8.13534,11.163) -- (8.25564,11.163) -- (8.25564,11.4843) --
 (8.37594,11.4843) -- (8.37594,11.0827) -- (8.49624,11.0827) -- (8.49624,11.206) -- (8.61654,11.206) -- (8.61654,11.6766) -- (8.73684,11.6766) -- (8.73684,11.5974) -- (8.85714,11.5974) -- (8.85714,11.7897) -- (8.97744,11.7897) -- (8.97744,11.7716) --
 (9.09774,11.7716) -- (9.09774,11.9503) -- (9.21805,11.9503) -- (9.21805,11.9311) -- (9.33835,11.9311) -- (9.33835,11.7942) -- (9.45865,11.7942) -- (9.45865,11.9322) -- (9.57895,11.9322) -- (9.57895,11.7309) -- (9.69925,11.7309) -- (9.69925,12.0126)
 -- (9.81955,12.0126) -- (9.81955,11.9775) -- (9.93985,11.9775) -- (9.93985,11.9402) -- (10.0602,11.9402) -- (10.0602,11.7252) -- (10.1805,11.7252) -- (10.1805,11.7524) -- (10.3008,11.7524) -- (10.3008,11.7309) -- (10.4211,11.7309) --
 (10.4211,11.732) -- (10.5414,11.732) -- (10.5414,11.611) -- (10.6617,11.611) -- (10.6617,11.37) -- (10.782,11.37) -- (10.782,11.2682) -- (10.9023,11.2682) -- (10.9023,11.0487) -- (11.0226,11.0487) -- (11.0226,10.8157) -- (11.1429,10.8157) --
 (11.1429,10.7003) -- (11.2632,10.7003) -- (11.2632,10.4073) -- (11.3835,10.4073) -- (11.3835,10.3214) -- (11.5038,10.3214) -- (11.5038,10.2037) -- (11.6241,10.2037) -- (11.6241,9.9096) -- (11.7444,9.9096) -- (11.7444,9.76594) -- (11.8647,9.76594) --
 (11.8647,9.61435) -- (11.985,9.61435) -- (11.985,9.25236) -- (12.1053,9.25236) -- (12.1053,9.04648) -- (12.2256,9.04648) -- (12.2256,8.87227) -- (12.3459,8.87227) -- (12.3459,8.55213) -- (12.4662,8.55213) -- (12.4662,8.48086) -- (12.5865,8.48086) --
 (12.5865,8.23539) -- (12.7068,8.23539) -- (12.7068,7.96389) -- (12.8271,7.96389) -- (12.8271,7.7286) -- (12.9474,7.7286) -- (12.9474,7.44692) -- (13.0677,7.44692) -- (13.0677,7.46389) -- (13.188,7.46389) -- (13.188,7.20144) -- (13.3083,7.20144) --
 (13.3083,6.69465) -- (13.4286,6.69465) -- (13.4286,6.68334) -- (13.5489,6.68334) -- (13.5489,6.34624) -- (13.6692,6.34624) -- (13.6692,6.13922) -- (13.7895,6.13922) -- (13.7895,5.96501) -- (13.9098,5.96501) -- (13.9098,5.84171) -- (14.0301,5.84171)
 -- (14.0301,5.60868) -- (14.1504,5.60868) -- (14.1504,5.44012) -- (14.2707,5.44012) -- (14.2707,5.11772) -- (14.391,5.11772) -- (14.391,4.9865) -- (14.5113,4.9865) -- (14.5113,4.72066) -- (14.6316,4.72066) -- (14.6316,4.57926) -- (14.7519,4.57926)
 -- (14.7519,4.49555) -- (14.8722,4.49555) -- (14.8722,4.24555) -- (14.9925,4.24555) -- (14.9925,4.15957) -- (15.1128,4.15957) -- (15.1128,3.95482) -- (15.2331,3.95482) -- (15.2331,3.78287) -- (15.3534,3.78287) -- (15.3534,3.65165) --
 (15.4737,3.65165) -- (15.4737,3.45708) -- (15.594,3.45708) -- (15.594,3.43785) -- (15.7143,3.43785) -- (15.7143,3.3417) -- (15.8346,3.3417) -- (15.8346,3.17088) -- (15.9549,3.17088) -- (15.9549,3.0374) -- (16.0752,3.0374) -- (16.0752,2.96839) --
 (16.1955,2.96839) -- (16.1955,2.88355) -- (16.3158,2.88355) -- (16.3158,2.72291) -- (16.4361,2.72291) -- (16.4361,2.65391) -- (16.5564,2.65391) -- (16.5564,2.59961) -- (16.6767,2.59961) -- (16.6767,2.53852) -- (16.797,2.53852) -- (16.797,2.43332) --
 (16.9173,2.43332) -- (16.9173,2.38581) -- (17.0376,2.38581) -- (17.0376,2.23196) -- (17.1579,2.23196) -- (17.1579,2.21386) -- (17.2782,2.21386) -- (17.2782,2.15051) -- (17.3985,2.15051) -- (17.3985,2.05549) -- (17.5188,2.05549) -- (17.5188,2.02947)
 -- (17.6391,2.02947) -- (17.6391,1.97744) -- (17.7594,1.97744) -- (17.7594,1.9435) -- (17.8797,1.9435) -- (17.8797,1.89938) -- (18,1.89938);
\definecolor{c}{rgb}{1,1,1};
\draw [color=c, fill=c] (14.012,8.60987) rectangle (19.521,13.3453);
\definecolor{c}{rgb}{0,0,0};
\draw [c,line width=0.9] (14.012,8.60987) -- (19.521,8.60987);
\draw [c,line width=0.9] (19.521,8.60987) -- (19.521,13.3453);
\draw [c,line width=0.9] (19.521,13.3453) -- (14.012,13.3453);
\draw [c,line width=0.9] (14.012,13.3453) -- (14.012,8.60987);
\draw [anchor= west] (14.2874,12.7534) node[scale=0.995951, color=c, rotate=0]{Entries };
\draw [anchor= east] (19.2455,12.7534) node[scale=0.995951, color=c, rotate=0]{ 4967080};
\draw [anchor= west] (14.2874,11.5695) node[scale=0.995951, color=c, rotate=0]{Constant };
\draw [anchor= east] (19.2455,11.5695) node[scale=0.995951, color=c, rotate=0]{$  9500 \pm 16.3$};
\draw [anchor= west] (14.2874,10.3856) node[scale=0.995951, color=c, rotate=0]{Mean     };
\draw [anchor= east] (19.2455,10.3856) node[scale=0.995951, color=c, rotate=0]{$ 1.221e+04 \pm 0.4$};
\draw [anchor= west] (14.2874,9.20179) node[scale=0.995951, color=c, rotate=0]{Sigma    };
\draw [anchor= east] (19.2455,9.20179) node[scale=0.995951, color=c, rotate=0]{$ 272.5 \pm 0.5$};
\definecolor{c}{rgb}{1,0,0};
\draw [c,line width=1.8] (3.02196,2.56972) -- (3.14105,2.67437) -- (3.26015,2.78482) -- (3.37925,2.9012) -- (3.49835,3.02362) -- (3.61744,3.15218) -- (3.73654,3.28695) -- (3.85564,3.428) -- (3.97474,3.57537) -- (4.09383,3.72905) -- (4.21293,3.88906)
 -- (4.33203,4.05534) -- (4.45113,4.22783) -- (4.57023,4.40643) -- (4.68932,4.59103) -- (4.80842,4.78146) -- (4.92752,4.97752) -- (5.04662,5.17901) -- (5.16571,5.38566) -- (5.28481,5.59717) -- (5.40391,5.81323) -- (5.52301,6.03347) --
 (5.64211,6.25749) -- (5.7612,6.48487) -- (5.8803,6.71514) -- (5.9994,6.94781) -- (6.1185,7.18236) -- (6.23759,7.41821) -- (6.35669,7.65481) -- (6.47579,7.89153) -- (6.59489,8.12775) -- (6.71398,8.36281) -- (6.83308,8.59605) -- (6.95218,8.82678) --
 (7.07128,9.0543) -- (7.19038,9.27791) -- (7.30947,9.4969) -- (7.42857,9.71057) -- (7.54767,9.91819) -- (7.66677,10.1191) -- (7.78586,10.3125) -- (7.90496,10.4979) -- (8.02406,10.6745) -- (8.14316,10.8417) -- (8.26226,10.9988) -- (8.38135,11.1454) --
 (8.50045,11.2809) -- (8.61955,11.4047) -- (8.73865,11.5163) -- (8.85774,11.6155);
\draw [c,line width=1.8] (8.85774,11.6155) -- (8.97684,11.7017) -- (9.09594,11.7746) -- (9.21504,11.8339) -- (9.33414,11.8795) -- (9.45323,11.9111) -- (9.57233,11.9287) -- (9.69143,11.932) -- (9.81053,11.9212) -- (9.92962,11.8963) --
 (10.0487,11.8573) -- (10.1678,11.8044) -- (10.2869,11.7379) -- (10.406,11.658) -- (10.5251,11.565) -- (10.6442,11.4592) -- (10.7633,11.3411) -- (10.8824,11.2111) -- (11.0015,11.0697) -- (11.1206,10.9175) -- (11.2397,10.755) -- (11.3588,10.5828) --
 (11.4779,10.4015) -- (11.597,10.2118) -- (11.7161,10.0144) -- (11.8352,9.80986) -- (11.9543,9.59897) -- (12.0734,9.38242) -- (12.1925,9.16091) -- (12.3116,8.93515) -- (12.4307,8.70586) -- (12.5498,8.47372) -- (12.6689,8.23944) -- (12.788,8.00369) --
 (12.9071,7.76713) -- (13.0262,7.5304) -- (13.1453,7.29411) -- (13.2644,7.05888) -- (13.3835,6.82525) -- (13.5026,6.59378) -- (13.6217,6.36497) -- (13.7408,6.1393) -- (13.8598,5.91721) -- (13.9789,5.69912) -- (14.098,5.48541) -- (14.2171,5.27642) --
 (14.3362,5.07245) -- (14.4553,4.87378) -- (14.5744,4.68065) -- (14.6935,4.49327);
\draw [c,line width=1.8] (14.6935,4.49327) -- (14.8126,4.3118);
\definecolor{c}{rgb}{0,0,0};
\draw [c,line width=0.9] (2,1.39372) -- (18,1.39372);
\draw [anchor= east] (18,0.613238) node[scale=1.07563, color=c, rotate=0]{Canali};
\draw [c,line width=0.9] (2.3609,1.72822) -- (2.3609,1.39372);
\draw [c,line width=0.9] (2.96241,1.56097) -- (2.96241,1.39372);
\draw [c,line width=0.9] (3.56391,1.56097) -- (3.56391,1.39372);
\draw [c,line width=0.9] (4.16541,1.56097) -- (4.16541,1.39372);
\draw [c,line width=0.9] (4.76692,1.72822) -- (4.76692,1.39372);
\draw [c,line width=0.9] (5.36842,1.56097) -- (5.36842,1.39372);
\draw [c,line width=0.9] (5.96992,1.56097) -- (5.96992,1.39372);
\draw [c,line width=0.9] (6.57143,1.56097) -- (6.57143,1.39372);
\draw [c,line width=0.9] (7.17293,1.72822) -- (7.17293,1.39372);
\draw [c,line width=0.9] (7.77444,1.56097) -- (7.77444,1.39372);
\draw [c,line width=0.9] (8.37594,1.56097) -- (8.37594,1.39372);
\draw [c,line width=0.9] (8.97744,1.56097) -- (8.97744,1.39372);
\draw [c,line width=0.9] (9.57895,1.72822) -- (9.57895,1.39372);
\draw [c,line width=0.9] (10.1805,1.56097) -- (10.1805,1.39372);
\draw [c,line width=0.9] (10.782,1.56097) -- (10.782,1.39372);
\draw [c,line width=0.9] (11.3835,1.56097) -- (11.3835,1.39372);
\draw [c,line width=0.9] (11.985,1.72822) -- (11.985,1.39372);
\draw [c,line width=0.9] (12.5865,1.56097) -- (12.5865,1.39372);
\draw [c,line width=0.9] (13.188,1.56097) -- (13.188,1.39372);
\draw [c,line width=0.9] (13.7895,1.56097) -- (13.7895,1.39372);
\draw [c,line width=0.9] (14.391,1.72822) -- (14.391,1.39372);
\draw [c,line width=0.9] (14.9925,1.56097) -- (14.9925,1.39372);
\draw [c,line width=0.9] (15.594,1.56097) -- (15.594,1.39372);
\draw [c,line width=0.9] (16.1955,1.56097) -- (16.1955,1.39372);
\draw [c,line width=0.9] (16.797,1.72822) -- (16.797,1.39372);
\draw [c,line width=0.9] (2.3609,1.72822) -- (2.3609,1.39372);
\draw [c,line width=0.9] (16.797,1.72822) -- (16.797,1.39372);
\draw [c,line width=0.9] (17.3985,1.56097) -- (17.3985,1.39372);
\draw [c,line width=0.9] (18,1.56097) -- (18,1.39372);
\draw [anchor=base] (2.3609,0.989543) node[scale=0.916275, color=c, rotate=0]{11600};
\draw [anchor=base] (4.76692,0.989543) node[scale=0.916275, color=c, rotate=0]{11800};
\draw [anchor=base] (7.17293,0.989543) node[scale=0.916275, color=c, rotate=0]{12000};
\draw [anchor=base] (9.57895,0.989543) node[scale=0.916275, color=c, rotate=0]{12200};
\draw [anchor=base] (11.985,0.989543) node[scale=0.916275, color=c, rotate=0]{12400};
\draw [anchor=base] (14.391,0.989543) node[scale=0.916275, color=c, rotate=0]{12600};
\draw [anchor=base] (16.797,0.989543) node[scale=0.916275, color=c, rotate=0]{12800};
\draw [c,line width=0.9] (2,1.39372) -- (2,12.5435);
\draw [anchor= east] (0.4,12.5435) node[scale=1.55368, color=c, rotate=90]{Eventi};
\draw [c,line width=0.9] (2.48,2.3168) -- (2,2.3168);
\draw [c,line width=0.9] (2.24,2.54305) -- (2,2.54305);
\draw [c,line width=0.9] (2.24,2.76929) -- (2,2.76929);
\draw [c,line width=0.9] (2.24,2.99554) -- (2,2.99554);
\draw [c,line width=0.9] (2.24,3.22179) -- (2,3.22179);
\draw [c,line width=0.9] (2.48,3.44803) -- (2,3.44803);
\draw [c,line width=0.9] (2.24,3.67428) -- (2,3.67428);
\draw [c,line width=0.9] (2.24,3.90052) -- (2,3.90052);
\draw [c,line width=0.9] (2.24,4.12677) -- (2,4.12677);
\draw [c,line width=0.9] (2.24,4.35301) -- (2,4.35301);
\draw [c,line width=0.9] (2.48,4.57926) -- (2,4.57926);
\draw [c,line width=0.9] (2.24,4.8055) -- (2,4.8055);
\draw [c,line width=0.9] (2.24,5.03175) -- (2,5.03175);
\draw [c,line width=0.9] (2.24,5.258) -- (2,5.258);
\draw [c,line width=0.9] (2.24,5.48424) -- (2,5.48424);
\draw [c,line width=0.9] (2.48,5.71049) -- (2,5.71049);
\draw [c,line width=0.9] (2.24,5.93673) -- (2,5.93673);
\draw [c,line width=0.9] (2.24,6.16298) -- (2,6.16298);
\draw [c,line width=0.9] (2.24,6.38922) -- (2,6.38922);
\draw [c,line width=0.9] (2.24,6.61547) -- (2,6.61547);
\draw [c,line width=0.9] (2.48,6.84171) -- (2,6.84171);
\draw [c,line width=0.9] (2.24,7.06796) -- (2,7.06796);
\draw [c,line width=0.9] (2.24,7.29421) -- (2,7.29421);
\draw [c,line width=0.9] (2.24,7.52045) -- (2,7.52045);
\draw [c,line width=0.9] (2.24,7.7467) -- (2,7.7467);
\draw [c,line width=0.9] (2.48,7.97294) -- (2,7.97294);
\draw [c,line width=0.9] (2.24,8.19919) -- (2,8.19919);
\draw [c,line width=0.9] (2.24,8.42543) -- (2,8.42543);
\draw [c,line width=0.9] (2.24,8.65168) -- (2,8.65168);
\draw [c,line width=0.9] (2.24,8.87792) -- (2,8.87792);
\draw [c,line width=0.9] (2.48,9.10417) -- (2,9.10417);
\draw [c,line width=0.9] (2.24,9.33041) -- (2,9.33041);
\draw [c,line width=0.9] (2.24,9.55666) -- (2,9.55666);
\draw [c,line width=0.9] (2.24,9.78291) -- (2,9.78291);
\draw [c,line width=0.9] (2.24,10.0092) -- (2,10.0092);
\draw [c,line width=0.9] (2.48,10.2354) -- (2,10.2354);
\draw [c,line width=0.9] (2.24,10.4616) -- (2,10.4616);
\draw [c,line width=0.9] (2.24,10.6879) -- (2,10.6879);
\draw [c,line width=0.9] (2.24,10.9141) -- (2,10.9141);
\draw [c,line width=0.9] (2.24,11.1404) -- (2,11.1404);
\draw [c,line width=0.9] (2.48,11.3666) -- (2,11.3666);
\draw [c,line width=0.9] (2.24,11.5929) -- (2,11.5929);
\draw [c,line width=0.9] (2.24,11.8191) -- (2,11.8191);
\draw [c,line width=0.9] (2.24,12.0454) -- (2,12.0454);
\draw [c,line width=0.9] (2.24,12.2716) -- (2,12.2716);
\draw [c,line width=0.9] (2.48,12.4979) -- (2,12.4979);
\draw [c,line width=0.9] (2.48,2.3168) -- (2,2.3168);
\draw [c,line width=0.9] (2.24,2.09056) -- (2,2.09056);
\draw [c,line width=0.9] (2.24,1.86431) -- (2,1.86431);
\draw [c,line width=0.9] (2.24,1.63807) -- (2,1.63807);
\draw [c,line width=0.9] (2.24,1.41182) -- (2,1.41182);
\draw [c,line width=0.9] (2.48,12.4979) -- (2,12.4979);
\draw [anchor= east] (1.9,2.3168) node[scale=0.916275, color=c, rotate=0]{1000};
\draw [anchor= east] (1.9,3.44803) node[scale=0.916275, color=c, rotate=0]{2000};
\draw [anchor= east] (1.9,4.57926) node[scale=0.916275, color=c, rotate=0]{3000};
\draw [anchor= east] (1.9,5.71049) node[scale=0.916275, color=c, rotate=0]{4000};
\draw [anchor= east] (1.9,6.84171) node[scale=0.916275, color=c, rotate=0]{5000};
\draw [anchor= east] (1.9,7.97294) node[scale=0.916275, color=c, rotate=0]{6000};
\draw [anchor= east] (1.9,9.10417) node[scale=0.916275, color=c, rotate=0]{7000};
\draw [anchor= east] (1.9,10.2354) node[scale=0.916275, color=c, rotate=0]{8000};
\draw [anchor= east] (1.9,11.3666) node[scale=0.916275, color=c, rotate=0]{9000};
\draw [anchor= east] (1.9,12.4979) node[scale=0.916275, color=c, rotate=0]{10000};
\definecolor{c}{rgb}{1,1,1};
\draw [color=c, fill=c] (14.012,8.60987) rectangle (19.521,13.3453);
\definecolor{c}{rgb}{0,0,0};
\draw [c,line width=0.9] (14.012,8.60987) -- (19.521,8.60987);
\draw [c,line width=0.9] (19.521,8.60987) -- (19.521,13.3453);
\draw [c,line width=0.9] (19.521,13.3453) -- (14.012,13.3453);
\draw [c,line width=0.9] (14.012,13.3453) -- (14.012,8.60987);
\draw [anchor= west] (14.2874,12.7534) node[scale=0.995951, color=c, rotate=0]{Entries };
\draw [anchor= east] (19.2455,12.7534) node[scale=0.995951, color=c, rotate=0]{ 4967080};
\draw [anchor= west] (14.2874,11.5695) node[scale=0.995951, color=c, rotate=0]{Constant };
\draw [anchor= east] (19.2455,11.5695) node[scale=0.995951, color=c, rotate=0]{$  9500 \pm 16.3$};
\draw [anchor= west] (14.2874,10.3856) node[scale=0.995951, color=c, rotate=0]{Mean     };
\draw [anchor= east] (19.2455,10.3856) node[scale=0.995951, color=c, rotate=0]{$ 1.221e+04 \pm 0.4$};
\draw [anchor= west] (14.2874,9.20179) node[scale=0.995951, color=c, rotate=0]{Sigma    };
\draw [anchor= east] (19.2455,9.20179) node[scale=0.995951, color=c, rotate=0]{$ 272.5 \pm 0.5$};
\draw (8.11914,13.2646) node[scale=1.47401, color=c, rotate=0]{Cobalto R2 Picco 1333 keV};
\end{tikzpicture}


\FloatBarrier
\newpage

\input{sections/appendix/PLACEHOLDER.txt}
\FloatBarrier
\newpage
\end{appendix}
\clearpage

\begin{thebibliography}{9}

\end{thebibliography}


\input{./sections/code.tex}



\end{document}
