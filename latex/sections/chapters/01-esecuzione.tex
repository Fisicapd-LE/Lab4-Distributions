\section{Esecuzione esperimento}
L'apparato sperimentale consiste in una serie di moduli NIM (un generatore di alta tensione per alimentare i due PMT, un fan in/out, un CFTD, un TAC, una scatola di ritardi e una coincidence unit), due scintillatore di NaI(Tl) collegati ciascuno ad un PMT XP2020, un oscilloscopio e un digitizer CAEN DT5720.\\

Durante la prima giornata si sono analizzate le varie parti dell'apparato strumentale e si sono calibrati i sistemi di acquisizione. Per prima cosa si sono collegate le uscite dei due rivelatori al fan in/out e da lì all'oscilloscopio, e si è analizzata la forma (polarità, ampiezza media e tempi caratteristici) dei due segnali. Si è inoltre identificata l'ampiezza caratteristica dei segnali corrispondenti al fotone da 1333~keV.\\

Subito dopo si è passati all'analisi del segnale del CFTD. Si è perciò collegato un uscita del fan in/out (su ciascun segnale) all'entrate del CFTD e le uscite prompt e delayed di quest ultimo all'oscilloscopio. Triggherando sul segnale di prompt si è analizzato l'effetto dei microswitch sul segnale delayed.\\

Per evitare che il CFTD scattasse sul rumore bianco dell'elettronica è stata poi settata la soglia del modulo. Si è collegata un uscita del fan in/out all'oscilloscopio, triggherando sull'uscita delayed del CFTD. Tramite l'uso della funzione ''persistenza'' dell'oscilloscopio si è regolata la soglia facendo in modo che in corrispondenza del trigger i segnali avessero tutti un ampiezza minima che li identificasse come eventi reali. Il procedimento è stato ripetuto per il secondo rivelatore.\\

Per la calibrazione in energia si è preso uno spettro con un campione di \isotope{Co}{60}, mandando il segnale del CFTD alla coincidence unit settata in modalità ''OR'' (ovvero semplicemente il segnale stesso), che in precedenza era stata collegata all'entrate TRG IN del digitizer. Le misure sono state acquisite per 10~min su ogni rivelatore. Dato che i fotoni del decadimento del \isotope{Co}{60} hanno energie molto alte e vicine tra loro, è stato necessario acquisire anche uno spettro con un campione di \isotope{Am}{241}, che contiene un fotone di energia di 59.5~keV, per eliminare la forte correlazione che si avrebbe in caso contrario tra i parametri della retta del fit.\\

Si è poi verificato che i segnali di CFTD si trovassero effettivamente sovrapposti in presenza di una coincidenza, trovando che effettivamente lo erano, e non è stato perciò necessario cambiare il ritardo del segnale delayed.\\

In preparazione ai giorni seguenti, si è definita la geometria dell'apparato. Le distanze dalla sorgente e le aree sottese dai rivelatori sono infatti necessarie sia per una stima dell'accettanza dei rivelatori, sia per una buona analisi delle misure della correlazione angolare eseguite il terzo giorno. Subito dopo è stato preso un campione di prova con i rivelatori a 180$^\circ$ l'uno dall'altro, per ottenere una misura della rate da confrontare con quella teorica ricavabile dai parametri geometrici appena misurati.\\

Durante la seconda giornata si sono completate le misure della geometria dell'apparato ed eseguite misure riguardanti l'efficienza dei due rivelatori. Nella prima parte si è cercata la posizione della sorgente rispetto all'asse di rotazione del braccio dell'apparato contenente il rivelatore 2. Per fare ciò si è posto il trigger del digitizer sul CFTD di tale rivelatore e si sono presi campioni da 10~min l'uno facendo variare l'angolo del braccio a 0, 20, 40, 50, 70 e 90$^\circ$. Dalle differenze delle rate misurate è possibile ricavare una stima della posizione della sorgente.\\

Una volta conosciuta la struttura precisa dell'apparato si è passati a misure dell'efficienza dei rivelatori. Questa misura è stata fatta utilizzando sia il metodo dei due fotoni, sia con il metodo del picco somma. Entrambe le misure hanno richiesto run di circa 60/90~min, con il trigger sulla coincidence unit in modalità ''OR'', ma mentre nella prima si cercavano gli eventi in cui un fotone era stato rivelato dal primo e uno dal secondo rivelatore, nella seconda si cercavano gli eventi in cui entrambi i fotoni erano stati raccolti dallo stesso rivelatore. Dato che quest'ultimo evento è molto raro e nello spettro in energia si trova sommerso dal rumore si è deciso che sarebbe stata presa anche una run notturna per avere una campione dalla statistica molto alta.\\

Si sono poi cominciate a prendere le misure per la correlazione angolare dei due fotoni, poi completate il giorno seguente. Tali misure sono state prese con il trigger sulla coincidence unit in modalità ''AND'', con una durata di 10~min per ogni run, facendo variare l'angolo del braccio dell'apparato di 10 in 10$^\circ$ tra 0 e 90$^\circ$. Grazie alla misura della rate di coincidenze al variare dell'angolo, si ha una stima dei parametri della funzione di correlazione angolare.\\
