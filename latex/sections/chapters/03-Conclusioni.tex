\section{Conclusioni}

La stima della distanza della sorgente dall'asse di rotazione del primo rivelatore ha restituito un risultato in linea con le aspettative.\\

Per quanto riguarda l'efficienza, i due metodi stimano dei valori tra loro abbastanza simili, però si ritiene altamente improbabile che rivelatori come quelli usati abbiano un efficienza così bassa. Probabilmente si stanno trascurando dei fattori correttivi che andrebbero a modificare le stime fatte.\\

La verifica dell'anisotropia dell'emissione del secondo gamma è stata ben osservata ed i dati sperimentali sono compatibili con la funzione di correlazione teorica. Sfortunatamente non si ha un numero di dati sufficiente per una stima adeguata del parametro b, che si attesta a 0, sebbene rimanga compatibile con il valore teorico di 0.042.
