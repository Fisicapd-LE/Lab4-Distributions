\subsection{Correlazione}
La parte finale dell'analisi consiste nello studiare la correlazione tra le direzioni dei due gamma emessi durante il decadimento. Tali distribuzioni infatti sono entrambe funzioni dello spin del nucleo originale e sono perciò correlate. Dalla letteratura si sa che la funzione di correlazione angolare è 
$$ C(\alpha)/C(\pi/2) = 1+a*\cos^2(\alpha)+b*\cos^4(\alpha)$$

Dalle misure della rate di coincidenza ai vari angoli si è ottenuto il grafico (lo aggiungerò). Nel grafico è gia presente la correzione basata sulla posizione della sorgente rispetto all'asse di rotazione. Interpolandolo con l'espressione conosciuta di $C(\alpha)/C(\pi/2)$ si ottengono dei valori per a e b di:
$$a=0.17\pm 0.01$$
$$b=0.00\pm 0.02$$

Tale interpolazione però non tiene conto dell'angolo solido finito sotteso dai rivelatori. Ciò corrisponde al dire che la funzione di correlazione è costante all'interno del rivelatore. Data la non  linearità della funzione, questo produce un errore sistematico. Per eliminarlo si è sostituito ad ogni angolo l'angolo in cui la funzione raggiunge il valore della media integrale all'interno del rivelatore. Grazie a questa sostituzione, si è ottenuto un nuovo fit, che restituisce i parametri
$$a=roba$$
$$b=roba$$

Per dare una verifica ulteriore si è poi eseguita una simulazione monte carlo dell'evento. Le rate simulate sono mostrate nel grafico (quello bello), sovrapposte alla funzione appena interpolata.
