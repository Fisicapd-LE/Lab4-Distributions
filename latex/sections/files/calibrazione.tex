\subsection{Calibrazione}
Un primo passo necessario per la successiva analisi dati vera e propria è la calibrazione dei sistema di acquisizione, effettuata tramite una conoscenza a priori dell'energia
associata ai fotoni emessi dalla sorgente. I fotoni utilizzati a tale scopo sono quelli relativi alla cascata gamma successiva al decadimento $\beta$ del nucleo di \isotope{Co}{60}, ovvero
i gamma con energia pari a 1173 keV e 1333 keV. Per ogni rivelatore è stato quindi acquisito uno spettro in cui fossero visibili i picchi associati a tali gamma che sono stati
successivamente fittati in maniera tale da potervici associare un centroide. A questo punto avendo due coppie di valori per rivelatore sarebbe in linea di principio possibile
ottenere una relazione lineare che permette di calibrare lo spettro, ma essendo i centroidi trovati di valore molto grande $\approx 10^4$ e relativamente molto vicini, ciò
porterebbe ad una grande incertezza sul parametro di ordine 0 del fit. Si è quindi acquisito un ulteriore spettro relativo ad una sorgente di \isotope{Am}{241} 
che presenta 
un picco a 59.5 keV, ottenendo in tal modo un terzo punto per la calibrazione. Di seguito i punti ottenuti per la calibrazione ed i parametri ricavati dal fit. I risultati sono
per il rivelatore 1:
$$ m = (0.1046 \pm 0.0002) \text{keV} q = (0.6 \pm 2) \text{keV}$$
mentre per il riveltore 2 si ha:
$$ m = (0.1092 \pm 0.0009) \text{keV} q = (0.3 \pm 8) \text{keV}$$
Per quanto riguarda le interpolazioni dei singoli fotopicchi si rimanda alle appendici.

\begin{figure}[h] \centering\includegraphics[width=0.9\textwidth]{../../graphs/calib_r1.tex}\caption{Grafico della calibrazione in energia del primo canale.}\label{gr:calib_r1} \end{figure}

\begin{figure}[h] \centering\includegraphics[width=0.9\textwidth]{../../graphs/calib_r2.tex}\caption{Grafico della calibrazione in energia del secondo canale.}\label{gr:calib_r2} \end{figure}

