\subsubsection{Simulazione tramite metodo Monte Carlo}

Per avere un ulteriore analisi degli effetti dell'angolo solido finito sotteso dai rivelatori, si è sviluppato un sistema di simulazione tramite metodo Monte Carlo \footnote{Per dettagli sulla simulazione si rimanda alla relazione del positronio, in quanto i due programmi sono quasi interamente identici}. Simulando l'evento corrispondente al metodo dei due fotoni (fotone 2 nel rivelatore 2 dopo che il fotone 1 è stato visto nel rivelatore 1), si ottiene una frazione di eventi accettati pari allo 0.072 su $10^7$ eventi. Simulando invece l'evento corrispondente ad entrambi i fotoni nel rivelatore 2, si ottiene una frazione pari allo 0.0692 su $10^7$ eventi. Questi valori sono stati utilizzati per la correzione della stima dell'efficienza. Un calcolo teorico considerando il primo fotone al centro del primo rivelatore restituisce un valore di 0.034, una valore dello stesso ordine, che risulta essere una sottostima a causa del vincolo. Tale valore però conferma la verosimiglianza della stima ottenuta simulando.
