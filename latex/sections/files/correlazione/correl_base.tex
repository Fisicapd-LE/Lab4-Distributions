\subsubsection{Analisi basilare}

La parte finale dell'analisi consiste nello studiare la correlazione tra le direzioni dei due gamma emessi durante il decadimento. Tali distribuzioni infatti sono entrambe funzioni dello spin del nucleo originale e sono perciò correlate. Dalla letteratura si sa che la funzione di correlazione angolare è 
$$ C(\alpha)/C(\pi/2) = 1+a\cos^2(\alpha)+b\cos^4(\alpha)$$

Dalle misure della rate di coincidenza ai vari angoli si è ottenuto il grafico di Figura \ref{gr:correl_base}. Nel grafico è gia presente la correzione basata sulla posizione della sorgente rispetto all'asse di rotazione. Interpolandolo con l'espressione conosciuta di $C(\alpha)/C(\pi/2)$ si ottengono dei valori per a e b di:
$$a=0.17\pm 0.01$$
$$b=0.00\pm 0.02$$

\begin{tikzpicture}
\pgfdeclareplotmark{cross} {
\pgfpathmoveto{\pgfpoint{-0.3\pgfplotmarksize}{\pgfplotmarksize}}
\pgfpathlineto{\pgfpoint{+0.3\pgfplotmarksize}{\pgfplotmarksize}}
\pgfpathlineto{\pgfpoint{+0.3\pgfplotmarksize}{0.3\pgfplotmarksize}}
\pgfpathlineto{\pgfpoint{+1\pgfplotmarksize}{0.3\pgfplotmarksize}}
\pgfpathlineto{\pgfpoint{+1\pgfplotmarksize}{-0.3\pgfplotmarksize}}
\pgfpathlineto{\pgfpoint{+0.3\pgfplotmarksize}{-0.3\pgfplotmarksize}}
\pgfpathlineto{\pgfpoint{+0.3\pgfplotmarksize}{-1.\pgfplotmarksize}}
\pgfpathlineto{\pgfpoint{-0.3\pgfplotmarksize}{-1.\pgfplotmarksize}}
\pgfpathlineto{\pgfpoint{-0.3\pgfplotmarksize}{-0.3\pgfplotmarksize}}
\pgfpathlineto{\pgfpoint{-1.\pgfplotmarksize}{-0.3\pgfplotmarksize}}
\pgfpathlineto{\pgfpoint{-1.\pgfplotmarksize}{0.3\pgfplotmarksize}}
\pgfpathlineto{\pgfpoint{-0.3\pgfplotmarksize}{0.3\pgfplotmarksize}}
\pgfpathclose
\pgfusepathqstroke
}
\pgfdeclareplotmark{cross*} {
\pgfpathmoveto{\pgfpoint{-0.3\pgfplotmarksize}{\pgfplotmarksize}}
\pgfpathlineto{\pgfpoint{+0.3\pgfplotmarksize}{\pgfplotmarksize}}
\pgfpathlineto{\pgfpoint{+0.3\pgfplotmarksize}{0.3\pgfplotmarksize}}
\pgfpathlineto{\pgfpoint{+1\pgfplotmarksize}{0.3\pgfplotmarksize}}
\pgfpathlineto{\pgfpoint{+1\pgfplotmarksize}{-0.3\pgfplotmarksize}}
\pgfpathlineto{\pgfpoint{+0.3\pgfplotmarksize}{-0.3\pgfplotmarksize}}
\pgfpathlineto{\pgfpoint{+0.3\pgfplotmarksize}{-1.\pgfplotmarksize}}
\pgfpathlineto{\pgfpoint{-0.3\pgfplotmarksize}{-1.\pgfplotmarksize}}
\pgfpathlineto{\pgfpoint{-0.3\pgfplotmarksize}{-0.3\pgfplotmarksize}}
\pgfpathlineto{\pgfpoint{-1.\pgfplotmarksize}{-0.3\pgfplotmarksize}}
\pgfpathlineto{\pgfpoint{-1.\pgfplotmarksize}{0.3\pgfplotmarksize}}
\pgfpathlineto{\pgfpoint{-0.3\pgfplotmarksize}{0.3\pgfplotmarksize}}
\pgfpathclose
\pgfusepathqfillstroke
}
\pgfdeclareplotmark{newstar} {
\pgfpathmoveto{\pgfqpoint{0pt}{\pgfplotmarksize}}
\pgfpathlineto{\pgfqpointpolar{44}{0.5\pgfplotmarksize}}
\pgfpathlineto{\pgfqpointpolar{18}{\pgfplotmarksize}}
\pgfpathlineto{\pgfqpointpolar{-20}{0.5\pgfplotmarksize}}
\pgfpathlineto{\pgfqpointpolar{-54}{\pgfplotmarksize}}
\pgfpathlineto{\pgfqpointpolar{-90}{0.5\pgfplotmarksize}}
\pgfpathlineto{\pgfqpointpolar{234}{\pgfplotmarksize}}
\pgfpathlineto{\pgfqpointpolar{198}{0.5\pgfplotmarksize}}
\pgfpathlineto{\pgfqpointpolar{162}{\pgfplotmarksize}}
\pgfpathlineto{\pgfqpointpolar{134}{0.5\pgfplotmarksize}}
\pgfpathclose
\pgfusepathqstroke
}
\pgfdeclareplotmark{newstar*} {
\pgfpathmoveto{\pgfqpoint{0pt}{\pgfplotmarksize}}
\pgfpathlineto{\pgfqpointpolar{44}{0.5\pgfplotmarksize}}
\pgfpathlineto{\pgfqpointpolar{18}{\pgfplotmarksize}}
\pgfpathlineto{\pgfqpointpolar{-20}{0.5\pgfplotmarksize}}
\pgfpathlineto{\pgfqpointpolar{-54}{\pgfplotmarksize}}
\pgfpathlineto{\pgfqpointpolar{-90}{0.5\pgfplotmarksize}}
\pgfpathlineto{\pgfqpointpolar{234}{\pgfplotmarksize}}
\pgfpathlineto{\pgfqpointpolar{198}{0.5\pgfplotmarksize}}
\pgfpathlineto{\pgfqpointpolar{162}{\pgfplotmarksize}}
\pgfpathlineto{\pgfqpointpolar{134}{0.5\pgfplotmarksize}}
\pgfpathclose
\pgfusepathqfillstroke
}
\definecolor{c}{rgb}{1,1,1};
\draw [color=c, fill=c] (0,0) rectangle (20,11.0085);
\draw [color=c, fill=c] (2,1.10085) rectangle (18,9.90762);
\definecolor{c}{rgb}{0,0,0};
\draw [c,line width=0.9] (2,1.10085) -- (2,9.90762) -- (18,9.90762) -- (18,1.10085) -- (2,1.10085);
\definecolor{c}{rgb}{1,1,1};
\draw [color=c, fill=c] (2,1.10085) rectangle (18,9.90762);
\definecolor{c}{rgb}{0,0,0};
\draw [c,line width=0.9] (2,1.10085) -- (2,9.90762) -- (18,9.90762) -- (18,1.10085) -- (2,1.10085);
\draw [c,line width=0.9] (2,1.10085) -- (18,1.10085);
\draw [anchor= east] (18,0.484373) node[scale=0.854877, color=c, rotate=0]{$\cos(\theta)$};
\draw [c,line width=0.9] (2,1.36505) -- (2,1.10085);
\draw [c,line width=0.9] (2.72738,1.23295) -- (2.72738,1.10085);
\draw [c,line width=0.9] (3.45476,1.23295) -- (3.45476,1.10085);
\draw [c,line width=0.9] (4.18215,1.23295) -- (4.18215,1.10085);
\draw [c,line width=0.9] (4.90953,1.36505) -- (4.90953,1.10085);
\draw [c,line width=0.9] (5.63691,1.23295) -- (5.63691,1.10085);
\draw [c,line width=0.9] (6.36429,1.23295) -- (6.36429,1.10085);
\draw [c,line width=0.9] (7.09167,1.23295) -- (7.09167,1.10085);
\draw [c,line width=0.9] (7.81906,1.36505) -- (7.81906,1.10085);
\draw [c,line width=0.9] (8.54644,1.23295) -- (8.54644,1.10085);
\draw [c,line width=0.9] (9.27382,1.23295) -- (9.27382,1.10085);
\draw [c,line width=0.9] (10.0012,1.23295) -- (10.0012,1.10085);
\draw [c,line width=0.9] (10.7286,1.36505) -- (10.7286,1.10085);
\draw [c,line width=0.9] (11.456,1.23295) -- (11.456,1.10085);
\draw [c,line width=0.9] (12.1833,1.23295) -- (12.1833,1.10085);
\draw [c,line width=0.9] (12.9107,1.23295) -- (12.9107,1.10085);
\draw [c,line width=0.9] (13.6381,1.36505) -- (13.6381,1.10085);
\draw [c,line width=0.9] (14.3655,1.23295) -- (14.3655,1.10085);
\draw [c,line width=0.9] (15.0929,1.23295) -- (15.0929,1.10085);
\draw [c,line width=0.9] (15.8203,1.23295) -- (15.8203,1.10085);
\draw [c,line width=0.9] (16.5476,1.36505) -- (16.5476,1.10085);
\draw [c,line width=0.9] (16.5476,1.36505) -- (16.5476,1.10085);
\draw [c,line width=0.9] (17.275,1.23295) -- (17.275,1.10085);
\draw [anchor=base] (2,0.737567) node[scale=0.854877, color=c, rotate=0]{0};
\draw [anchor=base] (4.90953,0.737567) node[scale=0.854877, color=c, rotate=0]{0.2};
\draw [anchor=base] (7.81906,0.737567) node[scale=0.854877, color=c, rotate=0]{0.4};
\draw [anchor=base] (10.7286,0.737567) node[scale=0.854877, color=c, rotate=0]{0.6};
\draw [anchor=base] (13.6381,0.737567) node[scale=0.854877, color=c, rotate=0]{0.8};
\draw [anchor=base] (16.5476,0.737567) node[scale=0.854877, color=c, rotate=0]{1};
\draw [c,line width=0.9] (2,1.10085) -- (2,9.90762);
\draw [anchor= east] (0.88,9.90762) node[scale=0.854877, color=c, rotate=90]{$C(\theta)$};
\draw [c,line width=0.9] (2.48,1.83946) -- (2,1.83946);
\draw [c,line width=0.9] (2.24,2.09065) -- (2,2.09065);
\draw [c,line width=0.9] (2.24,2.34184) -- (2,2.34184);
\draw [c,line width=0.9] (2.24,2.59303) -- (2,2.59303);
\draw [c,line width=0.9] (2.24,2.84422) -- (2,2.84422);
\draw [c,line width=0.9] (2.48,3.09541) -- (2,3.09541);
\draw [c,line width=0.9] (2.24,3.3466) -- (2,3.3466);
\draw [c,line width=0.9] (2.24,3.59779) -- (2,3.59779);
\draw [c,line width=0.9] (2.24,3.84898) -- (2,3.84898);
\draw [c,line width=0.9] (2.24,4.10017) -- (2,4.10017);
\draw [c,line width=0.9] (2.48,4.35137) -- (2,4.35137);
\draw [c,line width=0.9] (2.24,4.60256) -- (2,4.60256);
\draw [c,line width=0.9] (2.24,4.85374) -- (2,4.85374);
\draw [c,line width=0.9] (2.24,5.10494) -- (2,5.10494);
\draw [c,line width=0.9] (2.24,5.35613) -- (2,5.35613);
\draw [c,line width=0.9] (2.48,5.60732) -- (2,5.60732);
\draw [c,line width=0.9] (2.24,5.85851) -- (2,5.85851);
\draw [c,line width=0.9] (2.24,6.1097) -- (2,6.1097);
\draw [c,line width=0.9] (2.24,6.36089) -- (2,6.36089);
\draw [c,line width=0.9] (2.24,6.61208) -- (2,6.61208);
\draw [c,line width=0.9] (2.48,6.86327) -- (2,6.86327);
\draw [c,line width=0.9] (2.24,7.11446) -- (2,7.11446);
\draw [c,line width=0.9] (2.24,7.36565) -- (2,7.36565);
\draw [c,line width=0.9] (2.24,7.61684) -- (2,7.61684);
\draw [c,line width=0.9] (2.24,7.86803) -- (2,7.86803);
\draw [c,line width=0.9] (2.48,8.11922) -- (2,8.11922);
\draw [c,line width=0.9] (2.24,8.37041) -- (2,8.37041);
\draw [c,line width=0.9] (2.24,8.6216) -- (2,8.6216);
\draw [c,line width=0.9] (2.24,8.87279) -- (2,8.87279);
\draw [c,line width=0.9] (2.24,9.12398) -- (2,9.12398);
\draw [c,line width=0.9] (2.48,9.37517) -- (2,9.37517);
\draw [c,line width=0.9] (2.48,1.83946) -- (2,1.83946);
\draw [c,line width=0.9] (2.24,1.58827) -- (2,1.58827);
\draw [c,line width=0.9] (2.24,1.33708) -- (2,1.33708);
\draw [c,line width=0.9] (2.48,9.37517) -- (2,9.37517);
\draw [c,line width=0.9] (2.24,9.62636) -- (2,9.62636);
\draw [c,line width=0.9] (2.24,9.87755) -- (2,9.87755);
\draw [anchor= east] (1.9,1.83946) node[scale=0.854877, color=c, rotate=0]{2800};
\draw [anchor= east] (1.9,3.09541) node[scale=0.854877, color=c, rotate=0]{2900};
\draw [anchor= east] (1.9,4.35137) node[scale=0.854877, color=c, rotate=0]{3000};
\draw [anchor= east] (1.9,5.60732) node[scale=0.854877, color=c, rotate=0]{3100};
\draw [anchor= east] (1.9,6.86327) node[scale=0.854877, color=c, rotate=0]{3200};
\draw [anchor= east] (1.9,8.11922) node[scale=0.854877, color=c, rotate=0]{3300};
\draw [anchor= east] (1.9,9.37517) node[scale=0.854877, color=c, rotate=0]{3400};
\foreach \P in {(16.5472,8.45008), (16.3464,6.86826), (15.709,7.22063), (14.6529,6.10732), (13.2085,5.52819), (11.4189,5.18769), (9.33815,4.64236), (7.02947,4.01833), (4.56442,2.51106), (2.01904,2.61316)}{\draw[mark options={color=c,fill=c},mark
 size=2.402402pt,mark=o] plot coordinates {\P};}
\definecolor{c}{rgb}{1,0,0};
\draw [c,line width=1.8] (2.08,2.89929) -- (2.24,2.90045) -- (2.4,2.90277) -- (2.56,2.90624) -- (2.72,2.91087) -- (2.88,2.91666) -- (3.04,2.92361) -- (3.2,2.93172) -- (3.36,2.94098) -- (3.52,2.95141) -- (3.68,2.96299) -- (3.84,2.97573) -- (4,2.98962)
 -- (4.16,3.00468) -- (4.32,3.02089) -- (4.48,3.03826) -- (4.64,3.05679) -- (4.8,3.07648) -- (4.96,3.09732) -- (5.12,3.11933) -- (5.28,3.14249) -- (5.44,3.16681) -- (5.6,3.19228) -- (5.76,3.21892) -- (5.92,3.24671) -- (6.08,3.27566) -- (6.24,3.30577)
 -- (6.4,3.33704) -- (6.56,3.36947) -- (6.72,3.40305) -- (6.88,3.43779) -- (7.04,3.47369) -- (7.2,3.51075) -- (7.36,3.54897) -- (7.52,3.58834) -- (7.68,3.62887) -- (7.84,3.67056) -- (8,3.71341) -- (8.16,3.75742) -- (8.32,3.80258) -- (8.48,3.84891) --
 (8.64,3.89639) -- (8.8,3.94503) -- (8.96,3.99482) -- (9.12,4.04578) -- (9.28,4.09789) -- (9.44,4.15116) -- (9.6,4.20559) -- (9.76,4.26118) -- (9.92,4.31792);
\draw [c,line width=1.8] (9.92,4.31792) -- (10.08,4.37583) -- (10.24,4.43489) -- (10.4,4.49511) -- (10.56,4.55649) -- (10.72,4.61902) -- (10.88,4.68271) -- (11.04,4.74757) -- (11.2,4.81358) -- (11.36,4.88074) -- (11.52,4.94907) -- (11.68,5.01855) --
 (11.84,5.0892) -- (12,5.161) -- (12.16,5.23395) -- (12.32,5.30807) -- (12.48,5.38334) -- (12.64,5.45978) -- (12.8,5.53737) -- (12.96,5.61612) -- (13.12,5.69602) -- (13.28,5.77709) -- (13.44,5.85931) -- (13.6,5.94269) -- (13.76,6.02723) --
 (13.92,6.11293) -- (14.08,6.19978) -- (14.24,6.28779) -- (14.4,6.37696) -- (14.56,6.46729) -- (14.72,6.55878) -- (14.88,6.65143) -- (15.04,6.74523) -- (15.2,6.84019) -- (15.36,6.93631) -- (15.52,7.03359) -- (15.68,7.13202) -- (15.84,7.23162) --
 (16,7.33237) -- (16.16,7.43428) -- (16.32,7.53735) -- (16.48,7.64157) -- (16.64,7.74696) -- (16.8,7.8535) -- (16.96,7.9612) -- (17.12,8.07006) -- (17.28,8.18007) -- (17.44,8.29125) -- (17.6,8.40358) -- (17.76,8.51707);
\draw [c,line width=1.8] (17.76,8.51707) -- (17.92,8.63172);
\definecolor{c}{rgb}{0,0,0};
\draw [c,line width=0.9] (16.5472,8.45008) -- (16.5472,9.17372);
\draw [c,line width=0.9] (16.5164,9.17372) -- (16.578,9.17372);
\draw [c,line width=0.9] (16.5472,8.45008) -- (16.5472,7.72644);
\draw [c,line width=0.9] (16.5164,7.72644) -- (16.578,7.72644);
\draw [c,line width=0.9] (16.3464,6.86826) -- (16.3464,7.57949);
\draw [c,line width=0.9] (16.3156,7.57949) -- (16.3772,7.57949);
\draw [c,line width=0.9] (16.3464,6.86826) -- (16.3464,6.15703);
\draw [c,line width=0.9] (16.3156,6.15703) -- (16.3772,6.15703);
\draw [c,line width=0.9] (15.709,7.22063) -- (15.709,7.93497);
\draw [c,line width=0.9] (15.6783,7.93497) -- (15.7398,7.93497);
\draw [c,line width=0.9] (15.709,7.22063) -- (15.709,6.50629);
\draw [c,line width=0.9] (15.6783,6.50629) -- (15.7398,6.50629);
\draw [c,line width=0.9] (14.6529,6.10732) -- (14.6529,6.81319);
\draw [c,line width=0.9] (14.6221,6.81319) -- (14.6837,6.81319);
\draw [c,line width=0.9] (14.6529,6.10732) -- (14.6529,5.40144);
\draw [c,line width=0.9] (14.6221,5.40144) -- (14.6837,5.40144);
\draw [c,line width=0.9] (13.2085,5.52819) -- (13.2085,6.22957);
\draw [c,line width=0.9] (13.1777,6.22957) -- (13.2393,6.22957);
\draw [c,line width=0.9] (13.2085,5.52819) -- (13.2085,4.82681);
\draw [c,line width=0.9] (13.1777,4.82681) -- (13.2393,4.82681);
\draw [c,line width=0.9] (11.4189,5.18769) -- (11.4189,5.88669);
\draw [c,line width=0.9] (11.3881,5.88669) -- (11.4497,5.88669);
\draw [c,line width=0.9] (11.4189,5.18769) -- (11.4189,4.48869);
\draw [c,line width=0.9] (11.3881,4.48869) -- (11.4497,4.48869);
\draw [c,line width=0.9] (9.33815,4.64236) -- (9.33815,5.33709);
\draw [c,line width=0.9] (9.30736,5.33709) -- (9.36894,5.33709);
\draw [c,line width=0.9] (9.33815,4.64236) -- (9.33815,3.94763);
\draw [c,line width=0.9] (9.30736,3.94763) -- (9.36894,3.94763);
\draw [c,line width=0.9] (7.02947,4.01833) -- (7.02947,4.70802);
\draw [c,line width=0.9] (6.99868,4.70802) -- (7.06026,4.70802);
\draw [c,line width=0.9] (7.02947,4.01833) -- (7.02947,3.32863);
\draw [c,line width=0.9] (6.99868,3.32863) -- (7.06026,3.32863);
\draw [c,line width=0.9] (4.56442,2.51106) -- (4.56442,3.18737);
\draw [c,line width=0.9] (4.53363,3.18737) -- (4.59521,3.18737);
\draw [c,line width=0.9] (4.56442,2.51106) -- (4.56442,1.83474);
\draw [c,line width=0.9] (4.53363,1.83474) -- (4.59521,1.83474);
\draw [c,line width=0.9] (2.01904,2.61316) -- (2.01904,3.29044);
\draw [c,line width=0.9] (2,3.29044) -- (2.04984,3.29044);
\draw [c,line width=0.9] (2.01904,2.61316) -- (2.01904,1.93589);
\draw [c,line width=0.9] (2,1.93589) -- (2.04984,1.93589);
\end{tikzpicture}

