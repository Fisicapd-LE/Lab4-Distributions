\subsection{Analisi preliminare del segnale}
 Prima di iniziare con l'acquisizione dei dati veri e propri si è proceduto con un'analisi preliminare dei segnali forniti dai due rivelatori. A tal fine si è visualizzata 
 l'uscita prompt del CFD sull'oscilloscopio e si è analizzato uno dei segnali disponibili per ogniuno dei due rivelatori. Le caratteristiche prese in esame sono
 Ampiezza del segnale, tempo di salita (allontanamento dalla baseline) e tempo di discesa del segnale e l'entità del rumore elettronico, stimato in media e senza errore. 
 I dati ottenuti sono i seguenti:

\begin{table} 
	\begin{center}
		\begin{tabulary}{\textwidth}{CCCCCCCC}
		\toprule
		Rivelatore &	 Ampiezza [mV] &	Errore [mV] &	Tempo salita [ns] &	Errore [ns] &	Tempo discesa [ns] &	Errore [ns] 	& Rumore 		Elettronico [mV]\\ \midrule
		1 &	324 &	9 &	29 &	2 &	580 &	6	& 2\\ \midrule
		2 &	280 &	6 &	34 &	0.6 &	552 &	6 	& 1.6\\
		\bottomrule
		\end{tabulary}
	\end{center}
\end{table}
