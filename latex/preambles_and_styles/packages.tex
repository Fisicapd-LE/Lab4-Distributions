\usepackage{tikz}
\usetikzlibrary{spy}
\usepackage{pgfplots}

\usepackage{lipsum} % Package to generate dummy text throughout this template

%\usepackage[sc]{mathpazo} % Use the Palatino font
\usepackage{tgpagella} % TeX Gyre Pagella, versione migliorata di Palatino. Si ma bo, no
\usepackage{inconsolata} % Font monospace
%\usepackage{textcomp}
\usepackage[scale=0.98,ttdefault]{AnonymousPro}

%%%%%
%\usepackage{Alegreya} %% Option 'black' gives heavier bold face 
\usepackage{AlegreyaSans} %% Option 'black' gives heavier bold face 
%\renewcommand*\oldstylenums[1]{{\AlegreyaOsF #1}}
%\usepackage{opensans}
%\usepackage[euler-digits,euler-hat-accent]{eulervm}
\usepackage[euler-hat-accent]{eulervm}
%%%%%%

\usepackage[T1]{fontenc} % Use 8-bit encoding that has 256 glyphs
%\usepackage{fontspec}
%\setmainfont{tgpagella}
%\setsansfont{AlegreyaSans}
%\setmonofont{inconsolata}

\usepackage[utf8]{inputenc} % Consente l'uso caratteri accentati italiani
\linespread{1.08} % Line spacing - Palatino needs more space between lines, messo a 1.08 da 1.11 che era per alegreya
\usepackage{amsmath, amsthm, amssymb, amsfonts}
\usepackage[italian]{babel}
%\usepackage[kerning,spacing,tracking,letterspace = 2,babel]{microtype} % Slightly tweak font spacing for aesthetics. Il tre è pensato per Alegreya
\usepackage[kerning,spacing,babel]{microtype}
\SetTracking[]{encoding = *,shape = *}{3} % Aumenta la distanza fra le lettere
							     % http://tex.stackexchange.com/questions/66494/new-command-for-spacing-letters-in-microtype
%%%%%%%%%%%%%%%%%%%%%%%%%%%%%%%%%%%%%%%%%%%%%
%Miei package



\usepackage{graphicx}		% Per le immagini
%\usepackage{color}		% COLORI!

%\definecolor{grigio-molto-scuro}{gray}{0.1}	%colore

\usepackage{tabularx}		% Per le tabelle con le colonne tutte uguali
\usepackage{tabulary}		% Tabelle migliorate, nelle celle il testo va a capo da solo...
%%%%%%%%%%%%%%%%%%%%%%%%%%%%%%%%%%%%%%%%%%%%%
\newlength{\alphabet}
\settowidth{\alphabet}{\normalfont abcdefghijklmnopqrstuvwxyz}
\usepackage[
	    %hmargin=0.18\paperwidth,% metti la larghezza del testo (margini orizzontali) al 18% del foglio
	    textwidth=3.3\alphabet,  % http://tex.stackexchange.com/questions/59626/nicely-force-66-characters-per-line
	    hmarginratio=1:1,       % margini destro e sinistro uguali
	    top=35mm,	            % margine sopra a 32mm...
	    vmarginratio=4:5,       % quello sotto uguale (default 2:3)
	    columnsep=20pt]         % Spazio tra le colonne?
	    {geometry} % Document margins
\usepackage{multicol} % Used for the two-column layout of the document
\usepackage[hang, small,labelfont=bf,up,textfont=it,up]{caption} % Custom captions under/above floats in tables or figures
\usepackage{booktabs} % Horizontal rules in tables
\usepackage{float} % Required for tables and figures in the multi-column environment - they need to be placed in specific locations with the [H] (e.g. \begin{table}[H])
%\usepackage{tocloft} % Per customizzare le liste di floats (per i custom float!)
\usepackage[titles]{tocloft} %Pare causi meno casini con fancyhdr
\usepackage{nicefrac} % Per le frazioni tipo ⅛
\usepackage{pdfpages} % Per includere pagine intere in pdf (per la copertina)

\usepackage{lettrine} % The lettrine is the first enlarged letter at the beginning of the text
\usepackage{paralist} % Used for the compactitem environment which makes bullet points with less space between them
\usepackage[section]{placeins} % Per \FloatBarrier. L'opzione section comporta che le sezioni siano floatbarriers

\usepackage{abstract} % Allows abstract customization
\renewcommand{\abstractnamefont}{\normalfont\bfseries} % Set the "Abstract" text to bold
\renewcommand{\abstracttextfont}{\normalfont\small\itshape} % Set the abstract itself to small italic text

\usepackage{caption} % Per captions avanzate

\usepackage{listingsutf8} % Per includere codice sorgente meglio che con verbatim (e con caratteri non inglesi)
\lstset{ 
  %Preso anche questo da http://en.wikibooks.org/wiki/LaTeX/Source_Code_Listings
  %backgroundcolor=\color{white},   % choose the background color; you must add \usepackage{color} or \usepackage{xcolor}
  basicstyle=\footnotesize\ttfamily,        % the size of the fonts that are used for the code E MESSO IN MONOSPACE
  breakatwhitespace=true,         % sets if automatic breaks should only happen at whitespace
  breaklines=true,                 % sets automatic line breaking
  captionpos=b,                    % sets the caption-position to bottom
  %commentstyle=\color{mygreen},    % comment style
  %deletekeywords={...},            % if you want to delete keywords from the given language
  %escapeinside={\%*}{*)},          % if you want to add LaTeX within your code
  %extendedchars=true,              % lets you use non-ASCII characters; for 8-bits encodings only, does not work with UTF-8
  frame=l,                    % adds a frame around the code
				    %you can control the rules at the top, right, bottom, and left directly by using the four initial 
				    %letters for single rules and their upper case versions for double rules. http://mirror.hmc.edu/ctan/macros/latex/contrib/listings/listings.pdf
				    % Es frame frame=trBL ha doppia linea a sinistra e sotto, e singola a destra e sopra
  keepspaces=true,                 % keeps spaces in text, useful for keeping indentation of code (possibly needs columns=flexible)
  %keywordstyle=\color{blue},       % keyword style
  %language=Octave,                 % the language of the code
  %morekeywords={*,...},            % if you want to add more keywords to the set
  numbers=left,                    % where to put the line-numbers; possible values are (none, left, right)
  numbersep=5pt,                   % how far the line-numbers are from the code
  %numberstyle=\tiny\color{mygray}, % the style that is used for the line-numbers
  %rulecolor=\color{black},         % if not set, the frame-color may be changed on line-breaks within not-black text (e.g. comments (green here))
  showspaces=false,                % show spaces everywhere adding particular underscores; it overrides 'showstringspaces'
  showstringspaces=false,          % underline spaces within strings only
  showtabs=false,                  % show tabs within strings adding particular underscores
  stepnumber=1,                    % the step between two line-numbers. If it's 1, each line will be numbered
  %stringstyle=\color{mymauve},     % string literal style
  tabsize=2,                       % sets default tabsize to 2 spaces
  title=\lstname                   % show the filename of files included with \lstinputlisting; also try caption instead of title
}


\usepackage{titlesec} % Allows customization of titles
%\renewcommand\thesection{\Roman{section}} % Roman numerals for the sections
%\renewcommand\thesubsection{\Roman{subsection}} % Roman numerals for subsections
% \usefont {encoding} {family} {series} {shape}
%\AlegreyaSansSC
\titleformat{\section}[block]{ \bfseries \LARGE}{\thesection.}{1em}{} % Change the look of the section titles. Pezzi spostati \scshape\centering\bfseries
\titleformat{\subsection}[block]{\bfseries \Large}{\thesection.\thesubsection }{1em}{} % Change the look of the section titles

\usepackage{fancyhdr} % Headers and footers
\pagestyle{fancy} % All pages have headers and footers
\fancyhead{} % Blank out the default header
\fancyfoot{} % Blank out the default footer
\headheight=14pt % Perchè sennò continua a lamentarsi che 12pt è troppo poco e la mette a 14 lo stesso
%\fancyhead[C]{Chiappara, Labanca, Forcher - \textit{Ottica geometrica} $\bullet$ \thesection}
\fancyhead[L]{\textit{Gruppo 8}} % Custom header text. \nouppercase{\leftmark} per sezione, ma non ci sta
\fancyhead[R]{\textsc{\nouppercase{\leftmark}}}
\fancyfoot[RO,LE]{\thepage} % Custom footer text

\usepackage[hidelinks]{hyperref} % For hyperlinks in the PDF
\hypersetup{
    bookmarks=true,         % show bookmarks bar?
    unicode=false,          % non-Latin characters in Acrobat’s bookmarks
   % pdftoolbar=true,        % show Acrobat’s toolbar?
   % pdfmenubar=true,        % show Acrobat’s menu?
   % pdffitwindow=false,     % window fit to page when opened
    %pdfstartview={FitH},    % fits the width of the page to the window
    pdftitle={Relazione di laboratorio: Distribuzioni angolari},    % title
    pdfauthor={D.Chiappara, I. Di Terlizzi, E. Lusiani},     % author
    pdfsubject={Relazione di laboratorio},   % subject of the document
    %pdfcreator={Creator},   % creator of the document
    %pdfproducer={Producer}, % producer of the document
    %pdfkeywords={photomultipliers} {tesi} {fotomoltiplicatori} {JUNO}, % list of keywords
    %pdfnewwindow=true,      % links in new PDF window
    %colorlinks=false,       % false: boxed links; true: colored links
    linkcolor=red,          % color of internal links (change box color with linkbordercolor)
    citecolor=green,        % color of links to bibliography
    filecolor=magenta,      % color of file links
    urlcolor=cyan           % color of external links
}

\usepackage{etoolbox}

\usepackage{textgreek}

\usepackage{standalone}

%\usepackage{circuitikz}

\usepackage{siunitx}

\usepackage{subcaption}

%\extrafloats{100}
